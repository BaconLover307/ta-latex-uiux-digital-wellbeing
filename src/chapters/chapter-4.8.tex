\section{Kelayakan Implementasi}
\label{sec:kelayakan}

Setelah prototipe desain interaksi ditentukan telah memenuhi kebutuhan, desain tersebut perlu dipastikan kelayakan implementasinya. Hal ini dilakukan untuk memastikan bahwa prototipe desain interaksi memungkinkan untuk diimplementasi menjadi sebuah aplikasi.
% Menurut \textcite{gog2021feasibility}, pengujian kelayakan implementasi adalah salah satu dari 4 tahap dari Software Project Management Process yang berguna untuk menguji apakah suatu sistem atau produk perangkat lunak dapat diimplementasi.

Pertama-tama ditentukan terlebih dahulu beberapa spesifikasi dari perangkat yang mampu menjalankan aplikasi. Spesifikasi ini mengacu pada lingkup permasalahan pada Bab \ref{sec:perancangan_proses_desain} serta halaman spesifikasi aplikasi pada Google Play Store. Berikut adalah spesifikasi yang diperlukan sebuah perangkat untuk menjalankan aplikasi
% Mengacu pada lingkup permasalahan pada Bab \ref{sec:perancangan_proses_desain} serta halaman spesifikasi aplikasi pada Google Play Store \href{https://play.google.com/store/apps/details?id=com.google.android.apps.wellbeing&hl=en&gl=ID}, maka spesifikasi minimal yang perlu dimiliki sebuah perangkat agar dapat menjalankan prototipe desain interaksi solusi yang diimplementasi sebagai bagian dari kelayakan implementasi adalah sebagai berikut

\begin{enumerate}
  \item Platform: Android
  \item Sistem Operasi: minimal Android 9 Pie
\end{enumerate}

Selain perangkat, diperlukan beberapa kakas dalam mengimplementasi aplikasi. Kakas pengembangan yang dapat digunakan Android Studio berbasis bahasa pemrograman Java dengan versi kakas yang mampu mengembangkan aplikasi Android versi 9. Kakas ini dapat disebut sebagai kakas paling tepat untuk mengembangkan aplikasi karena Android Studio adalah kakas buatan Google sendiri yang dianjurkan dalam mengembangkan aplikasi Android \textit{native}. Kakas ini dapat mengatur bagian \textit{front-end} maupun \textit{back-end} dari aplikasi. Adapun database yang dapat digunakan adalah SQLite, yang merupakan implementasi database bawaan dari Android.

Pembuktian kelayakan implementasi aplikasi dilakukan dengan menjelaskan kelayakan implementasi dari fitur-fitur yang tertera pada Tabel \ref{tab:daftar_fitur}. Terdapat beberapa fitur dari prototipe yang tidak perlu dijelaskan kembali kelayakan implementasinya karena fitur-fitur tersebut sudah terbukti dapat diimplementasi pada aplikasi Digital Wellbeing milik Google dan tidak mengalami modifikasi. Fitur-fitur yang dimaksud adalah sebagai berikut

\begin{enumerate}
  \item F-01 Usage Tracker
  \item F-02 Pie Chart
  \item F-03 Bar Chart
  \item F-08 Daftar Aplikasi
  \item F-14 Bedtime Mode
  \item F-15 Greyscale Screen
  \item F-16 Do Not Disturb
\end{enumerate}

Selain daftar di atas, berikut adalah penjelasan tentang kelayakan implementasi dari fitur-fitur prototipe desain interaksi solusi

\begin{enumerate}
  \item \textbf{F-05} Searchbar
  \subitem
    
  \item \textbf{F-0} App Group
  \subitem
    
  \item \textbf{F-0} Daftar Jadwal Aktivasi
  \subitem
    
  \item \textbf{F-0} Daily Goal
  \subitem
    
  \item \textbf{F-0} fitur
  \subitem
    
  \item \textbf{F-0} fitur
  \subitem
    
  \item \textbf{F-0} fitur
  \subitem
    
  \item \textbf{F-0} fitur
  \subitem
    
  \item \textbf{F-0} fitur
  \subitem
    
  \item \textbf{F-0} fitur
  \subitem
    
  \item \textbf{F-0} fitur
  \subitem
    
  \item \textbf{F-0} fitur
  \subitem
    
  \item \textbf{F-0} fitur
  \subitem
    
  \item \textbf{F-0} fitur
  \subitem
    
\end{enumerate}