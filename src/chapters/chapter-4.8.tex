\section{Kelayakan Implementasi}
\label{sec:kelayakan}

Setelah prototipe desain interaksi ditentukan telah memenuhi kebutuhan, desain tersebut perlu diuji kelayakannya dengan menguji kelayakan implementasi.
% Menurut \textcite{gog2021feasibility}, pengujian kelayakan implementasi adalah salah satu dari 4 tahap dari Software Project Management Process yang berguna untuk menguji apakah suatu sistem atau produk perangkat lunak dapat diimplementasi.

Dalam pengerjaan tugas akhir ini, pengujian kelayakan implementasi difokuskan untuk menguji apakah desain memungkinkan untuk diimplementasi dengan menentukan hal-hal yang diperlukan dalam proses implementasi. Kelayakan implementasi ini dibagi menjadi tahap penentuan lingkungan pengembangan, dan tahap pengujian fitur.

\subsection{Penentuan Lingkungan Pengembangan}

Mengacu pada lingkup permasalahan pada Bab \ref{sec:perancangan_proses_desain} serta halaman spesifikasi aplikasi pada Google Play Store, maka spesifikasi minimal yang perlu dimiliki sebuah perangkat agar dapat menjalankan prototipe desain interaksi solusi yang diimplementasi sebagai bagian dari kelayakan implementasi adalah sebagai berikut
% Mengacu pada lingkup permasalahan pada Bab \ref{sec:perancangan_proses_desain} serta halaman spesifikasi aplikasi pada Google Play Store \href{https://play.google.com/store/apps/details?id=com.google.android.apps.wellbeing&hl=en&gl=ID}, maka spesifikasi minimal yang perlu dimiliki sebuah perangkat agar dapat menjalankan prototipe desain interaksi solusi yang diimplementasi sebagai bagian dari kelayakan implementasi adalah sebagai berikut

\begin{enumerate}
  \item Platform: Android
  \item Sistem Operasi: Android 9 Pie
\end{enumerate}

Dalam proses kelayakan implementasi, untuk keperluan menguji kelayakan fitur, digunakan kakas pengembangan Android Studio versi Chipmunk (2021.2.1) untuk mengimplementasinya. Aplikasi yang dikembangkan untuk keperluan kelayakan implementasi diberikan nama “Digital Wellbeing Plus” sebagai pembeda dengan aplikasi dasar pengembangan Digital Wellbeing milik Google. Adapun perangkat yang digunakan untuk pengujian memiliki spesifikasi sebagai berikut

\begin{enumerate}
  \item Model: Xiaomi Pocophone F1
  \item RAM: 6.00GB
  \item CPU: Octa-core Max 2.8 GHz
  \item Versi Android: Android 10
  \item Sistem Operasi: MIUI Global 12.0.3
  \item Storage: 128GB
\end{enumerate}

\subsection{Pengujian Fitur}

Bagian ini mengacu pada daftar fitur yang tertera pada Tabel \ref{tab:daftar_fitur}. Proses ini akan menjelaskan bagaimana fitur yang didesain memungkinkan untuk diimplementasi, baik dengan cara mengambil referensi dari aplikasi sejenis yang mengimplementasi fitur yang serupa, atau mengimplementasinya langsung.

Fitur-fitur yang tertera pada daftar dengan tipe "Sudah ada" tidak akan diuji kelayakannya, berhubungan fitur-fitur tersebut telah terbukti dapat diimplementasi pada aplikasi Digital Wellbeing milik Google, dan tidak mengalami modifikasi. Fitur-fitur yang dimaksud adalah sebagai berikut

\begin{enumerate}
  \item F-01 Usage Tracker
  \item F-02 Pie Chart
  \item F-03 Bar Chart
  \item F-08 Daftar Aplikasi
  \item F-14 Bedtime Mode
  \item F-15 Greyscale Screen
  \item F-16 Do Not Disturb  
\end{enumerate}

Berikut adalah penjelasan untuk setiap fitur yang diuji kelayakannya