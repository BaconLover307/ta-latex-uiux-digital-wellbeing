\chapter{Implementasi dan Pengujian Prototipe}

Bab Implementasi dan Pengujian Prototipe berisi tentang lanjutan dari metodologi \textit{User-Centered Design}, yaitu tahap perancangan serta evaluasi prototipe perangkat lunak. Kedua tahap tersebut dilakukan iterasi sesuai dengan kebutuhan. Perancangan dan evaluasi akan dilakukan pada \textit{low-fidelity prototype} berbentuk \textit{wireframe} dan \textit{high-fidelity prototype} berbentuk prototipe \textit{mobile application} pada perangkat Android. Bagian perancangan prototipe menjelaskan tentang proses implementasi dari prototipe, sedangkan bagian evaluasi akan menjelaskan tentang proses pengujian prototipe yang berisi skenario dan hasil pengujian yang telah dicocokan dengan \textit{usability goals} dan \textit{user experience goals} yang sudah ditentukan sebelumnya. 

\section{Pengembangan Prototipe \textit{Low-Fidelity}}
\blindtext

\section{Pengujian Prototipe \textit{Low-Fidelity}}
\blindtext

\section{Pengembangan Prototipe \textit{High-Fidelity}}
\blindtext

\section{Pengujian Prototipe \textit{High-Fidelity}}
\blindtext