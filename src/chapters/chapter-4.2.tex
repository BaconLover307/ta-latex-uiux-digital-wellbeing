\section{Pengujian Prototipe \textit{Low-Fidelity}}
Rancangan prototipe \textit{Low-Fidelity} perlu diuji untuk memastikan bahwa fungsionalitas inti yang telah direalisasikan pada prototipe dapat dipahami pengguna dengan benar, serta memastikan bahwa alur navigasi prototipe sesuai dengan Diagram Navigasi pada Gambar \ref{fig:diagram_navigasi} yang telah dibuat. Pengujian pada tahap ini belum melibatkan \textit{usability goals} dan \textit{user experience goals} karena terbatasnya elemen-elemen visual dan interaksi yang mendukung dalam mencapai tujuan-tujuan tersebut. Pengujian ini akan dilakukan kepada lima orang partisipan yang termasuk ke dalam persona primer. Hal ini sesuai dengan perkataan dari \textcite{nielsenusabilityproblems} yang menyebutkan bahwa pengujian dengan 5 orang sudah cukup untuk menemukan rata-rata 85\% masalah dari desain, dalam hal ini dari prototipe \textit{low-fidelity}.

% Pengujian ini akan dilakukan kepada 5 orang pengguna, yang terdiri dari 3 orang yang termasuk ke dalam Persona 2 (Maya) dan 2 orang yang termasuk ke dalam Persona 3 (Nathan). Hal ini sesuai 

Dalam melakukan pengujian, diperlukan skenario-skenario pengujian berisi \textit{task} yang perlu dilakukan partisipan, serta pertanyaan singkat mengenai pengalaman partisipan dalam mengerjakan \textit{task} tersebut. Skenario pengujian dapat ditemukan pada Lampiran \ref{chpt:skenario_lofi}, sedangkan hasil lengkap pengujian dapat ditemukan pada Lampiran F.

Dari melakukan pengujian, ditemukan beberapa temuan penting mengenai prototipe \textit{low-fidelity} sehingga perlu evaluasi sebelum lanjut merancan prototipe \textit{high-fidelity}. Temuan-temuan penting dapat dilihat pada Tabel \ref{tab:daftar_temuan_lofi}.

\newcommand{\cditem}[1]{\colorbox{white}{\raisebox{7pt}{\begin{minipage}[t]{0.7\textwidth}\linespread{1}\selectfont \begin{itemize}[parsep=0pt, leftmargin=*] #1 \end{itemize} \end{minipage}}}}

\RaggedLeft
\begin{small}
\begin{longtable}[c]{|W{c}{0.12\textwidth}|>{\ccnormspacingcenter}m{0.75\textwidth}|}
  \caption{Daftar Temuan Penting Pengujian Prototipe \textit{Low-Fidelity}}
  \label{tab:daftar_temuan_lofi} \\
  \hline \rowcolor[HTML]{A3E5F5}
  \textbf{Partisipan} & \textbf{Temuan Penting} \\ \hline \endfirsthead
  \hline \rowcolor[HTML]{A3E5F5}
  \textbf{Partisipan} & \textbf{Temuan Penting} \\ \hline \endhead
  \hline \endfoot

  1 & \cditem{\item Partisipan mengerti untuk mengerjakan seluruh \textit{task} yang diberikan saat pengujian \item Partisipan merasa penjelasan pada halaman-halaman pengenalan fitur terlalu kecil dan terlalu panjang} \\ \hline
  2 & \cditem{\item Partisipan merasa alur navigasi cukup mudah untuk diikuti dalam mencapai \textit{task} yang dikerjakan \item Partisipan merasa perlu penjelasan lebih tentang kemampuan yang akan aktif saat Bedtime Mode} \\ \hline
  3 & \cditem{\item Test \item Test} \\ \hline
  4 & \cditem{\item Test \item Test} \\ \hline
  5 & \cditem{\item Test \item Test} \\ \hline
  
\end{longtable}
\end{small}
\justifying
\FloatBarrier

Berdasarkan temuan-temuan yang telah disebutkan di atas, maka disusun beberapa rencana perbaikan terhadap prototipe \textit{low-fidelity}. Rencana perbaikan ini akan direalisasikan bersamaan dengan perancangan prototipe \textit{high-fidelity} yang akan dilakukan pada tahap berikutnya. Keseluruhan rancangan perbaikan dapat dilihat pada Tabel \ref{tab:daftar_perbaikian_lofi}.

\RaggedLeft
\begin{small}
\begin{longtable}[c]{|W{c}{0.05\textwidth}|>{\ccnormspacing}m{0.35\textwidth}|>{\ccnormspacing}m{0.35\textwidth}|}
  \caption{Daftar Temuan Penting Pengujian Prototipe \textit{Low-Fidelity}}
  \label{tab:daftar_temuan_lofi} \\
  \hline \rowcolor[HTML]{A3E5F5}
  \textbf{No} & \centering\textbf{Masalah / Saran} & \textbf{Rencana Perbaikan} \\ \hline \endfirsthead
  \hline \rowcolor[HTML]{A3E5F5}
  \textbf{No} & \centering\textbf{Masalah / Saran} & \textbf{Rencana Perbaikan} \\ \hline \endhead
  \hline \endfoot

  1 & Masalah & Rencana \\ \hline
  2 & Masalah & Rencana \\ \hline
  3 & Masalah & Rencana \\ \hline
  4 & Masalah & Rencana \\ \hline
  5 & Masalah & Rencana \\ \hline
  6 & Masalah & Rencana \\ \hline
  7 & Masalah & Rencana \\ \hline
  8 & Masalah & Rencana \\ \hline
  
\end{longtable}
\end{small}
\justifying
\FloatBarrier



