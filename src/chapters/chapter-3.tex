\chapter{Identifikasi Masalah dan Rancangan Solusi}

\newcommand{\ccnormspacing}{\baselineskip=12pt}
\newcommand{\ccnormspacingcenter}{\centering\arraybackslash\ccnormspacing}

Bab Identifikasi Masalah dan Rancangan Solusi berisi tentang penjelasan analisis permasalahan yang menjadi dasar dari tugas akhir ini. Secara garis besar, proses perancangan solusi akan mengikuti metodologi yang sudah ditetapkan yaitu pendekatan \textit{User-Centered Design} (UCD) menurut ISO 9241-210. Proses yang akan dibahas pada bab ini meliputi perancangan proses desain, identifikasi konteks penggunaan, analisis kebutuhan perangkat lunak, dan perancangan prototipe perangkat lunak. Gambaran alur UCD yang digunakan pada tugas akhir dapat dilihat pada Gambar \ref{fig:diagram_alur_kerja}.

\begin{figure}[h]
  \centering
  \includegraphics[width=0.8\textwidth]{chapter-1-method.png}
  \caption{Alur Kerja Penelitian}
  \label{fig:diagram_alur_kerja}
\end{figure}

\section{Perancangan Proses Desain}
\label{sec:perancangan_proses_desain}

Pada tahap perancangan proses desain dilakukan persiapan sumber daya yang diperlukan selama proses desain, serta penentuan ruang lingkup permasalahan.

% Sumber daya proses design

Ruang lingkup permasalahan yang ditentukan selama pengerjaan tugas akhir sebagai berikut

\begin{enumerate}
  \item Target Pengguna
  \subitem Target pengguna selama penelitian adalah pengguna \textit{smartphone} berbasis Android di Indonesia dengan rentang umur 18-34 tahun. Rentang usia tersebut adalah usia mayoritas pengguna media sosial di Indonesia. \parencite{mediasosial2020} 
  
  \item Fungsionalitas Aplikasi
  \subitem Lingkup fungsionalitas aplikasi adalah bagaimana desain interaksi yang baru dapat memperbaiki masalah yang ditemukan pada aplikasi Digital Wellbeing saat ini. Fungsionalitas dari aplikasi menyesuaikan dengan analisis hasil yang didapatkan dari riset dan wawancara. 
   
  \item Lingkup Pengembangan Aplikasi
  \subitem Desain interaksi aplikasi pencegah distraksi yang dibuat memiliki bentuk \textit{mobile interface} dengan mewujudkan sebuah prototipe aplikasi dalam \textit{platform Android}. Aplikasi Digital Wellbeing milik Google ditetapkan menjadi garis dasar pengembangan prototipe aplikasi tersebut.

\end{enumerate}

\section{Identifikasi Konteks Penggunaan}
\label{sec:identifikasi_konteks_penggunaan}

Pada tahap ini dilakukan proses analisis pengguna melalui data yang didapatkan dari ulasan pengguna aplikasi Digital Wellbeing dari situs Google Play Store, serta riset dengan metode wawancara.
% Hasil analisis pengguna digunakan untuk menyusun kebutuhan fungsional 

\subsection{Riset dan Analisis Pengguna}
\label{subsec:riset_analisis}

Tahap ini bertujuan untuk menganalis target pengguna, sehingga didapatkan data perilaku, masalah, tujuan, dan kebutuhan pengguna aplikasi Digital Wellbeing. Riset dilakukan dengan mengumpulkan data ulasan pengguna aplikasi Digital Wellbeing dari situs Google Play Store \textcite{dwplaystorereviews}. Metode pengumpulan data ini dipilih dengan alasan mengacu pada ISO 9241-210, bahwa informasi yang sudah tersedia dari suatu produk dapat dimanfaatkan untuk melakukan modifikasi atau peningkatan kualitas produk \parencite{iso9241-210:2010}, dalam hal ini informasi berbentuk ulasan pengguna.

Hasil terhadap analisis ulasan pengguna kemudian divalidasi dengan wawasan yang didapat dari wawancara. Karena pada data ulasan pengguna dari Google Play Store tidak terdapat informasi mengenai umur dan perilaku pengguna, wawancara dengan pengguna yang termasuk ke dalam lingkup permasalahan perlu dilakukan juga untuk mendapatkan wawasan mengenai perilaku pengguna. Perilaku pengguna berguna untuk menyusun persona pengguna.

\subsubsection{Ulasan Pengguna}
\label{subsubsec:ulasan_pengguna}

Berdasarkan situs Google Play Store per 13 April 2022, terdapat 609.005 ulasan untuk aplikasi Digital Wellbeing. Untuk menentukan jumlah \textit{sample size}, ditentukan \textit{confidence level} sebesar 95\%. Dikumpulkan data sebanyak 1000 ulasan dari target pengguna yang telah disebutkan pada bab \ref{sec:identifikasi_konteks_penggunaan},  kemudian dikategorikan secara manual, lalu didapatkan 288 ulasan yang dapat digunakan untuk menyusun masalah pengguna, sehingga \textit{sample} data memiliki \textit{margin of error} sebesar 5.78\%.

Pengkategorian ulasan dilakukan secara manual sehingga menghasilkan 11 kategori. Rincian tentang kategori tersebut dapat dilihat pada tabel \ref{tab:daftar_kategori}.

\RaggedLeft
\begin{small}
\begin{longtable}[c]{|W{c}{0.08\textwidth}|>{\ccnormspacing}m{0.72\textwidth}|>{\ccnormspacingcenter}m{0.1\textwidth}|}
  \caption{Daftar Kategori Ulasan}
  \label{tab:daftar_kategori} \\
  \hline \rowcolor[HTML]{A3E5F5} \textbf{ID} & \centering\textbf{Kategori Ulasan} & \textbf{Jumlah Ulasan} \\ \hline \endfirsthead
  \hline \rowcolor[HTML]{A3E5F5} \textbf{ID} & \centering\textbf{Kategori Ulasan} & \textbf{Jumlah Ulasan} \\ \hline \endhead
  
  \hline \endfoot
  
  KU-01    & Kurangnya widget untuk menampilkan data di Home Screen & 63 \\ \hline
  KU-02    & Perlu dikembangkannya fitur laporan penggunaan aplikasi untuk menampilkan data & 47 \\ \hline
  KU-03    & Perlu dikembangkannya fitur Focus Mode dengan menambah keketatan & 47 \\ \hline
  KU-04    & Perlu dikembangkannya kemampuan penjadwalan untuk fitur App Timer dan Focus Mode & 36 \\ \hline
  KU-05    & Kurangnya fitur pengaturan tingkat keketatan untuk fitur-fitur & 29 \\ \hline
  KU-06    & Kurangnya kemampuan penundaan untuk fitur App Timer & 18 \\ \hline
  KU-07    & Perlu dikembangkannya fitur Bedtime Mode dengan menambah keketatan & 12 \\ \hline
  KU-08    & Kurangnya penjelasan atau susunan kata yang dapat memotivasi pengguna untuk memakai fitur-fitur aplikasi & 12 \\ \hline
  KU-09    & Kurangnya fitur pengelompokkan aplikasi & 11 \\ \hline
  KU-10    & Kurangnya fitur pengaturan jam untuk akhir sebuah hari & 7 \\ \hline
  KU-11    & Kurangnya fitur \textit{whitelisting} untuk pembatasan akses aplikasi oleh Focus Mode & 6 \\ \hline
\end{longtable}
\end{small}
\justifying

\FloatBarrier

Adapun sejumlah ulasan yang tidak tergolong dalam kategori dinilai tidak relevan dalam penyusunan masalah pengguna, dengan penjelasan sebagai berikut

\begin{enumerate}
  \item Ulasan yang menilai positif aplikasi tanpa menyebutkan adanya masalah yang ditemukan dari aplikasi
  \item Ulasan yang menilai negatif aplikasi tanpa menyebutkan masalah yang ditemukan dari aplikasi
  \item Ulasan yang menyebutkan adanya bug dari aplikasi, seperti tidak berfungsinya sebuah fitur di perangkat tertentu
  \item Ulasan yang menyebutkan masalah yang tidak termasuk ke dalam batasan tugas akhir
  \item Ulasan yang tidak dapat dimengerti, seperti huruf-huruf yang tersusun secara acak 
  \item Ulasan yang melaporkan bahwa aplikasi tidak dapat dihapus dari perangkat
\end{enumerate}

\subsubsection{Perilaku Pengguna}
Setelah melakukan analisis ulasan pengguna untuk aplikasi Digital Wellbeing, dilakukan wawancara untuk mendapatkan perilaku pengguna, memvalidasi masalah pengguna dari analisis ulasan, serta menemukan masalah lain yang tidak ditemukan dari ulasan. Target berjumlah 5 orang dengan kriteria sebagaimana telah dijelaskan pada subbab \ref{sec:perancangan_proses_desain}. Jumlah tersebut dipilih karena menurut \textcite{nielsenusabilityproblems}, penelitian dengan 5 orang responden sudah cukup untuk menemukan rata-rata 85\% masalah dari desain sebuah produk, dan menambah responden lebih banyak akan mendapatkan wawasan tambahan yang semakin sedikit.

 Data perilaku responden digunakan untuk menyusun persona pengguna, serta membantu menganalisis kebutuhan pengguna terkait aplikasi Digital Wellbeing. Sedangkan hasil validasi ulasan pengguna digunakan untuk  menggali inti masalah yang dikeluhkan. Kedua pengamatan tersebut akan dibahas lebih lanjut dalam analisis masalah, kebutuhan, dan tujuan pengguna. Rancangan pertanyaan dapat dilihat pada Lampiran \ref{chpt:daftar_pertanyaan_wawancara}. Detail pemetaan pengamatan dengan pertanyaan wawancara dapat dilihat pada Tabel \ref{tab:pemetaan_pengamatan_wawancara}

\RaggedLeft
\begin{small}
\begin{longtable}[c]{|>{\ccnormspacing}m{0.72\textwidth}|p{0.2\textwidth}|}
  \caption{Pemetaan Pengamatan dengan Pertanyaan Wawancara}
  \label{tab:pemetaan_pengamatan_wawancara} \\
  \hline \rowcolor[HTML]{A3E5F5} \multicolumn{1}{|c|}{\textbf{Pengamatan}} & \multicolumn{1}{|c|}{\textbf{No. Pertanyaan}} \\ \hline \endfirsthead
  \hline \rowcolor[HTML]{A3E5F5} \multicolumn{1}{|c|}{\textbf{Pengamatan}} & \multicolumn{1}{|c|}{\textbf{No. Pertanyaan}} \\ \hline \endhead

  \hline \endfoot
  
  \rowcolor[HTML]{DCF3FC} \multicolumn{2}{|l|}{\textbf{A. Perilaku Responden}} \\ \hline
  Identitas responden & 1, 2, 3 \\ \hline
  Perilaku penggunaan \textit{smartphone} responden & 4, 5, 6, 7, 8 \\ \hline
  Perilaku responden terkait aplikasi pencegah distraksi & 9, 10, 11, 12 \\ \hline
  Perilaku responden terkait aplikasi \textit{Digital Wellbeing} & 13, 14, 15 \\ \hline
  \rowcolor[HTML]{DCF3FC} \multicolumn{2}{|l|}{\textbf{B. Validasi Ulasan Aplikasi Digital Wellbeing}} \\ \hline
  Validasi masalah kurangnya fitur widget pada Homescreen & 16, 17, 18 \\ \hline
  Validasi masalah pada fitur laporan data penggunaan aplikasi pada \textit{smartphone} & 19 \\ \hline
  Validasi masalah pada fitur Focus Mode & 20, 21 \\ \hline
  Validasi masalah untuk kemampuan penjadwalan pada fitur-fitur & 22, 23, 24 \\ \hline
  Validasi masalah kurangnya fitur pengaturan tingkat keketatan & 25, 26, 27 \\ \hline
  Validasi masalah kurangnya fitur penundaan pada App Timer & 28, 29, 30 \\ \hline
  Validasi masalah pada fitur Bedtime Mode & 31, 32, 33 \\ \hline
  Validasi masalah kurangnya penjelasan dan susunan kata & 34, 35 \\ \hline
  Validasi masalah kurangnya fitur pengelompokkan aplikasi & 36 \\ \hline
  Validasi masalah kurangnya fitur pengaturan jam akhir hari & 37 \\ \hline
  Validasi masalah kurangnya kemampuan \textit{whitelisting} & 38 \\ \hline
\end{longtable}
\end{small}
\justifying

\FloatBarrier

Jumlah responden yang diwawancarai adalah 10 (sepuluh) orang. Data hasil wawancara dapat dilihat pada Lampiran \ref{chpt:hasil_wawancara}. Dari 10 responden, 90\% mengakui tujuan utama dari penggunaan smartphone adalah berkomunikasi melalui aplikasi \textit{messenger}, dengan tujuan sekunder yaitu berinteraksi dengan media sosial atau sebagai sarana hiburan. Ditemukan bahwa 4 dari 10 orang menggunakan \textit{smartphone} sebagai alat utama yang membantu dalam pekerjaan, dengan tujuan untuk berkomunikasi dengan rekan atau klien, menggunakannya sebagai \textit{workstation}, atau mencari ide dan inspirasi.

Dari wawancara, ditemukan bahwa seluruh responden mengakui distraksi terkait dengan \textit{smartphone} lebih banyak berasal dari luar \textit{smartphone} itu sendiri, yaitu dari keinginan diri sendiri menggunakan \textit{smartphone} untuk memenuhi tujuan sekunder mereka. Walaupun hanya 30\% responden yang mengeluhkan notifikasi dari \textit{smartphone} dianggap mendistraksi mereka dari kegiatan utama, 70\% mengakui perlu untuk mencegah distraksi dari notifikasi dengan cara menggunakan aplikasi pencegah distraksi atau mengubah pengaturan notifikasi aplikasinya secara langsung.

Ditemukan juga bahwa rata-rata durasi penggunaan \textit{smartphone} harian keseluruhan responden adalah 6.2 jam per hari, di mana 70\% responden dapat menggunakan \textit{smartphone} selama lebih dari 6 jam sehari. Kedua angka tersebut melebihi rata-rata durasi penggunaan \textit{smartphone} di Indonesia pada tahun 2021 yaitu 5.4 jam per hari \parencite{dataai2022smartphoneindonesia}.  Keseluruhan dari responden menilai skala rata-rata 4 (empat) dari 5 (lima) terhadap durasi penggunaan \textit{smartphone} harian mereka. Di antara seluruh responden, 70\% mengakui butuh bantuan dari sebuah aplikasi pencegah distraksi untuk menurunkan durasi penggunaan tersebut.

Selain untuk mengurangi durasi penggunaan, responden juga memerlukan bantuan sebuah aplikasi untuk melakukan hal lain yang berhubungan dengan perbaikan kebiasaan digital mereka. Ditemukan 50\% responden memerlukan bantuan aplikasi untuk memantau penggunaan \textit{smartphone}, baik secara keseluruhan atau per aplikasi, dalam merencanakan perbaikan digital mereka. Lalu, 80\% dari responden merasa perlu diingatkan tentang tugas / kegiatan yang harus diselesaikan saat mereka menggunakan \textit{smartphone}. Keberadaan sebuah pengingat dapat menyadarkan pengguna terhadap alasan mereka menggunakan aplikasi pencegah distraksi. Selain itu, 50\% dari responden juga menggunakan aplikasi untuk membantu dalam memperbaiki jadwal tidurnya. Mereka mengakui bahwa saat menggunakan \textit{smartphone} sebelum tidur, seringkali mereka tidak menyadari waktu sehingga melewati jadwal tidur mereka. Keseluruhan data perilaku pengguna dirangkum pada Tabel \ref{tab:perilaku_pengguna}

\RaggedLeft
\begin{small}
\begin{longtable}[c]{|W{c}{0.08\textwidth}|>{\ccnormspacing}m{0.82\textwidth}|}
  \caption{Daftar Perilaku Pengguna}
  \label{tab:perilaku_pengguna} \\
  \hline \rowcolor[HTML]{A3E5F5} \textbf{ID} & \multicolumn{1}{|c|}{\textbf{Perilaku Pengguna}} \\ \hline \endfirsthead
  \hline \rowcolor[HTML]{A3E5F5} \textbf{ID} & \multicolumn{1}{|c|}{\textbf{Perilaku Pengguna}} \\ \hline \endhead
  
  \hline \endfoot
  
  P-01  &  Menggunakan \textit{smartphone} dengan tujuan primer untuk berkomunikasi melalui aplikasi \textit{messenger}, dan tujuan sekunder untuk berinteraksi lewat media sosial / sebagai sarana hiburan  \\ \hline
  P-02  &  Menilai skala rata-rata 4 (empat) dari 5 (lima) terhadap durasi penggunaaan \textit{smartphone} harian \\ \hline
  P-03  &  Merasa distraksi utama terkait \textit{smartphone} berasal dari diri sendiri \\ \hline
  P-04  &  Melakukan pengaturan terhadap notifikasi aplikasi yang dinilai sebagai distraksi \\ \hline
  P-05  &  Berusaha untuk mengurangi durasi penggunaan \textit{smartphone} harian \\ \hline
  P-06  &  Memantau kebiasaan penggunaan \textit{smartphone} \\ \hline
  P-07  &  Memerlukan bantuan untuk mengingatkan diri terhadap tugas / aktivitas yang harus dilakukan \\ \hline
  P-08  &  Memerlukan bantuan untuk mengingatkan diri terhadap jadwal tidur \\ \hline
\end{longtable}
\end{small}
\justifying

\FloatBarrier

\subsubsection{Masalah Pengguna}
\label{subsubsec:masalah_pengguna}

Ditemukan bahwa kategori ulasan yang didapat dari analisis ulasan pengguna cukup bervariasi dengan jumlah yang tersebar. Namun, ulasan pengguna tidak cukup dalam menggambarkan inti dari masalah yang mereka alami. Tahap validasi dari wawancara membantu menemukan inti masalah yang dikeluhkan serta gambaran utama dari keseluruhan masalah tersebut, yaitu aplikasi Digital Wellbeing kurang memberikan gambaran tentang bagaimana kebiasaan digital yang baik, dan alat-alat yang disediakan kurang dapat membantu penggunanya dalam memperbaiki kebiasaan digital mereka secara efisien.

Untuk menyelesaikannya, masalah tersebut perlu dipecahkan menjadi masalah-masalah pengguna yang dapat dirincikan. Hal tersebut dilakukan untuk mempermudah penentuan kebutuhan dan tujuan pengguna serta penyusunan solusi. Keseluruhan masalah pengguna dirangkum pada Tabel \ref{tab:daftar_masalah}.

\FloatBarrier

\RaggedLeft
\begin{small}
\begin{longtable}[c]{|W{c}{0.08\textwidth}|>{\ccnormspacing}m{0.6\textwidth}|>{\ccnormspacingcenter}m{0.2\textwidth}|}
  \caption{Daftar Masalah Pengguna}
  \label{tab:daftar_masalah} \\
  \hline \rowcolor[HTML]{A3E5F5}
  \textbf{ID} & \centering\textbf{Masalah Pengguna} & \textbf{Keterkaitan} \\ \hline \endfirsthead
  \hline \rowcolor[HTML]{A3E5F5}
  \textbf{ID} & \centering\textbf{Masalah Pengguna} & \textbf{Keterkaitan} \\ \hline \endhead

  \hline \endfoot

  MP-01  & Pengguna kesulitan dalam melakukan pengaturan fitur-fitur aplikasi Digital Wellbeing secara efisien & KU-01, KU-04, KU-05, KU-09, KU-10, KU-11 \\ \hline
  MP-02  & Pengguna kesulitan dalam menganalisis kebiasaan digital diri lewat aplikasi Digital Wellbeing & KU-02, P-06 \\ \hline
  MP-03  & Pengguna merasa fitur-fitur aplikasi Digital Wellbeing kurang ketat dalam membantu memperbaiki kebiasaan digital & KU-03, KU-04, KU-05, KU-06, KU-07, P-03, P-04, P-08\\ \hline
  MP-04  & Pengguna merasa fitur-fitur aplikasi Digital Wellbeing kurang fleksibel & KU-04, KU-06, KU-09, KU-10 \\ \hline
  MP-05  & Pengguna merasa interaksi dengan aplikasi Digital Wellbeing kurang pribadi & KU-08, P-03, P-07 \\ \hline
  MP-06  & Pengguna kurang dapat memahami penggunaan fitur-fitur yang disediakan oleh aplikasi Digital Wellbeing & KU-08 \\ \hline
  MP-07  & Pengguna kesulitan dalam mengakses informasi pada fitur-fitur aplikasi Digital Wellbeing & KU-01 \\ \hline
\end{longtable}
\end{small}
\justifying

\FloatBarrier


\subsubsection{Kebutuhan Pengguna}
\label{subsubsec:kebutuhan_pengguna}

Setelah masalah pengguna diidentifikasi, maka dapat dilakukan analisis mengenai kebutuhan pengguna. Kebutuhan ini adalah hal apa saja yang diperlukan pengguna untuk mencapai tujuannya dalam memakai aplikasi Digital Wellbeing. Analisis kebutuhan pengguna melibatkan masalah pengguna serta perilaku pengguna yang telah dibahas sebelumnya. Kebutuhan pengguna dirangkum di dalam Tabel \ref{tab:daftar_kebutuhan}.


\FloatBarrier

\RaggedLeft
\begin{small}
\begin{longtable}[c]{|W{c}{0.07\textwidth}|>{\ccnormspacing}m{0.67\textwidth}|>{\ccnormspacingcenter}m{0.17\textwidth}|}
  \caption{Daftar Kebutuhan Pengguna}
  \label{tab:daftar_kebutuhan} \\
  \hline \rowcolor[HTML]{A3E5F5}
  \textbf{ID} & \centering\textbf{Kebutuhan Pengguna} & \textbf{Keterkaitan} \\ \hline \endfirsthead
  \hline \rowcolor[HTML]{A3E5F5}
  \textbf{ID} & \centering\textbf{Kebutuhan Pengguna} & \textbf{Keterkaitan} \\ \hline \endhead

  \hline \endfoot

  K-01  & Pengalaman pengaturan fitur yang lebih efisien dengan kemampuan seperti pencarian dan pengelompokan aplikasi & MP-01 \\ \hline
  K-02  & Widget untuk mengakses data serta melakukan pengaturan terhadap fitur-fitur lewat Homescreen & MP-01, MP-07 \\ \hline
  K-03  & Fitur rekomendasi untuk memberikan informasi tentang kebiasaan digital yang baik dan aksi yang dapat dilakukan & MP-02, P-05, P-06 \\ \hline
  K-04  & Laporan penggunaan \textit{smartphone} dengan rentang waktu lebih banyak dan ringkasan informasi seperti rata-rata penggunaan & MP-02, P-06 \\ \hline
  K-05  & Pengaturan tingkat keketatan untuk kemampuan tertentu dari fitur-fitur & MP-03, P-04 \\ \hline
  K-06  & Kemampuan penguncian pengaturan untuk mencegah pengubahan oleh pengguna & MP-03, P-04, P-05 \\ \hline
  K-07  & Fitur penjadwalan dengan kemampuan menambah lebih dari satu jadwal aktivasi fitur & MP-04, P-08 \\ \hline
  K-08  & Kemampuan penundaan pada restriksi yang diterapkan oleh fitur-fitur & MP-04 \\ \hline
  K-09  & Kemampuan pengaturan pesan untuk melakukan personalisasi pesan-pesan pengingat dari fitur-fitur & MP-05, P-07 \\ \hline
  K-10  & Tampilan aplikasi yang lebih menarik dengan deskripsi fitur yang lebih jelas & MP-06 \\ \hline
\end{longtable}
\end{small}
\justifying

\FloatBarrier

\subsubsection{Tujuan dan Kegiatan Pengguna}
\label{subsubsec:tujuan_kegiatan_pengguna}

Dengan ditentukannya kebutuhan pengguna, maka dapat dianalisis tujuan yang ingin dicapai oleh pengguna. Dalam mencapai tujuan-tujuan tersebut, maka perlu ditentukan kegiatan yang harus dilakukan oleh pengguna. Analisis tujuan dan kegiatan pengguna dilakukan dengan mengkaitkan kebutuhan pengguna dan perilaku pengguna. Hasil analisis dirangkum pada Tabel \ref{tab:daftar_tujuan_kegiatan}.

\newlength{\cccolgoal}
\setlength{\cccolgoal}{0.3\textwidth}

\newlength{\cccolneed}
\setlength{\cccolneed}{0.13\textwidth}

\newcommand{\ccgoal}[2]{\multirow{#1}{\cccolgoal}{\linespread{1}\selectfont #2}}
\newcommand{\ccneed}[2]{\multirow{#1}{\cccolneed}{\centering\linespread{1}\selectfont #2}}
\newcommand{\ccline}{\hhline{|-|~|-|~|}}

\newpage

\FloatBarrier

\RaggedLeft
\begin{small}
\begin{longtable}[c]{|W{c}{0.07\textwidth}|>{\ccnormspacing}m{\cccolgoal}|>{\ccnormspacing}m{0.38\textwidth}|>{\ccnormspacingcenter}m{\cccolneed}|}
  \caption{Daftar Tujuan dan Kegiatan Pengguna}
  \label{tab:daftar_tujuan_kegiatan} \\
  \hline \rowcolor[HTML]{A3E5F5}
  \textbf{ID} & \centering\textbf{Tujuan Pengguna} & \centering\textbf{Kegiatan Pengguna} & \textbf{Kebutuhan} \\ \hline \endfirsthead
  \hline \rowcolor[HTML]{A3E5F5}
  \textbf{ID} & \centering\textbf{Tujuan Pengguna} & \centering\textbf{Kegiatan Pengguna} & \textbf{Kebutuhan} \\ \hline \endhead

  \hline \endfoot

  UT-01 & & Mengunci pengaturan pada waktu tertentu & \\ \ccline
  UT-02 & & Membatasi istirahat yang dapat diambil pada fitur Focus Mode & \\ \ccline
  UT-03 & \ccgoal{-5}{Membatasi diri dari melonggarkan pengaturan} & Memilih tingkat keketatan dari fitur-fitur & \ccneed{-5}{P-03, K-05, K-06}\\ \hline
  
  UT-04 & & Membuat jadwal aktivasi fitur pemblokiran akses aplikasi & \\ \ccline
  UT-05 & \ccgoal{-3}{Mencegah distraksi dari smartphone di waktu tertentu} & Memilih aplikasi yang diblokir aksesnya & \ccneed{-3}{P-04, K-01, K-07}\\ \hline
  
  UT-06 & & Memasang batas waktu penggunaan harian aplikasi & \\ \ccline
  UT-07 & & Melihat sisa waktu penggunaan aplikasi lewat widget & \\ \ccline
  UT-08 & \ccgoal{-5}{Membatasi waktu penggunaan smartphone harian} & Melihat total waktu penggunaan smartphone lewat widget & \ccneed{-5}{P-05, K-02, K-01}\\ \hline
  
  UT-09 & & Mengakses laporan penggunaan smartphone & \\ \ccline
  UT-10 & & Memilih rentang waktu laporan penggunaan smartphone & \\ \ccline
  UT-11 & \ccgoal{-5}{Menganalisis kebiasaan penggunaan smartphone} & Melihat rekomendasi kebiasaan penggunaan smartphone yang baik & \ccneed{-5}{P-06, K-03, K-04}\\ \hline
  
  UT-12 & & Memasang pesan pengingat harian & \\ \ccline
  UT-13 & \ccgoal{-3}{Mengingatkan diri terhadap aktivitas utama yang seharusnya dilakukan} & Mengatur frekuensi notifikasi fitur pengingat  & \ccneed{-3}{P-07, K-09}\\ \hline
  
  UT-14 & & Mengatur jadwal aktivasi fitur Bedtime Mode & \\ \ccline
  UT-15 & \ccgoal{-3}{Membantu mengatur kebiasaan tidur yang sehat} & Membatasi aplikasi yang dapat diakses di jadwal tidur & \ccneed{-3}{P-08, K-05}\\ \hline
  
  UT-16 & & Mengambil waktu istirahat dari Focus Mode & \\ \ccline
  UT-17 & & Menunda aktivasi Bedtime Mode & \\ \ccline
  UT-18 & \ccgoal{-4}{Mengambil istirahat sejenak dari restriksi aplikasi} & Memperpanjang waktu penggunaan aplikasi yang diblokir oleh App Timer & \ccneed{-4}{K-08}\\ \hline

\end{longtable}
\end{small}

\justifying

\FloatBarrier


\subsection{Persona}
Setelah melakukan analisis mengenai ulasan pengguna tentang aplikasi Digital Wellbeing dan wawancara langsung dengan pengguna, dilakukan segmentasi pengguna menjadi beberapa kelompok persona. Persona merepresentasikan kelompok-kelompok pengguna dengan karakteristik dan perilaku yang berbeda. Persona berperan menjadi arah pengembangan interaksi aplikasi dan mengurangi kemungkinan mendesain aplikasi untuk semua orang sehingga menghasilkan desain yang tidak disenangi oleh siapapun. \parencite{cooper2014face}



% \blindtext

\section{Analisis Kebutuhan Perangkat Lunak}

% Proses perancangan solusi mengacu kepada metode \textit{User-Centered Design} sesuai dengan standar ISO 9241-210, di mana pada tahap perancangan desain interaksi untuk memenuhi kebutuhan pengguna akan disertai dengan prototipe aplikasi. Gambar \ref{fig:diagram_alur_kerja} menjelaskan tentang alur kerja penelitian yang dilakukan.

% \begin{figure}[h]
%   \centering
%   \includegraphics[width=0.8\textwidth]{chapter-3-alur-penelitian.png}
%   \caption{Alur Kerja Penelitian}
%   \label{fig:diagram_alur_kerja}
% \end{figure}

% \subsection{Perancangan Proses Desain}
% Ruang lingkup yang ditentukan pada penelitian ini adalah sebagai berikut

% \begin{enumerate}
%   \item Lingkup Pengguna
%   \subitem Target pengguna dari penelitian ini adalah masyarakat Indonesia yang pernah menggunakan atau memiliki ketertarikan terhadap aplikasi pencegah distraksi. Rentang usia dari target pengguna tidak dibatasi, namun difokuskan kepada golongan \textit{millenials} dengan rentang usia 18-30 tahun.
%   \item Lingkup Pengembangan
%   \subitem Desain interaksi aplikasi pencegah distraksi yang dibuat memiliki bentuk \textit{mobile interface} dengan mewujudkan sebuah prototipe aplikasi dalam \textit{platform Android}. Aplikasi Digital Wellbeing milik Google ditetapkan menjadi garis dasar pengembangan prototipe aplikasi tersebut.
% \end{enumerate}


% \subsection{Identifikasi Konteks Penggunaan}

% Pada tahap ini dilakukan analisis hasil riset penggunaanalisis terhadap hasil riset yang


% \subsection{Identifikasi Kebutuhan Pengguna}


% \subsection{Perancangan Desain}


% \subsection{Evaluasi Solusi Masalah}



% % Ketiga solusi yang telah diuraikan pada subbab \ref{sec:analisis_solusi} akan diimplementasikan dalam prototipe aplikasi, beserta fitur-fitur lain pada Digital Wellbeing yang akan mendukung solusi tersebut. Prototipe aplikasi ini akan diimpementasikan pada \textit{platform} Android. Secara garis besar, proses perancangan prototipe aplikasi akan menggunakan pendekatan \textit{user-centered design} (UCD).

% Seperti yang telah disebutkan pada subbab \ref{sec:metodologi}, metodologi yang digunakan dalam pengerjaan Tugas Akhir ini akan menggunakan pendekatan UCD. Dengan maksud mengikuti prosesnya, maka langkah selanjutnya yang akan dilakukan adalah mengumpulkan data. Pengumpulan data akan dilakukan dengan menyebarkan form secara online serta melakukan wawancara dengan responden yang bersedia untuk bekerja sama lebih lanjut. Proses ini akan dilaksanakan pada periode pengerjaan Tugas Akhir 2. Pengumpulan data ini bertujuan untuk melakukan validasi terhadap permasalahan yang sudah dianalisis, dan juga tidak menutup kemungkinan untuk menemukan permasalahan desain interaksi lain dari masukan pengguna.

% Setelah melakukan pengumpulan data, akan dilakukan analisis terhadap masukan yang didapat untuk mejadi kebutuhan perangkat lunak. Hasil analisis juga akan memvalidasi analisis masalah dan solusi yang didapat dari observasi penulis pada subbab \ref{sec:analisis_masalah} dan \ref{sec:analisis_solusi}.

% Kebutuhan perangkat lunak yang telah disusun akan diimplementasi dalam bentuk prototipe \textit{low-fidelity} terlebih dahulu. Setelah dilakukan evaluasi, maka implementasi akan dilanjutkan dalam bentuk prototipe \textit{high-fidelity}. Setelah menjalani evaluasi, maka perancangan prototipe aplikasi akan dikerjakan. Prototipe aplikasi diharapkan akan menghasilkan data dengan kualitas yang lebih tinggi pada saat evaluasi dibandingkan saat menggunakan prototipe \textit{low-fidelity} atau \textit{high-fidelity}. Hasil evaluasi juga akan menentukan apakah aplikasi akan menjalani proses iterasi atau diimplementasi lebih lanjut.

% \blindtext