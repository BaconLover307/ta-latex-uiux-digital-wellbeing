\section{\textit{Digital Wellbeing}}
\label{sec:digital_wellbeing}

Untuk menanggapi permasalahan pada penggunaan \textit{smartphone} yang berlebihan, peneliti di bidang HCI mulai gencar untuk melakukan studi terhadap kesengajaan untuk tidak menggunakan teknologi. Sebagai jawabannya, peneliti, perusahaan, dan organisasi dunia muncul dengan istilah \textit{Digital Wellbeing} untuk menyatakan kesehatan hubungan antara pengguna dan teknologinya. 

Menurut Forum for Well-being in Digital Media yang ada di bawah \textcite{unesco2015dwconference}, \textit{Digital Wellbeing} adalah peningkatan kesejahteraan pengguna dalam pemakaian media digital. Kesejahteraan yang dimaksud adalah aset psikologis berharga seseorang untuk bertahan hidup dan merasakan pengalaman positif yang berkelanjutan. Sebuah lingkungan atau media digital untuk dapat memberikan pengembangan kesejahteraan jangka panjang bagi penggunanya perlu memenuhi syarat-syarat berikut:

\begin{enumerate}
  \item Membawakan rasa sambut dan empati di antara penggunanya,
  \item Mendorong pengembangan rasa kompetensi untuk penggunanya,
  \item Memungkinkan penggunanya untuk bertingkah sesuai hati nuraninya, dan
  \item Mendorong penggunanya untuk bereksplorasi pada bidang yang diminati untuk memicu pengembangan diri.
\end{enumerate}

Salah satu perusahaan yang memunculkan komitmen untuk menanggulangi permasalahan \textit{smartphone addiction} adalah Google. Pada bulan Mei tahun 2018 dalam konferensi Google I/O, Google meluncurkan langkah \textit{Digital Wellbeing}, sebuah filosofi desain yang akan dipakai dalam produk-produknya yang bertujuan untuk memberikan hubungan yang lebih baik antara pengguna dan teknologi yang dipakainya. Dalam sebuah \textit{online course} yang disediakan Google dijelaskan bahwa \textit{Digital Wellbeing} adalah tentang membuat sebuah hubungan yang sehat dengan teknologi dan menjaga kesehatan hubungan tersebut. Google menyadari bahwa teknologi yang berkembang dengan pesat telah menjadi sebuah tantangan dalam menjaga keseimbangan waktu yang dihabiskan antara dunia nyata dan dunia maya. Maka dari itu konsep \textit{Digital Wellbeing} yang dibawakan oleh Google mengajak penggunanya untuk mengambil kendali teknologi agar dapat memberikan manfaat dan potensial semaksimal mungkin serta membantu mencapai tujuan, melainkan menjadi pengganggu, distraksi, atau rintangan \parencite{google2019dwcourse}.

\subsection{Manfaat \textit{Digital Wellbeing}}

Melalui bantuan konsep \textit{Digital Wellbeing}, membuat sebuah kebiasaan yang sehat dalam menggunakan teknologi dapat memberikan penggunanya beberapa manfaat. Menurut Google, manfaat yang didapatkan adalah sebagai berikut:

\begin{enumerate}
  \item Meningkatkan fokus untuk digunakan pada kegiatan utama
  \item Menjaga atau memperbaiki hubungan sesama berkat tersedianya perhatian penuh untuk lawan bicara
  \item Meningkatkan produktifitas serta efektifitas dalam pekerjaan
  \item Meningkatkan keterlibatan serta kesadaran diri atas lingkungannya
\end{enumerate}

\subsection{Cara Penerapan \textit{Digital Wellbeing}}

Untuk mencapai manfaat-manfaat yang telah disebutkan sebelumnya, berikut adalah beberapa cara yang perlu dilakukan menurut Google:

\begin{enumerate}
  \item Meningkatkan kesadaran diri terhadap kebiasaan digital di dunia maya
  \subitem Untuk mengubah pola pikir dan perilaku, langkah pertama terbaik yang sebaiknya dilakukan dapat dimulai dari diri sendiri. Merefleksikan berapa banyak waktu yang dihabiskan di dunia maya dapat menyadarkan terhadap kebiasaan digital yang dilakukan sehari-hari. Dari sana, seseorang dapat menilai apakah mereka puas akan kebiasaan-kebiasaan tersebut. Hal ini juga perlu dilakukan sendiri karena penggunaan teknologi antarindividu dipastikan berbeda.
  
  \item Menyadari ulang tujuan utama dari pemakaian teknologi digital
  \subitem Terkadang seseorang dapat terjebak dalam kebiasaan digitalnya sehingga mereka hanya melakukan atau memakai teknologi tanpa menyadari apa yang ingin dicapai. Mengambil langkah mundur untuk berefleksi dapat menyadarkan diri akan tujuan utama dari pemakaian teknologi digital dan mengadakan kemungkinan untuk mengubah pola pemakaian tersebut.
  
  \item Meminta pertolongan eksternal untuk menilai kebiasaan digital diri
  \subitem Mendapatkan perspektif lain adalah cara yang baik dalam melakukan refleksi karena terkadang penilaian diri dapat bersifat subjektif atau terjadi estimasi yang tidak akurat dari nilai aslinya. Dengan meminta bantuan teman, rekan kerja, atau keluarga yang dapat membantu memantau kebiasaan digital diri dapat membuka perspektif baru untuk mengkonfirmasi penilaian diri sendiri.
  
  \item Memantau penggunaan teknologi digital dengan bantuan alat
  \subitem Keberadaan data yang jelas tentang penggunaan teknologi digital, seperti waktu penggunaan aplikasi dan jumlah notifikasi yang diterima dapat membantu memberi gambaran saat melakukan refleksi. Aplikasi-aplikasi untuk memantau aktivitas tersebut dapat didapatkan dengan mudah di \textit{mobile appstore} pada masing-masing platform atau di \textit{website} untuk PC.
  
  \item Membuat perubahan kecil untuk membentuk kebiasaan baru
  \subitem Setelah mendapatkan tujuan dan bayangan akan bagaimana kebiasaan yang ingin dicapai, langkah-langkah dapat diambil untuk membentuk kebiasaan lama menjadi yang lebih sehat dan bermanfaat. Langkah-langkah yang diambil dapat bertahap dan tidak terlalu besar, hal ini ditujukan agar tidak memaksa diri terlalu jauh dan memicu stress yang tidak diinginkan dari perubahan yang terlalu besar.
  
\end{enumerate}

  \subsection{Panduan Penerapan \textit{Digital Wellbeing}}
  
  Google menyadari bahwa manfaat-manfaat dari \textit{Digital Wellbeing} tidak dapat dicapai dengan cara yang sama untuk semua orang. Oleh karena itu, terdapat beberapa opsi yang disarankan oleh Google untuk menerapkan \textit{Digital Wellbeing} yang terbagi ke dalam 2 kategori panduan, yaitu panduan digital dan panduan fisik.
  
  \subsubsection{Panduan Digital}
  
  Panduan digital adalah kumpulan aplikasi serta teknologi yang didesain untuk membantu pengguna teknologi digital untuk mengambil alih kendali terhadap teknologi yang dipakai. Berikut adalah beberapa panduan digital yang disarankan oleh Google:
  
  \begin{enumerate}
    \item Meminimalisir masuknya notifikasi
    \item Mengubah warna tampilan \textit{smartphone} menjadi berskala abu-abu
    \item Mengatur \textit{smartphone} ke dalam mode Do Not Disturb
    \item Membatasi jumlah aplikasi atau alat pada layar utama
  \end{enumerate}

  \subsubsection{Panduan Fisik}

  Panduan fisik adalah panduan penerapan \textit{Digital Wellbeing} yang memandang dari segi lingkungan atau ruang personal di sekitar diri. Panduan ini dapat dilakukan dengan atau tanpa bantuan teknologi, namun diutamakan untuk kondisi ketidakberadaannya teknologi. Berikut adalah beberapa panduan fisik yang disarankan oleh Google:

  \begin{enumerate}
    \item Menghabiskan waktu sebanyak mungkin di luar ruangan
    \item Memulai dan mengakhiri hari tanpa menggunakan \textit{smartphone}
    \item Melakukan pertemuan atau percakapan tanpa melibatkan perangkat digital
    \item Membedakan perangkat yang digunakan untuk keperluan pekerjaan dengan keperluan hidup sehari-hari
    \item Meletakkan \textit{smartphone} di lokasi yang berbeda dari tempat bekerja
    \item Menjadwalkan akses terhadap \textit{e-mail} 
  \end{enumerate}

\subsection{Prinsip Desain \textit{Digital Wellbeing}}
\label{subsec:prinsip_desain_dw}

Dalam upaya membantu \textit{developer-developer} lain mendukung konsep dari \textit{Digital Wellbeing}, \textcite{google2021dwframework} telah menyusun sebuah kakas desain berisi prinsip-prinsip desain interaksi untuk diterapkan kepada produk yang dibuat. Tujuan Google membuat kakas ini adalah agar \textit{developer} dapat membuat produk atau aplikasi yang mampu meningkatkan \textit{digital wellbeing} penggunanya dengan mendukung intensi baik dalam menggunakan teknologi. Berikut adalah penjelasan dari prinsip-prinsip desain tersebut

\begin{enumerate}
  \item \textit{Empowerment}
  \subitem Produk sebaiknya memiliki pengaturan \textit{default} yang dapat mendukung pengguna untuk memperbaiki perilakunya. Ketika terdapat pilihan terhadap pengaturan, sebaiknya tampilan yang pertama kali dilihat pengguna adalah pengaturan \textit{default} yang memberikan bayangan terbaik untuk mendukung tujuan pengguna. Namun pengguna tetap perlu dapat mengubahnya jika diperlukan.

  \item \textit{Awareness}
  \subitem Tindakan refleksi diri membuat orang lebih sadar tentang perilakunya atau bagaimana seseorang menggunakan waktunya. Menampilkan data penggunaan dengan langsung dapat mendorong pengguna untuk merefleksi perilakunya, membantunya dalam meluruskan tujuan mereka untuk memperbaiki kebiasaan serta menentukan langkah yang harus diambil. Contoh pemaparan yang dapat membantu adalah tampilan \textit{dashboard}, visualisasi data, dan wawasan perilaku pengguna.

  \item \textit{Control}
  \subitem Transparansi terhadap pengaturan dapat mengantisipasi pengguna dengan kebutuhan, kemampuan, dan latar belakang yang beragam. Pemberian pengaturan yang mendalam dapat membantu banyak pengguna mencapai tujuan spesifik mereka. Hal ini harus didukung dengan penjelasan yang jelas dan transparansi tentang cara kerja atau fungsi dari pengaturannya, termasuk bagaimana data pengguna dikumpulkan dan digunakan.

  \item \textit{Adaptability}
  \subitem Perlu dipertimbangkan kemampuan fitur untuk beradaptasi terhadap beragam konteks pengguna. Dengan mengintegrasikan pengalaman penggunaan aplikasi dengan konteks pengguna seperti waktu, lokasi, atau perangkat yang sedang digunakan dapat mengurangi beban navigasi pengguna.

\end{enumerate}

\subsection{Aplikasi Google Digital Wellbeing}

Aplikasi Google Digital Wellbeing yang telah disebutkan pada awal subbab \ref{sec:digital_wellbeing} adalah bagian dari langkah \textit{Digital Wellbeing} yang diluncurkan pada konferensi Google I/O. Aplikasi ini sudah terintegrasi pada sistem operasi Android sejak versi Android 9.0 \parencite{google2021dwsupport}. Menurut \textcite{8976353}, aplikasi ini berperan sebagai alat untuk membantu mengoptimisasi penggunaan \textit{smartphone}, didesain untuk membantu penggunanya hidup berdampingan dengan teknologi digital yang selalu menarik perhatian dan menyita waktu.

Fitur-fitur yang terdapat di aplikasi ini didesain untuk membantu penggunanya menerapkan konsep \textit{Digital Wellbeing} dengan menggunakan panduan penerapan konsep tersebut dalam desain aplikasinya. Berikut adalah fitur-fitur yang tersedia.

% https://lup.lub.lu.se/luur/download?func=downloadFile&recordOId=8976353&fileOId=8981518
% https://static.googleusercontent.com/media/wellbeing.google/en//static/pdf/digital-wellbeing-product-experience-toolkit.pdf
% https://experiments.withgoogle.com/collection/digitalwellbeing
% Cold Turkey

\subsubsection{Dashboard}
Dashboard adalah fitur yang berperan seperti kendali pusat dari aplikasi Google Digital Wellbeing. Dashboard dapat menampilkan jumlah waktu yang dihabiskan untuk membuka aplikasi, jumlah berapa kali pengguna membuka \textit{smartphone}, dan jumlah notifikasi yang diterima pada hari tersebut dalam sebuah grafik \parencite{android2019digitalwellbeing}. Dashboard juga memiliki kemampuan untuk menampilkan ringkasan dari data pada hari-hari sebelumnya, memungkinkan pengguna untuk memantau dan menganalisis kebiasaannya. Gambar fitur pada aplikasi dapat dilihat pada Lampiran \ref{chpt:gambar_dw}.

\subsubsection{App Timers}
App Timers adalah fitur yang memungkinkan pengguna untuk memberikan batas waktu akses pada aplikasi tertentu. Fitur ini berperan sebagai pelengkap dari fitur Dashboard yang telah disebutkan. Jika pengguna telah mengakses aplikasi yang diatur melebihi dari batas waktu yang ditentukan, maka semua notifikasinya akan diheningkan dan pengguna tidak dapat mengakses aplikasinya lagi untuk hari tersebut \parencite{android2019digitalwellbeing}. Ikon aplikasi yang diblokir akan memiliki warna berskala abu-abu, serta jika ditekan akan ada pengingat bahwa penggunaan aplikasi tersebut telah mencapai batas waktu sehingga dapat dilanjutkan esok hari. Gambar fitur pada aplikasi dapat dilihat pada Lampiran \ref{chpt:gambar_dw}.

\subsubsection{Bedtime Mode}
Bedtime Mode adalah fitur yang bertujuan untuk membantu penggunanya menjaga jam tidur yang sehat. Fitur Bedtime Mode akan mengubah warna tampilan \textit{smartphone} menjadi berskala abu-abu, dan menghambat notifikasi yang masuk dengan bantuan fitur Do Not Disturb. Bedtime Mode memungkinkan pengguna untuk mengatur jadwal tidurnya dari jam mulai tidur, jam bangun, serta hari apa saja fitur tersebut akan menyala. Bedtime Mode juga dapat diatur untuk menyala hanya saat pengisian baterai \textit{smartphone} \parencite{android2019digitalwellbeing}. Bedtime Mode juga memiliki kemampuan untuk memberikan notifikasi kepada pengguna untuk mengingatkan bahwa mode akan segera aktif. Gambar fitur pada aplikasi dapat dilihat pada Lampiran \ref{chpt:gambar_dw}.

\subsubsection{Focus Mode}
Focus Mode adalah fitur yang bertujuan untuk membantu penggunanya memblokir distraksi dari \textit{smartphone} dan memfokuskan diri untuk bekerja. Focus Mode memungkinkan pengguna untuk memilih aplikasi yang dinilai dapat menjadi distraksi, kemudian memblokir notifikasi dari aplikasi tersebut serta memblokir akses untuk membuka aplikasi tersebut. Pengguna juga dapat mengatur jadwal menyalanya fitur Focus Mode \parencite{android2019digitalwellbeing}.

Pada saat Focus Mode aktif, ikon aplikasi pada halaman utama serta \textit{app drawer} akan berskala abu-abu. Ketika ikon diklik maka akan muncul sebuah pesan yang mengingatkan bahwa aplikasi tersebut dinilai sebagai distraksi dan sedang diblokir sementara. Kemudian pengguna memiliki pilihan untuk menutupnya atau menggunakan aplikasi tersebut selama 5 menit, hal ini bertujuan agar pengguna dapat menggunakan aplikasi tersebut dalam kondisi darurat. Selain itu, status aktif Focus Mode akan ditampilkan pada bar notifikasi disertai 2 tombol, "Take a break" dan "Turn off for now". Tombol "Take a break" dapat memberikan pilihan kepada pengguna untuk mematikan Focus Mode selama 5 menit, 15 menit, atau 30 menit. Tombol ini bertujuan agar pengguna dapat beristirahat sejenak dari sesi pekerjaannya dan menggunakan aplikasi-aplikasi yang dinilai sebagai distraksi. Tombol "Turn off for now" dapat mematikan Focus Mode pada hari tersebut walaupun jadwal yang ditentukan belum terpenuhi. Gambar fitur pada aplikasi dapat dilihat pada Lampiran \ref{chpt:gambar_dw}.


\subsubsection{Do Not Disturb}
Do Not Disturb adalah fitur yang bertujuan untuk membantu penggunanya memblokir gangguan dari notifikasi pada \textit{smartphone}. Fitur Do Not Disturb akan mematikan suara dari \textit{smartphone} sehingga notifikasi atau panggilan yang masuk tidak mampu mengeluarkan suara. Selain itu, saat Do Not Disturb aktif maka layar \textit{smartphone} tidak akan menyala saat adanya notifikasi atau panggilan yang masuk \parencite{android2019digitalwellbeing}. Selain dari aplikasi Digital Wellbeing, fitur ini juga dapat diaktifkan dari \textit{control center}.

\subsubsection{Customize Notifications}
Customize Notifications adalah fitur yang memungkinkan penggunanya untuk mengatur notifikasi yang diterima. Pengguna dapat mengatur notifikasi dari aplikasi apa saja yang dapat diterima oleh \textit{smartphone} serta bagaimana bentuk notifikasi yang ingin diterima. Pengaturan notifikasi juga dapat diatur untuk fitur spesifik dari sebuah aplikasi, jika aplikasi tersebut memberikan izin bagi pengguna untuk melakukan pengaturan tersebut. 
