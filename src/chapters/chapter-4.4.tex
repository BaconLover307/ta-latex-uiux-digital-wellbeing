\section{Pengujian Prototipe \textit{High-Fidelity} Iterasi Pertama}
\label{sec:test_1}

Prototipe \textit{High-Fidelity} yang sudah dirancang perlu dilakukan \textit{usability testing}. Pengujian prototipe \textit{low-fidelity} dilakukan untuk mengukur capaian dari tujuan-tujuan yang sudah ditentukan pada subbab \ref{subsec:analisis_goals}, yaitu \textit{usability goals} \textit{efficiency} dan \textit{learnability} serta \textit{user experience goals} \textit{helpful} dan \textit{motivating}. Selain itu, \textit{usability testing} juga menguji \textit{overall usability} dari prototipe. Untuk mengukur capaian-capaian tersebut, partisipan diminta untuk menyelesaikan beberapa \textit{task} terkait halaman, \textit{widget}, dan fitur yang telah dirancang. Pada Tabel \ref{tab:daftar_pengujian_goals}, dilakukan pemetaan \textit{usability goals} dan \textit{user experience goals} dengan tujuan dan kriteria pengujian yang berkaitan.

\RaggedLeft
\begin{footnotesize}
\begin{longtable}[c]{|>{\ccnormspacingcenter}m{0.12\textwidth}|>{\ccnormspacing}m{0.38\textwidth}|>{\ccnormspacing}m{0.38\textwidth}|}
  \caption{Pemetaan Tujuan dan Kriteria Pengujian terhadap Aspek Pengujian}
  \label{tab:daftar_pengujian_goals} \\
  \hline \rowcolor[HTML]{A3E5F5}
  \textbf{Aspek} & \multicolumn{1}{|c|}{\textbf{Tujuan Pengujian}} & \multicolumn{1}{|c|}{\textbf{Kriteria Pengujian}} \\ \hline \endfirsthead
  \hline \rowcolor[HTML]{A3E5F5}
  \textbf{Aspek} & \multicolumn{1}{|c|}{\textbf{Tujuan Pengujian}} & \multicolumn{1}{|c|}{\textbf{Kriteria Pengujian}}\\ \hline \endhead
  \hline \endfoot

  % Usability Goals
  \textit{Overall Usability} & Mengukur \textit{perceived usability} dari prototipe \textit{high-fidelity} & Mengukur \textit{perceived usability} menggunakan kuesioner dari \textit{System Usability Scale} \\ \hline
  \rowcolor[HTML]{DCF3FC} \multicolumn{3}{|l|}{\textbf{\textit{Usability Goals}}} \\ \hline
  \textbf{G-01} \textit{Efficiency} & Mengukur seberapa efisien pengguna dalam melakukan aktivitasnya di dalam prototipe \textit{high-fidelity} & Mengukur tingkat \textit{efficiency} dengan menggunakan pertanyaan yang sesuai dari kuesioner \textit{System Usability Scale} \\ \hline
  
  \textbf{G-02} \textit{Learnability} & Mengukur seberapa mudah fitur-fitur aplikasi untuk dipelajari dan digunakan oleh pengguna & Mengukur tingkat kemudahan penggunaan aplikasi dengan \textit{Single Ease Question}\\ \hline
  
  % UX Goals
  \rowcolor[HTML]{DCF3FC} \multicolumn{3}{|l|}{\textbf{\textit{User Experience Goals}}} \\ \hline
  \textbf{G-03} \textit{Helpful} & Mengetahui apakah pengguna dapat merasa terbantu dalam melakukan aktivitasnya oleh fitur-fitur yang disediakan prototipe \textit{high-fidelity}  & Menggunakan subskala \textit{value/usefulness} dari \textit{Intrinsic Motivation Inventory} untuk mengukur seberapa membantu aplikasi kepada pengguna \\ \hline
  
  \textbf{G-04} \textit{Motivating} & Mengetahui apakah pengguna dapat merasa termotivasi untuk fokus pada pekerjaannya oleh fitur-fitur yang disediakan prototipe \textit{high-fidelity} & Menggunakan subskala \textit{interest/enjoyment} dan \textit{pressure/tension} dari \textit{Intrinsic Motivation Inventory} untuk mengukur tingkat motivasi pengguna setelah melakukan pengujian \\ \hline

\end{longtable}
\end{footnotesize}
\justifying
\FloatBarrier

\subsection{Langkah Pengujian}
\label{subsec:test_1_langkah}

Pengujian dilakukan dengan beberapa tahap, yaitu perkenalan, eksplorasi, pengerjaan \textit{task}, lalu diakhiri dengan penutupan. Rancangan pengujian yang lebih detail dapat dilihat pada Lampiran \ref{chpt:testing_hifi}. Berikut adalah penjelasan untuk setiap tahap pengujian

\begin{enumerate}
  \item Perkenalan
  \subitem Pada tahap ini, dilakukan perkenalan diri serta pengarahan singkat kepada partisipan. Pengarahan akan berisi pemaparan tentang prototipe yang akan diuji serta penjelasan terkait prosedur pengujian. 

  \item Eksplorasi
  \subitem Pada tahap ini, partisipan diberi kesempatan untuk melakukan eksplorasi terhadap prototipe yang akan diuji. Hal ini bertujuan agar partisipan memiliki wawasan yang cukup tentang aplikasi sebelum pengujian.

  \item Pengerjaan \textit{task}
  \subitem Pada tahap ini, partisipan mulai mengerjakan \textit{task-task} yang diberikan. Detail lebih lengkap dari task dapat dilihat pada Lampiran \ref{chpt:skenario_hifi1}. Selain itu, setiap kali partisipan selesai mengerjakan sebuah \textit{task}, mereka diminta untuk mengisi \textit{post-task questionnaire} yaitu \textit{Single Ease Question} (SEQ). SEQ menggunakan pertanyaan dengan jawaban \textit{likert-scale} untuk mengukur tingkat kemudahan penggunaan fitur dalam aplikasi.

  \item Pengisian \textit{post-test questionnaire}
  \subitem Pada tahap ini, partisipan telah selesai mengerjakan seluruh \textit{task} yang diberikan penguji. Partisipan akan diminta untuk mengisi \textit{post-test questionnaire} dalam bentuk \textit{System Usability Scale} (SUS) dan \textit{Intrinsic Motivation Inventory} (IMI). 

  \item Penutupan
  \subitem Pada tahap ini, pengujian diakhiri dengan penyampaian kesan pesan dari partisipan, serta ucapan terima kasih dari penguji. 

\end{enumerate}


\subsection{Hasil Pengujian Prototipe \textit{High-Fidelity} Iterasi Pertama}
\label{subsec:test_1_hasil}

Pengujian prototipe \textit{high-fidelity} iterasi pertama dilakukan dengan 5 (lima) orang, sesuai dengan perkataan dari \textcite{nielsenusabilityproblems} yang menyebutkan bahwa pengujian dengan 5 (lima) orang partisipan sudah cukup untuk menemukan rata-rata 85\% masalah dari desain, dalam hal ini prototipe \textit{high-fidelity}. Hasil pengujian lengkap dapat dilihat pada Lampiran \ref{chpt:hasil_test_hifi1}. Dari pengujian, didapatkan beberapa temuan penting dari partisipan tentang prototipe, yang rangkumannya dapat dilihat pada Tabel \ref{tab:daftar_temuan_hifi}
% Pengujian prototipe \textit{high-fidelity} iterasi pertama dilakukan dengan 5 (lima) orang, sesuai dengan perkataan dari \textcite{nielsenusabilityproblems} yang menyebutkan bahwa pengujian dengan 5 (lima) orang partisipan sudah cukup untuk menemukan rata-rata 85\% masalah dari desain, dalam hal ini prototipe \textit{high-fidelity}. Hasil pengujian lengkap dapat dilihat pada Lampiran \ref{chpt:hasil_test_hifi1}. Dari pengujian, didapatkan beberapa temuan penting dari partisipan tentang prototipe, yang rangkumannya dapat dilihat pada Tabel \ref{tab:daftar_temuan_hifi}


\RaggedLeft
\begin{footnotesize}
\begin{longtable}[c]{|W{c}{0.12\textwidth}|>{\ccnormspacingcenter}m{0.8\textwidth}|}
  \caption{Daftar Temuan Penting Pengujian Prototipe \textit{High-Fidelity} Iterasi Pertama}
  \label{tab:daftar_temuan_hifi} \\
  \hline \rowcolor[HTML]{A3E5F5}
  \textbf{Partisipan} & \textbf{Temuan Penting} \\ \hline \endfirsthead
  \hline \rowcolor[HTML]{A3E5F5}
  \textbf{Partisipan} & \textbf{Temuan Penting} \\ \hline \endhead
  \hline \endfoot

  1 & \cditem{
    \item Partisipan merasa \textit{widget} Dashboard sebaiknya memuat lebih dari 3 aplikasi teratas
    \item Partisipan merasa penempatan fitur Smartphone Usage Evaluation kurang tepat pada halaman Daily Goal
    } \\ \hline
    
    2 & \cditem{
      \item Partisipan berekspektasi diberikan pengaturan \textit{default} Bedtime Mode berupa opsi jadwal
      \item Partisipan merasa penempatan fitur Smartphone Usage Evaluation kurang tepat pada halaman Daily Goal
    } \\ \hline
    
  3 & \cditem{
    \item Partisipan merasa navigasi dari halaman Dashboard langsung ke App Timer tidak diperlukan dan cukup sulit dibedakan dengan navigasi ke halaman penggunaan aplikasi
  } \\ \hline
  
  4 & \cditem{
    \item Partisipan merasa memerlukan sebuah pesan pengingat sebelum mematikan Focus Mode atau App Timer untuk hari ini
    \item Partisipan merasa evaluasi Daily Goal sebaiknya tidak langsung muncul ketika selesai menentukan Daily Goal
  } \\ \hline
  
  5 & \cditem{
    \item Partisipan merasa penempatan fitur Smartphone Usage Evaluation kurang tepat pada halaman Daily Goal
    \item Partisipan merasa tombol + (plus) pada \textit{widget} App Timer tidak diperlukan
  } \\ \hline

\end{longtable}
\end{footnotesize}
\justifying
\FloatBarrier

Berdasarkan temuan-temuan yang telah disebutkan di atas, maka disusun beberapa rencana perbaikan untuk direalisasikan dalam prototipe \textit{high-fidelity} iterasi kedua. Pada Tabel \ref{tab:daftar_perbaikan_hifi} dapat ditemukan daftar masalah yang disimpulkan dari temuan-temuan penting, beserta rencana perbaikan yang berkaitan.

\RaggedLeft
\begin{footnotesize}
  \begin{longtable}[c]{|W{c}{0.065\textwidth}|>{\ccnormspacing}m{0.3\textwidth}|>{\ccnormspacing}m{0.32\textwidth}|>{\ccnormspacingcenter}m{0.07\textwidth}|>{\ccnormspacingcenter}m{0.09\textwidth}|}
  \caption{Daftar Rencana Perbaikan Prototipe \textit{High-Fidelity} Iterasi Pertama}
  \label{tab:daftar_perbaikan_hifi} \\
  \hline \rowcolor[HTML]{A3E5F5}
  \textbf{ID} & \centering\textbf{Kesimpulan Masalah dari Temuan} & \centering\textbf{Rencana Perbaikan} & \textbf{\textit{Goals}} & \textbf{Prinsip Desain} \\ \hline \endfirsthead
  \hline \rowcolor[HTML]{A3E5F5}
  \textbf{ID} & \centering\textbf{Kesimpulan Masalah dari Temuan} & \centering\textbf{Rencana Perbaikan} & \textbf{\textit{Goals}} & \textbf{Prinsip Desain} \\ \hline \endhead
  \hline \endfoot

  PH-01 & Kurangnya opsi \textit{default} sebagai rekomendasi bagi pengguna untuk Bedtime Mode & Mengatur opsi \textit{default} pada Bedtime Mode menjadi Based on schedule & - & DP-01 \\ \hline
  PH-02 & Kurangnya fungsionalitas dari \textit{widget} Dashboard & Hal ini merupakan pilihan desain yang disengaja untuk mempertahankan kesederhanaan dari sebuah \textit{widget} & G-01 & - \\ \hline
  PH-03 & Adanya navigasi ke halaman App Timer dari halaman Dashboard mengganggu navigasi ke halaman penggunaan aplikasi & Menghapus navigasi dari halaman Dashboard ke halaman Pengaturan Jadwal App Timer, tanpa menghapus indikator App Timer & G-03 & - \\ \hline
  PH-04 & Kurangnya elemen yang menjaga pengguna dari kesalahan aksi pada fungsi yang cukup kritis & Menambahkan pesan pengingat untuk aksi yang cukup kritis seperti mematikan Focus Mode atau App Timer & - & DP-06 \\ \hline
  PH-05 & Kurangnya penundaan waktu untuk memunculkan evaluasi dari penentuan Daily Goal  & Menunda memunculkan evaluasi tepat setelah menentukan Daily Goal & - & DP-06 \\ \hline
  PH-06 & Kurang tepatnya penempatan fitur Smartphone Usage Evaluation pada halaman Daily Goal & Memindahkan fitur Smartphone Usage Evaluation dari halaman Daily Goal ke halaman baru & G-01, G-02, G-04 & - \\ \hline
  PH-07 & Tidak diperlukannya tombol penambahan App Timer untuk \textit{widget} App Timer & Menghapus tombol penambahan App Timer dari \textit{widget} App Timer & G-01, G-03 & - \\ \hline
  
\end{longtable}
\end{footnotesize}
\justifying
\FloatBarrier

\newpage

\subsection{Analisis Hasil Pengujian Prototipe \textit{High-Fidelity} Iterasi Pertama}
\label{subsec:test_1_analisis}

Dari hasil pengujian pada prototipe \textit{high-fidelity} iterasi pertama, didapatkan beberapa skor dan temuan penting dari pengguna menurut kriteria-kriteria pengujian yang telah disebutkan pada Tabel \ref{tab:daftar_pengujian_goals}. Berikut adalah beberapa penjelasannya

\begin{enumerate}
  \item \textit{Single Ease Question} (SEQ)
  \subitem  Penilaian SEQ digunakan untuk mengetahui tingkat kemudahan sebuah \textit{task} untuk dapat diselesaikan oleh partisipan. Berdasarkan hasil nilai SEQ yang terdapat pada Gambar \ref{img:seq_1}, terlihat bahwa nilai yang diberikan setiap partisipan terhadap kemudahan pengerjaan setiap task sudah cukup baik, dengan nilai rata-rata keseluruhan \textit{task} sebesar 6.56, di atas rata-rata nilai SEQ yang dianjurkan yaitu antara 5.3 hingga 5.6 (mengacu studi literatur Bab \ref{subsubsec:seq}). Adapun \textit{task} dengan nilai rata-rata terendah sebesar 5.7 pada \textit{task} 7. Oleh karena itu, aplikasi dinilai memiliki \textit{learnability} yang cukup baik. 

  \begin{figure}[h]
    \centering
    \includegraphics[width=0.6\textwidth]{hifi/hasil-seq.png}
    \caption{Hasil \textit{Single Ease Question} Pengujian Prototipe \textit{High-Fidelity} Iterasi Pertama}
    \label{img:seq_1}
  \end{figure}
  \FloatBarrier

  \item \textit{System Usability Scale} (SUS)
  \subitem  Berdasarkan hasil nilai SUS yang tertera pada Gambar \ref{img:sus_1} terlihat bahwa nilai yang diberikan setiap partisipan untuk \textit{usability} dari prototipe sudah baik dengan nilai rata-rata 82,5. Menurut studi pada Bab \ref{subsubsec:sus}, penilaian menyatakan \textit{usability} prototipe \textit{high-fidelity} iterasi pertama sudah dapat diterima dan termasuk kategori B (\textit{Excellent}). Selain itu, ditemukan bahwa nilai SUS terendah adalah 65. Maka dari itu, desain solusi aplikasi masih ada potensi untuk diperbaiki dalam prototipe \textit{high-fidelity} iterasi kedua agar dapat memiliki \textit{usability} yang lebih baik. 

  \begin{figure}[h]
    \centering
    \includegraphics[width=0.6\textwidth]{hifi/hasil-sus.png}
    \caption{Hasil \textit{System Usability Scale} Pengujian Prototipe \textit{High-Fidelity} Iterasi Pertama}
    \label{img:sus_1}
  \end{figure}
  \FloatBarrier

  \subitem Adapun pertanyaan nomor 2, 3, dan 8 dari kuesioner SUS yang berhubungan dengan efisiensi sebuah sistem digunakan untuk mengukur ketercapaian \textit{usability goal efficiency}. Pertanyaan lengkap dari kuesioner SUS dapat dilihat pada Lampiran \ref{chpt:testing_hifi}. Jawaban dari setiap partisipan dapat dilihat pada Gambar \ref{img:sus_eff_1}.

  \begin{figure}[h]
    \centering
    \includegraphics[width=0.9\textwidth]{hifi/hasil-sus-efficiency.png}
    \caption{Jawaban \textit{System Usability Scale} Pertanyaan Nomor 2, 3, dan 8 untuk Prototipe \textit{High-Fidelity} Iterasi Pertama}
    \label{img:sus_eff_1}
  \end{figure}
  \FloatBarrier

  \subitem Dapat dilihat bahwa 3 orang tidak setuju dan 1 orang sangat tidak setuju bahwa prototipe dirasa terlalu rumit untuk dipakai, 3 orang sangat setuju bahwa prototipe mudah untuk dipakai, serta 4 orang sangat tidak setuju bahwa prototipe tidak praktis untuk dipakai. Berdasarkan ketiga pertanyaan kuesioner tersebut, maka dapat dinyatakan bahwa desain solusi masih ada potensi untuk diperbaiki dalam prototipe \textit{high-fidelity} iterasi kedua agar dapat mencapai \textit{usability goal efficiency} dengan lebih baik. 

  \item \textit{Intrinsic Motivation Inventory} (IMI)
  \subitem  Berdasarkan Gambar \ref{img:imi1_1} dapat dilihat dari metrik IMI untuk subskala \textit{Value/Usefulness}, para partisipan setuju rancangan prototipe \textit{high-fidelity} iterasi pertama sudah mengarah kepada \textit{user experience goal helpful} dengan baik dengan skor terendah sebesar 6,14.
  %  namun tetap dibutuhkan beberapa perbaikan yang perlu diperhatikan kembali.
  
  \begin{figure}[h]
    \centering
    \includegraphics[width=0.6\textwidth]{hifi/hasil-imi1.png}
    \caption{Hasil \textit{Intrinsic Motivation Inventory} Subskala \textit{Value/Usefulness} Pengujian Prototipe \textit{High-Fidelity} Iterasi Pertama}
    \label{img:imi1_1}
  \end{figure}
  \FloatBarrier
  
  \subitem  Berdasarkan Gambar \ref{img:imi2_1} dapat dilihat bahwa skor IMI untuk subskala \textit{Interest/Enjoyment} yang didapatkan dari setiap partisipan cukup variatif, dengan skor terendah sebesar 4,29. Di sisi lain, dari skor IMI untuk subskala \textit{Pressure/Tension} ditemukan 3 partisipan menyatakan sangat tidak setuju bahwa prototipe membuatnya merasa tertekan, dan terdapat 1 partisipan yang menilai skor 2,40. Kedua hal tersebut dapat menunjukkan bahwa rancangan prototipe \textit{high-fidelity} mulai mengarah kepada \textit{user experience goal motivating}, namun perlu perbaikan dalam prototipe \textit{high-fidelity} iterasi kedua agar dapat mencapai \textit{user experience goal motivating} dengan lebih baik.

  \begin{figure}[h]
    \centering
    \includegraphics[width=\textwidth]{hifi/hasil-imi2.png}
    \caption{Hasil \textit{Intrinsic Motivation Inventory} Subskala \textit{Interest/Enjoyment} dan \textit{Pressure/Tension} Pengujian Prototipe \textit{High-Fidelity} Iterasi Pertama}
    \label{img:imi2_1}
  \end{figure}
  \FloatBarrier

\end{enumerate}




