\chapter{Rincian Analisis Permasalahan}
\label{chpt:rincian_analisis_permasalahan}

Analisis terhadap permasalahan yang telah disebutkan pada subbab \ref{sec:analisis_masalah} adalah sebagai berikut:

\begin{enumerate}
  \item M-001: Fitur "Take a break" dari Focus Mode dapat disalahgunakan untuk mengambil istirahat terus menerus
  \subitem Fitur ”Take a break” yang berbentuk tombol ini dapat memberikan pengguna kembali akses untuk aplikasi yang diblok dengan pilihan waktu 5 menit, 15 menit, atau 30 menit. Tombol ini dapat digunakan jika pengguna ingin beristirahat dan menggunakan aplikasi yang diblok. Namun, fitur ini berpotensi untuk disalahgunakan oleh pengguna untuk mengambil istirahat terus menerus. Pengguna cukup mengambil pilihan istirahat selama 30 menit, dan mengambil lagi tepat setelah waktunya habis untuk memperpanjang waktu istirahatnya.
  
  \item M-002: Fitur "Turn off for now" dari Focus Mode dapat disalahgunakan untuk mematikan Focus Mode sebelum tenggat waktu yang ditentukan
  \subitem Fitur ”Turn off for now” yang berbentuk tombol dapat digunakan untuk memberhentikan Focus Mode hanya untuk hari tersebut. Tombol ini dapat digunakan jika pengguna merasa kegiatannya sudah selesai lebih awal dari yang telah dijadwalkan dan ingin menggunakan sisa harinya untuk mengakses kembali aplikasi yang diblokir. Namun fitur ini dapat disalahgunakan untuk menghindari Focus Mode secara keseluruhan.
   
  \item M-003: Memberikan sugesti atas langkah yang dapat diambil untuk meningkatkan kualitas Digital Wellbeing
  \subitem Aplikasi Digital Wellbeing telah memiliki fungsionalitas yang melaksanakan salah satu cara penerapan konsep Digital Wellbeing, yaitu memberikan pantauan terhadap penggunaan teknologi digital. Hal ini direpresentasikan dengan sebuah ringkasan dari penggunaan aplikasi per harinya dengan metrik jumlah waktu penggunaan aplikasi, jumlah notifikasi yang diterima, dan jumlah pembukaan aplikasi. Namun ringkasan tersebut hanya memberikan informasi kepada pengguna tanpa adanya penilaian apakah performa dari pengguna terbilang baik atau buruk. Hal ini kurang memotivasi pengguna untuk memulai membuat evaluasi terhadap kebiasaannya karena pengguna belum tentu dapat menilai kesehatan dari kebiasaan digitalnya.
   
\end{enumerate}