\section{Adiksi \textit{Smartphone}}
Seperti yang telah disebutkan pada subbab \ref{sec:latarbelakang}, \textit{smartphone} yang telah menjadi bagian dari kehidupan sehari-hari manusia memiliki banyak fungsionalitas yang dapat meningkatkan kualitas hidup manusia, tapi di sisi lain dapat memberikan pengaruh buruk. Efek negatif yang timbul dari pengaruh-pengaruh buruk tersebut menunjukkan kemiripan pada pola-pola perilaku korban adiksi. Menurut penelitian tentang \textit{Smartphone Addiction Scale} oleh \textcite{10.1371/journal.pone.0083558}, walaupun \textit{smartphone addiction} belum terdaftar sebagai \textit{behavioral addiction} dalam DSM-5 (\textit{Diagnostic and Statistical Manual of Mental Disorders}), sebuah standar klasifikasi terhadap penyakit mental yang digunakan oleh ahli kesehatan mental di Amerika Serikat, adiksi untuk aktivitas yang dapat dilakukan pada internet melalui \textit{smartphone}, seperti bermain \textit{game}, \textit{chatting}, dan pornografi menunjukkan tingkat adiksi yang sama dengan korban adiksi narkotika dan alkohol.

Menurut penelitian oleh \textcite{CHI2019SOCIALIZE}, penggunaan \textit{smartphone} yang berlebihan menimbulkan pengaruh negatif terhadap kesehatan mental dan interaksi sosial. Hal ini terlihat pada kualitas interaksi sesama secara langsung, yang biasanya membutuhkan usaha dan komitmen untuk menjalin hubungan baik, terpengaruh oleh konsep interaksi tidak langsung melalui \textit{smartphone}, di mana hubungan lebih menyebar dengan lebih sedikit interaksi.

Roffarello dan De Russis melanjutkan \textit{smartphone} juga sering berperan sebagai sumber distraksi yang mengalihkan perhatian dari kegiatan penting. Distraksi tersebut dapat berasal dari stimuli eksternal seperti notifikasi pada \textit{smartphone}, namun dapat juga dari stimuli internal seperti keinginan untuk memeriksa \textit{e-mail} atau bermain \textit{game}. Pengguna yang merasakan gangguan internal dan eksternal rutin dan tidak dapat diprediksi ini cenderung merasa tidak produktif dan lebih sering stress.
