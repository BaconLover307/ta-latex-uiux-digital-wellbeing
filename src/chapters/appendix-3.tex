\chapter{Rincian Analisis Solusi}
\label{chpt:rincian_analisis_solusi}

Analisis terhadap solusi yang telah disebutkan pada subbab \ref{sec:analisis_solusi} adalah sebagai berikut:

\begin{enumerate}
  \item S-001: Memberikan langkah tambahan setiap kali pengguna mengakses fungsionalitas "Take a break"
  \subitem
  Tujuan utama dari tombol "Take a break" adalah untuk pengguna dapat mengambil istirahat dari kegiatan utamanya dan mengakses kembali aplikasi yang diblokir, atau jika sekali-kali pengguna butuh menggunakan aplikasi tersebut. Namun pengaksesan berkali-kali terhadap fungsionalitas ini dapat berarti pengguna sedang menyalahgunakannya. Maka dari itu, jika diberikan sebuah langkah tambahan setiap kali pengguna mengambil istirahat maka diharapkan pengguna akan semakin sulit mengaksesnya sehingga menghindari aplikasi distraksi yang diblokir. Hal ini memanfaatkan \textit{user experience goal} yang tidak diharapkan, \textit{frustrating}, untuk menjauhkan pengguna dari mengakses kembali aplikasi distraksi.
  
  \item S-002: Memindahkan fungsionalitas dari tombol "Turn off for now" ke pengaturan aplikasi
  \subitem
  Tujuan utama dari tombol "Turn off for now" adalah untuk pengguna dapat mengakses kembali aplikasi-aplikasi distraksi lebih awal dari tenggat waktu yang telah ditentukan. Namun, lokasi tombol tersebut yang mudah diakses yaitu pada Notification Bar membuat fitur tersebut mudah disalahgunakan pengguna. Maka dari itu dengan memindahkan fungsionalitas ke halaman aplikasi yang lebih dalam membuat pengguna perlu langkah lebih banyak untuk mematikan fitur Focus Mode. Seperti sebelumnya, hal ini memanfaatkan \textit{user experience goal} yang tidak diharapkan, \textit{frustrating}, untuk menjauhkan pengguna dari mengakses kembali aplikasi distraksi. 
  
  \item S-003: Memberikan sugesti atas langkah yang dapat diambil untuk meningkatkan kualitas pola penggunaan aplikasi
  \subitem Fungsionalitas aplikasi Digital Wellbeing yang hanya memberikan ringkasan dari penggunaan aplikasi kurang dapat memotivasi pengguna untuk melakukan evaluasi terhadap kebiasaannya. Maka dari itu, penilaian lebih atas kebiasaan pengguna dapat memberikan bayangan terhadap apa saja yang bisa ditingkatkan. Penilaian tersebut dapat berbentuk sebuah sugesti kepada pengguna untuk mengambil langkah-langkah tertentu, seperti mengutilisasi fitur lain yang terdapat pada aplikasi Digital Wellbeing. Hal ini memanfaatkan \textit{user experience goal} yang diharapkan, \textit{motivating}, untuk memotivasi pengguna untuk meningkatkan kualitas pola penggunaan aplikasi-aplikasi pada \textit{smartphone}.

\end{enumerate}