\chapter{Skenario Pengujian Prototipe \textit{Low-Fidelity}}
\label{chpt:skenario_lofi}

Setelah setiap pengerjaan skenario, partisipan akan diberikan pertanyaan mengenai skenario dan tugas yang telah mereka kerjakan. Pertanyaan ini berguna untuk mengevaluasi alur prototipe, informasi yang terdapat pada prototipe, serta mencari saran atau kritik dari partisipan. Pertanyaan-pertanyaannya adalah sebagai berikut

\begin{enumerate}
  \item Apa tanggapan Anda mengenai alur tampilan untuk menyelesaikan task tersebut?
  \item Apakah informasi pada tampilan cukup untuk membantu menyelesaikan task tersebut?
  \item Apakah ada masukan / saran mengenai tampilan dari halaman yang dilalui?
\end{enumerate}

\RaggedLeft
\begin{small}
\begin{longtable}[c]{|>{\ccnormspacing}m{0.19\textwidth}|>{\ccnormspacing}p{0.73\textwidth}|}
  
  \hline
  \rowcolor[HTML]{A3E5F5} \multicolumn{2}{|l|}{\textbf{Skenario Pengujian 1}} \\ \hline
  Kaitan Skenario Pengguna & SP-02 \\ \hline
  Tujuan & Mengukur pemahaman pengguna dalam menganalisis data penggunaan smartphone \\ \hline
  Skenario & Pengguna ingin melihat data penggunaan smartphone harian. \\ \hline
  Task & Lihat data penggunaan smartphone \\ \hline
  Pra kondisi & Pengguna berada di Halaman Main Menu  \\ \hline

  \rowcolor[HTML]{A3E5F5} \multicolumn{2}{|l|}{\textbf{Skenario Pengujian 2}} \\ \hline
  Kaitan Skenario Pengguna & SP-03 \\ \hline
  Tujuan & Mengukur pemahaman pengguna dalam membuat sebuah App Group \\ \hline
  Skenario & Pengguna ingin mengelompokkan aplikasi-aplikasi ke dalam sebuah kategori. \\ \hline
  Task & Kelompokan aplikasi-aplikasi ke dalam sebuah kategori  \\ \hline
  Pra kondisi & Pengguna berada di Halaman Dashboard \\ \hline

  \rowcolor[HTML]{A3E5F5} \multicolumn{2}{|l|}{\textbf{Skenario Pengujian 3}} \\ \hline
  Kaitan Skenario Pengguna & SP-02 \\ \hline
  Tujuan & Mengukur pemahaman pengguna dalam menganalisis data penggunaan satu buah aplikasi \\ \hline
  Skenario & Pengguna sudah melihat data penggunaan smartphone. Pengguna ingin melihat data penggunaan untuk hanya sebuah aplikasi. \\ \hline
  Task & Lihat data penggunaan dari sebuah aplikasi \\ \hline
  Pra kondisi & Pengguna berada di Halaman Dashboard \\ \hline

  \rowcolor[HTML]{A3E5F5} \multicolumn{2}{|l|}{\textbf{Skenario Pengujian 4}} \\ \hline
  Kaitan Skenario Pengguna & SP-02 \\ \hline
  Tujuan & Mengukur pemahaman pengguna dalam menganalisis data penggunaan App Group \\ \hline
  Skenario & Pengguna ingin melihat data penggunaan App Group yang sudah dibuat. \\ \hline
  Task & Lihat data penggunaan dari kategori aplikasi yang sudah dibuat \\ \hline
  Pra kondisi & Pengguna berada di Halaman Dashboard \\ \hline

  \rowcolor[HTML]{A3E5F5} \multicolumn{2}{|l|}{\textbf{Skenario Pengujian 5}} \\ \hline
  Kaitan Skenario Pengguna & SP-03 \\ \hline
  Tujuan & Mengukur pemahaman pengguna dalam memasang App Timer pada sebuah aplikasi \\ \hline
  Skenario & Pengguna ingin membatasi waktu penggunaan sebuah aplikasi. \\ \hline
  Task & Batasi waktu penggunaan dari sebuah aplikasi \\ \hline
  Pra kondisi & Pengguna berada di Halaman Main Menu \\ \hline

  \rowcolor[HTML]{A3E5F5} \multicolumn{2}{|l|}{\textbf{Skenario Pengujian 6}} \\ \hline
  Kaitan Skenario Pengguna & SP-05 \\ \hline
  Tujuan & Mengukur pemahaman pengguna dalam menunda aktivasi App Timer \\ \hline
  Skenario & Pengguna ingin membatasi waktu penggunaan sebuah App Group. \\ \hline
  Task & Tunda pemblokiran untuk aplikasi yang batas waktu penggunaannya sudah habis \\ \hline
  Pra kondisi & Pengguna berada di Halaman Main Menu \\ \hline

  \rowcolor[HTML]{A3E5F5} \multicolumn{2}{|l|}{\textbf{Skenario Pengujian 7}} \\ \hline
  Kaitan Skenario Pengguna & SP-04 \\ \hline
  Tujuan & Mengukur pemahaman pengguna dalam menambah jadwal Focus Mode \\ \hline
  Skenario & Pengguna ingin fokus pada pekerjaannya di jam kerja. Pengguna ingin memblokir beberapa aplikasi di jam kerjanya. \\ \hline
  Task & Buat jadwal pemblokiran aplikasi-aplikasi mendistraksi sesuai dengan jam kerja \\ \hline
  Pra kondisi & Pengguna berada di Halaman Main Menu \\ \hline

  \rowcolor[HTML]{A3E5F5} \multicolumn{2}{|l|}{\textbf{Skenario Pengujian 8}} \\ \hline
  Kaitan Skenario Pengguna & SP-05 \\ \hline
  Tujuan & Mengukur pemahaman pengguna dalam menunda aktivasi Fokus Mode \\ \hline
  Skenario & Pengguna ingin beristirahat dari kerjanya sejenak. Pengguna ingin menggunakan aplikasi yang diblokir secara sementara. \\ \hline
  Task & Tunda pemblokiran aplikasi yang sudah dipasang sebelumnya \\ \hline
  Pra kondisi & Pengguna berada di Halaman Main Menu \\ \hline
  
  \rowcolor[HTML]{A3E5F5} \multicolumn{2}{|l|}{\textbf{Skenario Pengujian 9}} \\ \hline
  Kaitan Skenario Pengguna & SP-01 \\ \hline
  Tujuan & Mengukur pemahaman pengguna dalam menentukan Daily Goal \\ \hline
  Skenario & Pengguna memiliki sebuah target yang harus dipenuhi hari ini. Pengguna ingin mengingatkan diri terhadap capaian tersebut. \\ \hline
  Task & Pasang pengingat untuk capaian yang harus dipenuhi hari ini \\ \hline
  Pra kondisi & Pengguna berada di Halaman Main Menu \\ \hline
  
  \rowcolor[HTML]{A3E5F5} \multicolumn{2}{|l|}{\textbf{Skenario Pengujian 10}} \\ \hline
  Kaitan Skenario Pengguna & SP-06 \\ \hline
  Tujuan & Mengukur kecukupan informasi dalam membantu pengguna mengatur jadwal Bedtime Mode \\ \hline
  Skenario & Pengguna ingin memperbaiki jam tidurnya. Pengguna ingin mengurangi penggunaan smartphone di saat jam tidur. \\ \hline
  Task & Aturlah agar smartphone menjadi lebih hening di saat jam tidur. \\ \hline
  Pra kondisi & Pengguna berada di Halaman Main Menu \\ \hline

\end{longtable}
\end{small}

