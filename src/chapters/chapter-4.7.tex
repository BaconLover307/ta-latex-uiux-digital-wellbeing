\section{Desain Final Prototipe Solusi}
\label{sec:final}

Berdasarkan hasil pengujian prototipe \textit{high-fidelity} iterasi kedua, dapat ditentukan bahwa solusi desain yang dirancang sudah dapat memenuhi kebutuhan pengguna, sehingga proses \textit{User-Centered Design} sudah dapat diakhiri. Meskipun terdapat temuan penting PH-08 dan PH-09 pada Tabel \ref{tab:daftar_perbaikan_hifi_2}, dapat disebutkan bahwa perbaikan-perbaikan tersebut cukup minor dan tidak perlu dilakukan pengujian ulang. Maka dari itu, perbaikan-perbaikan tersebut dimuat pada desain final dari prototipe solusi, yang dipaparkan pada Tabel \ref{tab:daftar_final_halaman} dan Tabel \ref{tab:daftar_final_widget} berikut.

\newlength{\finalwidth}
\setlength{\finalwidth}{0.325\textwidth}

\newlength{\finaldescwidth}
\setlength{\finaldescwidth}{0.33\textwidth}

\newcommand{\finaldesc}[1]{\desc{\finaldescwidth}{#1}}

\newcommand{\final}[1]{\begin{center}\includegraphics[width=\finalwidth]{#1}\end{center}}
\newcommand{\finalwidget}[2]{\begin{center}\includegraphics[width=#1]{#2}\end{center}}

\RaggedLeft
\begin{footnotesize}
\begin{longtable}[c]{|>{\ccnormspacingcenter}p{0.12\textwidth}|>{\ccnormspacing}p{\finaldescwidth}|>{\ccnormspacingcenter}p{0.12\textwidth}|>{\ccnormspacingcenter}p{\finalwidth}|}
  \caption{Daftar Tampilan Halaman Prototipe \textit{High-Fidelity}}
  \label{tab:daftar_final_halaman} \\
  \hline \rowcolor[HTML]{A3E5F5}
  \centering\textbf{Halaman} & \centering\textbf{Penjelasan Halaman} & \centering\textbf{\textit{Goals}} & \textbf{Prototipe \textit{High-Fidelity}} \\ \hline \endfirsthead
  \hline \rowcolor[HTML]{A3E5F5}
  \centering\textbf{Halaman} & \centering\textbf{Penjelasan Halaman} & \centering\textbf{\textit{Goals}} & \textbf{Prototipe \textit{High-Fidelity}} \\ \hline \endhead
  \hline \endfoot

  \textbf{H-01} Halaman Main Menu & 
    \finaldesc{
      Halaman ini adalah tampilan utama dari aplikasi Digital Wellbeing yang memuat navigasi utama ke fitur-fitur lainnya. Pada prototipe \textit{high-fidelity} terdapat perubahan pada bentuk menu navigasi menjadi lebih bundar agar lebih \textit{user-friendly}. Adapun implementasi ilustrasi untuk menu Focus Mode dan Bedtime Mode, dan ikon-ikon pada menu lainnya, menggantikan \textit{placeholder} pada tampilan \textit{low-fidelity}.
    } & G-02, G-04 & \final{hifi/h-01} \\ \hline
  
  \end{longtable}
\end{footnotesize}
\justifying
\FloatBarrier

\RaggedLeft
\begin{footnotesize}
\begin{longtable}[c]{|>{\ccnormspacingcenter}p{0.12\textwidth}|>{\ccnormspacing}p{\finaldescwidth}|>{\ccnormspacingcenter}p{0.12\textwidth}|>{\ccnormspacingcenter}p{\finalwidth}|}
  \caption{Daftar Tampilan Widget Desain Final Solusi}
  \label{tab:daftar_final_widget} \\
  \hline \rowcolor[HTML]{A3E5F5}
  \centering\textbf{Widget} & \centering\textbf{Penjelasan Widget} & \centering\textbf{\textit{Goals}} & \textbf{Prototipe \textit{High-Fidelity}} \\ \hline \endfirsthead
  \hline \rowcolor[HTML]{A3E5F5}
  \centering\textbf{Widget} & \centering\textbf{Penjelasan Widget} & \centering\textbf{\textit{Goals}} & \textbf{Prototipe \textit{High-Fidelity}} \\ \hline \endhead
  \hline \endfoot

  \textbf{H-01} Halaman Main Menu & 
    \finaldesc{
      Halaman ini adalah tampilan utama dari aplikasi Digital Wellbeing yang memuat navigasi utama ke fitur-fitur lainnya. Pada prototipe \textit{high-fidelity} terdapat perubahan pada bentuk menu navigasi menjadi lebih bundar agar lebih \textit{user-friendly}. Adapun implementasi ilustrasi untuk menu Focus Mode dan Bedtime Mode, dan ikon-ikon pada menu lainnya, menggantikan \textit{placeholder} pada tampilan \textit{low-fidelity}.
    } & G-02, G-04 & \final{hifi/h-01} \\ \hline
  
  \end{longtable}
\end{footnotesize}
\justifying
\FloatBarrier
