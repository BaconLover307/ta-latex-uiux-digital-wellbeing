\section{Identifikasi Konteks Penggunaan}
\label{sec:identifikasi_konteks_penggunaan}

Pada tahap ini dilakukan proses analisis pengguna melalui data yang didapatkan dari ulasan pengguna aplikasi Google Digital Wellbeing dari situs Google Play Store, serta riset dengan metode wawancara. Kemudian akan dilakukan penyusunan persona pengguna dan pengidentifikasian fungsionalitas aplikasi yang akan membantu menentukan kebutuhan dan tujuan pengguna, serta skenario pengguna.

\subsection{Riset dan Analisis Pengguna}
\label{subsec:riset_analisis}

Tahap ini bertujuan untuk menganalisis target pengguna, sehingga didapatkan data perilaku dan masalah pengguna aplikasi Google Digital Wellbeing. Riset dilakukan dengan mengumpulkan data ulasan pengguna aplikasi Google Digital Wellbeing dari situs Google Play Store \textcite{dwplaystorereviews}. Metode pengumpulan data ini dipilih dengan alasan mengacu pada ISO 9241-210, bahwa informasi yang sudah tersedia dari suatu produk dapat dimanfaatkan untuk melakukan modifikasi atau peningkatan kualitas produk \parencite{iso9241-210:2010}, dalam hal ini informasi berbentuk ulasan pengguna.

Hasil terhadap analisis ulasan pengguna kemudian divalidasi dengan wawasan yang didapat dari wawancara. Karena pada data ulasan pengguna dari Google Play Store tidak terdapat informasi mengenai umur dan perilaku pengguna, wawancara dengan pengguna yang termasuk ke dalam lingkup permasalahan perlu dilakukan juga untuk mendapatkan wawasan mengenai perilaku pengguna. Perilaku pengguna berguna untuk menyusun persona pengguna.

\subsubsection{Ulasan Pengguna}
\label{subsubsec:ulasan_pengguna}

Berdasarkan situs Google Play Store per 13 April 2022, terdapat 609.005 ulasan untuk aplikasi Google Digital Wellbeing. Dikumpulkan data sebanyak 1000 ulasan dari target pengguna yang telah disebutkan pada bab \ref{sec:identifikasi_konteks_penggunaan},  kemudian dikategorikan secara manual, lalu didapatkan 288 ulasan yang dapat digunakan untuk menyusun masalah pengguna. Dihasilkan 11 kategori ulasan yang rinciannya dapat dilihat pada tabel \ref{tab:daftar_kategori}.

\newpage

\RaggedLeft
\begin{footnotesize}
\begin{longtable}[c]{|W{c}{0.08\textwidth}|>{\ccnormspacing}m{0.72\textwidth}|>{\ccnormspacingcenter}m{0.1\textwidth}|}
  \caption{Daftar Kategori Ulasan}
  \label{tab:daftar_kategori} \\
  \hline \rowcolor[HTML]{A3E5F5} \textbf{ID} & \centering\textbf{Kategori Ulasan} & \textbf{Jumlah Ulasan} \\ \hline \endfirsthead
  \hline \rowcolor[HTML]{A3E5F5} \textbf{ID} & \centering\textbf{Kategori Ulasan} & \textbf{Jumlah Ulasan} \\ \hline \endhead
  
  \hline \endfoot
  
  KU-01    & Kurangnya \textit{widget} untuk menampilkan data di Homescreen & 63 \\ \hline
  KU-02    & Perlu dikembangkannya fitur laporan penggunaan aplikasi untuk menampilkan data & 47 \\ \hline
  KU-03    & Perlu dikembangkannya fitur Focus Mode dengan menambah keketatan & 47 \\ \hline
  KU-04    & Perlu dikembangkannya kemampuan penjadwalan untuk fitur App Timer dan Focus Mode & 36 \\ \hline
  KU-05    & Kurangnya fitur pengaturan tingkat keketatan untuk fitur-fitur & 29 \\ \hline
  KU-06    & Kurangnya kemampuan penundaan untuk fitur App Timer & 18 \\ \hline
  KU-07    & Perlu dikembangkannya fitur Bedtime Mode dengan menambah keketatan & 12 \\ \hline
  KU-08    & Kurangnya penjelasan atau susunan kata yang dapat memotivasi pengguna untuk memakai fitur-fitur aplikasi & 12 \\ \hline
  KU-09    & Kurangnya fitur pengelompokkan aplikasi & 11 \\ \hline
  KU-10    & Kurangnya fitur pengaturan jam untuk akhir sebuah hari & 7 \\ \hline
  KU-11    & Kurangnya fitur \textit{whitelisting} untuk pembatasan akses aplikasi oleh Focus Mode & 6 \\ \hline
\end{longtable}
\end{footnotesize}
\justifying
\FloatBarrier

Adapun sejumlah ulasan yang tidak tergolong dalam kategori karena dinilai tidak relevan dalam penyusunan masalah pengguna, dengan penjelasan sebagai berikut

\begin{enumerate}
  \item Ulasan yang menilai positif aplikasi tanpa menyebutkan adanya masalah yang ditemukan dari aplikasi
  \item Ulasan yang menilai negatif aplikasi tanpa menyebutkan masalah yang ditemukan dari aplikasi
  \item Ulasan yang menyebutkan adanya bug dari aplikasi, seperti tidak berfungsinya sebuah fitur di perangkat tertentu
  \item Ulasan yang menyebutkan masalah yang tidak termasuk ke dalam batasan tugas akhir
  \item Ulasan yang tidak dapat dimengerti, seperti huruf-huruf yang tersusun secara acak 
  \item Ulasan yang melaporkan bahwa aplikasi tidak dapat dihapus dari perangkat
\end{enumerate}

\subsubsection{Perilaku Pengguna}
Setelah melakukan analisis ulasan pengguna untuk aplikasi Google Digital Wellbeing, dilakukan wawancara untuk mendapatkan perilaku pengguna, memvalidasi masalah pengguna dari analisis ulasan, serta menemukan masalah lain yang tidak ditemukan dari ulasan. Target narasumber berjumlah 5 (lima) orang dengan kriteria sebagaimana telah dijelaskan pada subbab \ref{sec:perencanaan_proses_desain}. Jumlah tersebut dipilih karena menurut \textcite{nielsenusabilityproblems}, penelitian dengan 5 (lima) orang responden sudah cukup untuk menemukan rata-rata 85\% masalah dari desain sebuah produk, dan menambah responden lebih banyak akan mendapatkan wawasan tambahan yang semakin sedikit.

Data perilaku responden digunakan untuk menyusun persona pengguna, serta membantu menganalisis kebutuhan pengguna terkait aplikasi Digital Wellbeing. Data perilaku tentang aplikasi pencegah distraksi penting untuk dianalisis karena salah satu tujuan dari aplikasi Google Digital Wellbeing adalah mencegah distraksi terkait \textit{smartphone}, sehingga perilaku tentang aplikasi serupa dapat memberikan wawasan yang berguna. Selain itu, hasil validasi ulasan pengguna digunakan untuk menggali inti masalah yang dikeluhkan. Kedua pengamatan tersebut akan dibahas lebih lanjut dalam analisis masalah, kebutuhan, dan tujuan pengguna. Rancangan pertanyaan dapat dilihat pada Lampiran \ref{chpt:daftar_pertanyaan_wawancara}. Detail pemetaan pengamatan dengan pertanyaan wawancara dapat dilihat pada Tabel \ref{tab:pemetaan_pengamatan_wawancara}.

\RaggedLeft
\begin{footnotesize}
\begin{longtable}[c]{|>{\ccnormspacing}m{0.72\textwidth}|p{0.2\textwidth}|}
  \caption{Pemetaan Pengamatan dengan Pertanyaan Wawancara}
  \label{tab:pemetaan_pengamatan_wawancara} \\
  \hline \rowcolor[HTML]{A3E5F5} \multicolumn{1}{|c|}{\textbf{Pengamatan}} & \multicolumn{1}{|c|}{\textbf{No. Pertanyaan}} \\ \hline \endfirsthead
  \hline \rowcolor[HTML]{A3E5F5} \multicolumn{1}{|c|}{\textbf{Pengamatan}} & \multicolumn{1}{|c|}{\textbf{No. Pertanyaan}} \\ \hline \endhead

  \hline \endfoot
  
  \rowcolor[HTML]{DCF3FC} \multicolumn{2}{|l|}{\textbf{A. Perilaku Responden}} \\ \hline
  Identitas responden & 1, 2, 3 \\ \hline
  Perilaku penggunaan \textit{smartphone} responden & 4, 5, 6, 7, 8 \\ \hline
  Perilaku responden terkait aplikasi pencegah distraksi & 9, 10, 11, 12 \\ \hline
  Perilaku responden terkait aplikasi Digital Wellbeing & 13, 14, 15 \\ \hline
  \rowcolor[HTML]{DCF3FC} \multicolumn{2}{|l|}{\textbf{B. Validasi Ulasan Aplikasi Digital Wellbeing}} \\ \hline
  Validasi masalah kurangnya fitur \textit{widget} pada Homescreen & 16, 17, 18 \\ \hline
  Validasi masalah pada fitur laporan data penggunaan aplikasi pada \textit{smartphone} & 19 \\ \hline
  Validasi masalah pada fitur Focus Mode & 20, 21 \\ \hline
  Validasi masalah untuk kemampuan penjadwalan pada fitur-fitur & 22, 23, 24 \\ \hline
  Validasi masalah kurangnya fitur pengaturan tingkat keketatan & 25, 26, 27 \\ \hline
  Validasi masalah kurangnya fitur penundaan pada App Timer & 28, 29, 30 \\ \hline
  Validasi masalah pada fitur Bedtime Mode & 31, 32, 33 \\ \hline
  Validasi masalah kurangnya penjelasan dan susunan kata & 34, 35 \\ \hline
  Validasi masalah kurangnya fitur pengelompokkan aplikasi & 36 \\ \hline
  Validasi masalah kurangnya fitur pengaturan jam akhir hari & 37 \\ \hline
  Validasi masalah kurangnya kemampuan \textit{whitelisting} & 38 \\ \hline
\end{longtable}
\end{footnotesize}
\justifying
\FloatBarrier

Jumlah responden yang diwawancarai adalah 10 (sepuluh) orang. Data hasil wawancara dapat dilihat pada Lampiran \ref{chpt:hasil_wawancara}. Dari 10 responden, 90\% mengakui tujuan utama dari penggunaan \textit{smartphone} adalah berkomunikasi melalui aplikasi \textit{messenger}, dengan tujuan sekunder yaitu berinteraksi dengan media sosial atau sebagai sarana hiburan. Ditemukan bahwa 4 dari 10 orang menggunakan \textit{smartphone} sebagai alat utama yang membantu dalam pekerjaan, dengan tujuan untuk berkomunikasi dengan rekan atau klien, menggunakannya sebagai \textit{workstation}, atau mencari ide dan inspirasi.

Dari wawancara, ditemukan bahwa seluruh responden mengakui distraksi terkait dengan \textit{smartphone} lebih banyak berasal dari luar \textit{smartphone} itu sendiri, yaitu dari keinginan diri sendiri menggunakan \textit{smartphone} untuk memenuhi tujuan sekunder mereka. Walaupun hanya 30\% responden yang mengeluhkan notifikasi dari \textit{smartphone} dianggap mendistraksi mereka dari kegiatan utama, 70\% mengakui perlu untuk mencegah distraksi dari notifikasi dengan cara menggunakan aplikasi pencegah distraksi atau mengubah pengaturan notifikasi aplikasinya secara langsung.

Ditemukan juga bahwa rata-rata durasi penggunaan \textit{smartphone} harian keseluruhan responden adalah 6.2 jam per hari, di mana 70\% responden dapat menggunakan \textit{smartphone} selama lebih dari 6 jam sehari. Kedua angka tersebut melebihi rata-rata durasi penggunaan \textit{smartphone} di Indonesia pada tahun 2021 yaitu 5.4 jam per hari \parencite{dataai2022smartphoneindonesia}.  Keseluruhan dari responden menilai skala rata-rata 4 (empat) dari 5 (lima) terhadap durasi penggunaan \textit{smartphone} harian mereka. Di antara seluruh responden, 70\% mengakui butuh bantuan dari sebuah aplikasi pencegah distraksi untuk menurunkan durasi penggunaan tersebut.

Selain untuk mengurangi durasi penggunaan, responden juga memerlukan bantuan sebuah aplikasi untuk melakukan hal lain yang berhubungan dengan perbaikan kebiasaan digital mereka. Ditemukan 50\% responden memerlukan bantuan aplikasi untuk memantau penggunaan \textit{smartphone}, baik secara keseluruhan atau per aplikasi, dalam merencanakan perbaikan digital mereka. Lalu, 80\% dari responden merasa perlu diingatkan tentang tugas / kegiatan yang harus diselesaikan saat mereka menggunakan \textit{smartphone}. Keberadaan sebuah pengingat dapat menyadarkan pengguna terhadap alasan mereka menggunakan aplikasi pencegah distraksi. Selain itu, 50\% dari responden juga menggunakan aplikasi untuk membantu dalam memperbaiki jadwal tidurnya. Mereka mengakui bahwa saat menggunakan \textit{smartphone} sebelum tidur, seringkali mereka tidak menyadari waktu sehingga melewati jadwal tidur mereka.

Wawasan tentang perilaku pengguna yang didapat disusun dalam bentuk variabel-variabel perilaku yang berguna dalam pembentukan persona. \textcite{cooper2014face} menyarankan bahwa dalam menyusun variabel dapat ditemukan perbedaan yang cukup jelas jika fokus kepada tipe-tipe variabel berikut
\begin{enumerate}
  \item Aktivitas, yaitu apa yang dilakukan pengguna dan seberapa sering 
  \item Sikap, yaitu pendapat pengguna tentang domain produk dan teknologi
  \item Kemampuan, yaitu bakat pengguna dan kemampuan untuk belajar
  \item Motivasi, yaitu alasan keterlibatan pengguna pada domain produk 
  \item Keterampilan, yaitu kemampuan pengguna terkait domain produk dan teknologi
\end{enumerate}

Keseluruhan variabel perilaku pengguna dirangkum pada Tabel \ref{tab:perilaku_pengguna}

\newpage

\RaggedLeft
\begin{footnotesize}
\begin{longtable}[c]{|W{c}{0.07\textwidth}|>{\ccnormspacing}m{0.66\textwidth}|>{\ccnormspacingcenter}m{0.165\textwidth}|}
  \caption{Daftar Variabel Perilaku Pengguna}
  \label{tab:perilaku_pengguna} \\
  \hline \rowcolor[HTML]{A3E5F5} \textbf{ID} & \multicolumn{1}{|c|}{\textbf{Variabel Perilaku Pengguna}} & \multicolumn{1}{|c|}{\textbf{Tipe Perilaku}} \\ \hline \endfirsthead
  \hline \rowcolor[HTML]{A3E5F5} \textbf{ID} & \multicolumn{1}{|c|}{\textbf{Variabel Perilaku Pengguna}} & \multicolumn{1}{|c|}{\textbf{Tipe Perilaku}} \\ \hline \endhead
  
  \hline \endfoot
  
  PP-01  &  Menggunakan \textit{smartphone} dengan tujuan primer untuk berkomunikasi melalui aplikasi \textit{messenger}  & Aktivitas \\ \hline
  PP-02  &  Menggunakan \textit{smartphone} dengan tujuan primer untuk membantu dalam pekerjaan & Aktivitas \\ \hline
  PP-03  &  Menggunakan \textit{smartphone} dengan tujuan sekunder untuk berinteraksi lewat media sosial & Aktivitas \\ \hline
  PP-04  &  Menggunakan \textit{smartphone} dengan tujuan sekunder sebagai sarana hiburan & Aktivitas \\ \hline
  PP-05  &  Menilai skala rata-rata 4 (empat) dari 5 (lima) terhadap durasi penggunaan \textit{smartphone} harian & Aktivitas \\ \hline
  PP-06  &  Merasa sering terdistraksi oleh keinginan diri sendiri untuk menggunakan \textit{smartphone}  & Sikap \\ \hline
  PP-07  &  Merasa sering terdistraksi oleh notifikasi dari \textit{smartphone} & Sikap \\ \hline
  PP-08  &  Merasa \textit{smartphone} sebaiknya membatasi pengguna seminimal mungkin & Sikap \\ \hline
  PP-09  &  Merasa perlu ada sebuah penghargaan jika berhasil mengikuti jadwal pembatasan \textit{smartphone} & Sikap \\ \hline
  PP-10  &  Mampu membatasi diri dari menggunakan \textit{smartphone} tanpa bantuan & Kemampuan \\ \hline
  PP-11  &  Ingin memblokir notifikasi aplikasi yang dinilai sebagai distraksi & Motivasi \\ \hline
  PP-12  &  Ingin mengurangi durasi penggunaan \textit{smartphone} harian & Motivasi \\ \hline
  PP-13  &  Ingin memantau kebiasaan penggunaan \textit{smartphone} & Motivasi \\ \hline
  PP-14  &  Ingin dibantu mengingatkan diri terhadap tugas / aktivitas yang harus dilakukan & Motivasi \\ \hline
  PP-15  &  Ingin mengingatkan diri terhadap jadwal tidur & Motivasi \\ \hline
  PP-16  &  Ingin diingatkan ketika terlalu lama menggunakan aplikasi & Motivasi \\ \hline
  PP-17  &  Ingin memblokir akses ke aplikasi yang dinilai mendistraksi tanpa menghapusnya & Motivasi \\ \hline
  PP-18  &  Mampu mengatur jadwal kegiatan dengan baik sehingga tidak mudah terdistraksi & Keterampilan \\ \hline
  PP-19  &  Terbiasa dalam mengoperasikan aplikasi pencegah distraksi pada \textit{smartphone}  & Keterampilan \\ \hline
\end{longtable}
\end{footnotesize}
\justifying
\FloatBarrier

\subsubsection{Masalah Pengguna}
\label{subsubsec:masalah_pengguna}

Ditemukan bahwa kategori ulasan yang didapat dari analisis ulasan pengguna cukup bervariasi dengan jumlah yang tersebar. Namun, ulasan pengguna tidak cukup dalam menggambarkan inti dari masalah yang mereka alami. Tahap verifikasi dari wawancara membantu menemukan inti masalah yang dikeluhkan serta gambaran utama dari keseluruhan masalah tersebut. Pengguna mengeluhkan bahwa mereka kurang mampu termotivasi dalam memperbaiki kebiasaan digitalnya oleh aplikasi Digital Wellbeing. Selain itu, pengguna juga kesulitan dalam menggunakan aplikasi Digital Wellbeing secara efisien untuk mencapai tujuan-tujuannya.

Untuk menyelesaikannya, masalah tersebut perlu dipecahkan menjadi masalah-masalah pengguna yang dapat dirincikan. Hal tersebut dilakukan untuk mempermudah penentuan kebutuhan dan tujuan pengguna serta penyusunan solusi. Adapun masalah-masalah tersebut dikaitkan dengan \textit{usability goals} atau \textit{user experience goals} yang berhubungan. Keseluruhan masalah pengguna dirangkum pada Tabel \ref{tab:daftar_masalah}.

\RaggedLeft
\begin{footnotesize}
\begin{longtable}[c]{|W{c}{0.07\textwidth}|>{\ccnormspacing}m{0.43\textwidth}|>{\ccnormspacingcenter}m{0.2\textwidth}|>{\ccnormspacingcenter}m{0.16\textwidth}|}
  \caption{Daftar Masalah Pengguna}
  \label{tab:daftar_masalah} \\
  \hline \rowcolor[HTML]{A3E5F5}
  \textbf{ID} & \centering\textbf{Masalah Pengguna} & \textbf{Kategori Ulasan} & \textbf{\textit{Usability Goals / UX Goals}} \\ \hline \endfirsthead
  \hline \rowcolor[HTML]{A3E5F5}
  \textbf{ID} & \centering\textbf{Masalah Pengguna} & \textbf{Kategori Ulasan} & \textbf{\textit{Usability Goals / UX Goals}} \\ \hline \endhead

  \hline \endfoot

  MP-01  & Pengguna kesulitan dalam melakukan pengaturan fitur-fitur aplikasi Digital Wellbeing secara efisien & KU-01, KU-04, KU-05, KU-09, KU-10, KU-11 & \textit{Usability Goal Efficiency} \\ \hline
  MP-02  & Pengguna kesulitan dalam menganalisis kebiasaan digital diri lewat aplikasi Digital Wellbeing & KU-02 & \textit{Usability Goal Learnability} \\ \hline
  MP-03  & Pengguna merasa fitur-fitur aplikasi Digital Wellbeing kurang ketat dalam membantu memperbaiki kebiasaan digital & KU-03, KU-04, KU-05, KU-06, KU-07 & \textit{Usability Goal Effectiveness} \\ \hline
  MP-04  & Pengguna merasa fitur-fitur aplikasi Digital Wellbeing kurang fleksibel dalam pengaturannya & KU-04, KU-06, KU-09, KU-10 & \textit{Usability Goal Efficiency}\\ \hline
  MP-05  & Pengguna merasa interaksi dengan aplikasi Digital Wellbeing kurang dapat memotivasi untuk memperbaiki kebiasaan digital & KU-08 & \textit{UX Goal Motivating} \\ \hline
  MP-06  & Pengguna kurang dapat memahami penggunaan fitur-fitur yang disediakan oleh aplikasi Digital Wellbeing & KU-08 & \textit{Usability Goal Learnability} \\ \hline
  MP-07  & Pengguna kesulitan dalam mengakses informasi pada fitur-fitur aplikasi Digital Wellbeing & KU-01 & \textit{Usability Goal Efficiency} \\ \hline
\end{longtable}
\end{footnotesize}
\justifying
\FloatBarrier

% $ =====================================================
% $   +  +  +  +  +  +  +  +  +  +  +  +  +  +  +  +  +
% $ =====================================================


\subsection{Persona}
\label{subsec:persona_pengguna}
Setelah dilakukan analisis mengenai perilaku dan masalah pengguna, dilakukan segmentasi pengguna menjadi beberapa kelompok persona. Persona merepresentasikan kelompok-kelompok pengguna dengan karakteristik dan perilaku yang berbeda. Persona berperan menjadi arah pengembangan interaksi aplikasi dan mengurangi kemungkinan mendesain aplikasi untuk semua orang sehingga menghasilkan desain yang tidak disenangi oleh siapapun \parencite{cooper2014face}.

\subsubsection{Pengelompokkan Awal Pengguna}
\label{subsubsec:pengelompokkan_pengguna}
Tahap awal dalam menyusun persona adalah mengelompokkan pengguna-pengguna berdasarkan perannya. Dari 10 orang partisipan riset wawancara, dikelompokkan menjadi 3 peran berdasarkan kemampuannya dalam membatasi diri dari \textit{smartphone} tanpa membutuhkan bantuan, dengan kata lain kemampuan penguasaan diri mereka. Kemampuan ini dianalisis dari data-data yang didapat melalui wawancara, di mana semakin tinggi tingkat kemampuannya berarti pengguna tersebut dapat membatasi dirinya dengan bantuan sesedikit mungkin. Pemilihan kemampuan ini sebagai dasar dari pengelompokkan awal pengguna berdasar pertimbangan-pertimbangan berikut

\begin{enumerate}
  \item Pengalaman dalam membutuhkan aplikasi pencegah distraksi. Semakin banyak aplikasi yang pernah dicoba maka kemampuan penguasaan diri dinilai semakin rendah.
  \item Penilaian diri terhadap durasi penggunaan \textit{smartphone} harian yang dinilai tidak produktif. Semakin tinggi penilaian maka kemampuan penguasaan diri dinilai semakin rendah.
  \item Keterampilan dalam menyusun jadwal harian. Semakin terbiasa dalam menyusun jadwal untuk mengatur kegiatan maka kemampuan penguasaan diri dinilai semakin tinggi.
  \item Jumlah fitur bantuan yang diinginkan dari sebuah aplikasi pencegah distraksi. Semakin banyak fitur bantuan yang diinginkan maka kemampuan penguasaan diri dinilai semakin rendah.
\end{enumerate}

Terdapat pengguna dengan kemampuan penguasaan diri tingkat rendah beranggotakan 20\% partisipan, pengguna berkemampuan tingkat sedang sebanyak 30\%, dan pengguna dengan kemampuan tingkat tinggi sebanyak 50\%. Maka dari itu, pembagian untuk kelompok pengguna adalah sebagai berikut

\begin{enumerate}
  \item Kelompok 1: Pengguna dengan kemampuan penguasaan diri rendah yang sangat memerlukan bantuan untuk membatasi diri dari \textit{smartphone} 
  \item Kelompok 2: Pengguna dengan kemampuan penguasaan diri sedang yang memerlukan bantuan ringan untuk membatasi diri dari \textit{smartphone}
  \item Kelompok 3: Pengguna dengan kemampuan penguasaan diri tinggi yang tidak memerlukan bantuan untuk membatasi diri dari \textit{smartphone}
\end{enumerate}

\subsubsection{Pemetaan Kelompok Pengguna dengan Variabel Perilaku}
Setelah dilakukan pengelompokkan pengguna, variabel-variabel perilaku yang telah diidentifikasi pada tahap riset dan analisis pengguna perlu dipetakan ke setiap kelompok pengguna. Hal ini bertujuan untuk mengidentifikasi perilaku dari setiap kelompok untuk kemudian disusun personanya.

Variabel perilaku P-10, tentang kemampuan membatasi diri dari menggunakan \textit{smartphone} tanpa bantuan, menunjukkan pemetaan sebagai dasar dari pembagian kelompok pengguna sebagaimana dijelaskan pada bagian \ref{subsubsec:pengelompokkan_pengguna}. Hasil pemetaan juga menunjukkan lebih lengkap ciri-ciri dari tiap kelompok pengguna dan perilaku yang mencerminkan kemampuannya dalam membatasi diri dari \textit{smartphone}. Hasil proses pemetaan dapat dilihat pada Tabel \ref{tab:pemetaan_perilaku}.

\RaggedLeft
\begin{footnotesize}
\begin{longtable}[c]{|>{\ccnormspacing}m{0.10\textwidth}|>{\ccnormspacing}m{0.31\textwidth}|>{\ccnormspacingcenter}m{0.145\textwidth}|>{\ccnormspacingcenter}m{0.145\textwidth}|>{\ccnormspacingcenter}m{0.145\textwidth}|}
  \caption{Daftar Pemetaan Kelompok Pengguna dengan Variabel Perilaku}
  \label{tab:pemetaan_perilaku} \\
  \hline \rowcolor[HTML]{A3E5F5}
  \centering\textbf{Variabel Perilaku} & \centering\textbf{Deskripsi Perilaku} & \textbf{Kelompok 1} & \textbf{Kelompok 2} & \textbf{Kelompok 3} \\ \hline \endfirsthead
  \hline \rowcolor[HTML]{A3E5F5}
  \centering\textbf{Variabel Perilaku} & \centering\textbf{Deskripsi Perilaku} & \textbf{Kelompok 1} & \textbf{Kelompok 2} & \textbf{Kelompok 3} \\ \hline \endhead

  \hline \endfoot

  \centering PP-10  & Kemampuan penguasaan diri terkait penggunaan  \textit{smartphone} & Rendah & Sedang & Tinggi \\ \hline
  \centering PP-01 PP-02  & Tujuan primer menggunakan \textit{smartphone} & Komunikasi & Pekerjaan, Komunikasi & Pekerjaan, Komunikasi \\ \hline
  \centering PP-03 PP-04 & Tujuan sekunder menggunakan \textit{smartphone} & Hiburan, Media sosial & Hiburan, Media sosial & Hiburan, Media sosial \\ \hline
  \centering PP-05 & Durasi penggunaan \textit{smartphone} harian & 6-10 jam per hari & 5-8 jam per hari & 4-7 jam per hari \\ \hline
  \centering PP-06 PP-07 & Sumber distraksi terkait \textit{smartphone} & Keinginan diri sendiri, \textit{smartphone}, pihak lain & Keinginan diri sendiri, \textit{smartphone}, pihak lain & Keinginan diri sendiri \\ \hline
  \centering PP-18 & Keterampilan dalam menyusun jadwal kegiatan & Sangat jarang menyusun jadwal & Selalu menyusun jadwal harian & Menyusun jadwal jika merasa penting \\ \hline
  \centering PP-19 & Jumlah aplikasi pencegah distraksi yang pernah dipakai & 2 aplikasi & 1-2 aplikasi & 1-2 aplikasi \\ \hline
  \centering PP-08 & Merasa \textit{smartphone} sebaiknya membatasi pengguna seminimal mungkin & Tidak Merasa & Merasa & Merasa \\ \hline
  \centering PP-09 & Merasa perlu ada sebuah penghargaan jika berhasil mengikuti jadwal pembatasan \textit{smartphone} & Merasa & Tidak Merasa & Tidak Merasa \\ \hline
  \centering PP-11 & Ingin memblokir notifikasi aplikasi yang dinilai sebagai distraksi & Ya & Ya & Ya \\ \hline
  \centering PP-12 & Ingin mengurangi durasi penggunaan \textit{smartphone} harian & Ya & Ya & Ya \\ \hline
  \centering PP-13 & Ingin memantau kebiasaan penggunaan \textit{smartphone} & Ya & Ya & Ya \\ \hline
  \centering PP-14 & Ingin dibantu mengingatkan diri terhadap tugas / aktivitas yang harus dilakukan & Ya & Ya & Tidak \\ \hline
  \centering PP-15 & Ingin mengingatkan diri terhadap jadwal tidur & Ya & Ya & Tidak \\ \hline
  \centering PP-16 & Ingin diingatkan ketika terlalu lama menggunakan aplikasi & Ya & Tidak & Tidak \\ \hline
  \centering PP-17 & Ingin memblokir akses ke aplikasi yang dinilai mendistraksi tanpa menghapusnya & Ya & Tidak & Tidak \\ \hline

\end{longtable}
\end{footnotesize}
\justifying
\FloatBarrier

\subsubsection{Pemetaan Kelompok Pengguna dengan Masalah Pengguna}
Selain variabel perilaku, masalah pengguna pun perlu dipetakan dengan kelompok pengguna yang telah dibuat. Hal ini dilakukan untuk mengenali masalah yang dirasakan oleh kelompok pengguna spesifik, dan juga membantu dalam menyusun persona.

Ditemukan bahwa ketiga kelompok pengguna mengalami sebagian besar dari masalah pengguna yang ada. Masalah pengguna MP-03 di mana aplikasi Digital Wellbeing dinilai kurang ketat tidak dirasakan oleh kelompok pengguna 2 dan 3 karena kemampuan yang cukup dalam membatasi diri dari \textit{smartphone} tanpa membutuhkan bantuan. Di sisi lain, masalah MP-04 tentang fleksibilitas aplikasi tidak dirasakan oleh kelompok pengguna 1 dengan tingkat penguasaan diri yang rendah dan kebutuhan mereka atas bantuan yang ketat untuk membatasi penggunaan \textit{smartphone}. Selain itu, akibat keterampilan kelompok pengguna 1 dalam menggunakan aplikasi pencegah distraksi, mereka tidak mengalami masalah MP-06 karena mereka sudah cukup mengerti fitur-fitur yang digunakan. Hasil dari proses pemetaan dapat dilihat pada Tabel \ref{tab:pemetaan_masalah}, di mana tanda centang menandakan kelompok pengguna merasakan masalah tersebut.

\RaggedLeft
\begin{footnotesize}
\begin{longtable}[c]{|>{\ccnormspacing}m{0.08\textwidth}|>{\ccnormspacing}m{0.31\textwidth}|>{\ccnormspacingcenter}m{0.145\textwidth}|>{\ccnormspacingcenter}m{0.145\textwidth}|>{\ccnormspacingcenter}m{0.145\textwidth}|}
  \caption{Daftar Pemetaan Kelompok Pengguna dengan Masalah Pengguna}
  \label{tab:pemetaan_masalah} \\
  \hline \rowcolor[HTML]{A3E5F5}
  \centering\textbf{ID} & \centering\textbf{Deskripsi Masalah} & \textbf{Kelompok 1} & \textbf{Kelompok 2} & \textbf{Kelompok 3} \\ \hline \endfirsthead
  \hline \rowcolor[HTML]{A3E5F5}
  \centering\textbf{ID} & \centering\textbf{Deskripsi Masalah} & \textbf{Kelompok 1} & \textbf{Kelompok 2} & \textbf{Kelompok 3} \\ \hline \endhead

  \hline \endfoot

  \centering MP-01 & Pengguna kesulitan dalam melakukan pengaturan fitur-fitur aplikasi Digital Wellbeing secara efisien & \textbf{V} & \textbf{V} & \textbf{V} \\ \hline
  \centering MP-02 & Pengguna kesulitan dalam menganalisis kebiasaan digital diri lewat aplikasi Digital Wellbeing & \textbf{V} & \textbf{V} & \textbf{V} \\ \hline
  \centering MP-03 & Pengguna merasa fitur-fitur aplikasi Digital Wellbeing kurang ketat dalam membantu memperbaiki kebiasaan digital & \textbf{V} &  &  \\ \hline
  \centering MP-04 & Pengguna merasa fungsionalitas fitur-fitur aplikasi Digital Wellbeing kurang fleksibel &  & \textbf{V} & \textbf{V} \\ \hline
  \centering MP-05 & Pengguna merasa interaksi dengan aplikasi Digital Wellbeing kurang dapat memotivasi untuk memperbaiki kebiasaan digital & \textbf{V} & \textbf{V} & \textbf{V} \\ \hline
  \centering MP-06 & Pengguna kurang dapat memahami penggunaan fitur-fitur yang disediakan oleh aplikasi Digital Wellbeing &  & \textbf{V} & \textbf{V} \\ \hline
  \centering MP-07 & Pengguna kesulitan dalam mengakses informasi pada fitur-fitur aplikasi Digital Wellbeing & \textbf{V} & \textbf{V} & \textbf{V} \\ \hline

\end{longtable}
\end{footnotesize}
\justifying
\FloatBarrier

\subsubsection{Penyusunan Karakteristik Persona}
Setelah dilakukan pemetaan kelompok pengguna terhadap variabel perilaku dan masalah, kelompok pengguna tersebut dapat disusun menjadi persona yang utuh dengan diberikan identitas dan karakter dalam bentuk sebuah narasi. Di dalam narasi tersebut, garis besar hasil pemetaan juga perlu disebutkan ulang. Hal-hal tersebut dapat membuat persona terasa hidup sehingga mempermudah proses analisis kebutuhan dan tujuan pengguna, serta pembuatan solusi desain.

Berikut adalah hasil penyusunan karakteristik persona untuk ketiga kelompok pengguna

\begin{enumerate}
  \item Kelompok 1 \textemdash \space Persona 1: Nico
  \subitem Nico adalah seorang mahasiswa berumur 21 tahun yang sedang menjalani semester akhir di universitasnya. Nico mengerjakan tugas akhir pada laptopnya, namun ia merasa mudah terdistraksi oleh \textit{smartphone}nya, baik dari keinginannya untuk memeriksa media sosial atau permainan, maupun notifikasi yang sering muncul. Oleh karena itu, Nico menggunakan aplikasi Digital Wellbeing untuk memblokir akses ke aplikasi yang ia rasa mendistraksi serta mematikan notifikasinya. Namun terkadang Nico membuka blokirnya terlalu sering sehingga dia merasa adanya fitur dari Digital Wellbeing tidak berpengaruh dalam membantunya mencegah distraksi dari \textit{smartphone}-nya. Nico merasa butuh bantuan yang cukup besar dalam mengendalikan dirinya terkait \textit{smartphone} karena ia sering menyadari bahwa dirinya terlalu sering menggunakan \textit{smartphone}, melupakan tugasnya, dan jadwal tidurnya pun terganggu.

  \item Kelompok 2 \textemdash \space Persona 2: Maya
  \subitem Maya adalah seorang pegawai swasta berumur 28 tahun. Dalam kesehariannya, Maya menggunakan \textit{smartphone} untuk berkomunikasi dengan rekan kerjanya, membantu dalam pekerjaannya, dan sebagai hiburan di jam istirahat. Maya sering terdistraksi di jam kerjanya oleh notifikasi atau keinginannya untuk memeriksa \textit{smartphone}, sehingga menggunakan aplikasi Digital Wellbeing untuk memblokirnya. Maya menemukan fitur-fitur menarik lain yang ia rasa dapat membantu mencegah distraksi, dan menganalisa serta memperbaiki kebiasaan penggunaan \textit{smartphone}nya. Namun Maya merasa bahwa beberapa fitur dari Digital Wellbeing kurang fleksibel untuk menyesuaikan dengan jadwal kerjanya yang bervariasi, seperti kurangnya kemampuan membuat jadwal lebih dari satu. Maya juga merasa pengalaman yang dirasakan kurang cukup personal untuk cukup memotivasinya karena pesan pengingat yang ia dapatkan terlalu membosankan.

  \item Kelompok 3 \textemdash \space Persona 3: Nathan
  \subitem Nathan adalah seorang lulusan baru dan pekerja yang berumur 23 tahun. Dia terbiasa dalam mengatur jadwal pekerjaannya, dan sesekali menyelipkan jadwal untuk beristirahat. Ia melakukan sebagian besar pekerjaannya menggunakan \textit{smartphone}, seperti berkomunikasi, mengatur jadwal, dan menulis catatan. Nathan sesekali merasa bahwa dirinya memeriksa media sosial di \textit{smartphone} di waktu yang tidak tepat di saat jadwal kerjanya. Maka Nathan menggunakan fitur Focus Mode dari Digital Wellbeing untuk mengingatkan dirinya jika akan bermain \textit{smartphone} di jam kerjanya dan memblokir notifikasi dari aplikasi-aplikasi yang dianggap mendistraksi. Namun, Nathan tidak menggunakan fitur lain dari Digital Wellbeing karena ia merasa tidak perlu bantuan cukup banyak, tampilan dari aplikasi tidak cukup menarik, dan fitur-fitur yang ada tidak memiliki deskripsi yang jelas atau pengaturan yang cukup mudah.

\end{enumerate}

\subsubsection{Pemilihan Tipe Persona}
Dari persona-persona yang telah disusun, perlu dipilih sebuah persona primer. Persona primer akan dijadikan target utama atau panduan dalam membuat desain solusi. Persona primer yang dipilih harus dipertimbangkan agar tidak mengecewakan persona-persona lainnya. Dalam hal ini, persona 2, Maya, akan dijadikan sebagai persona primer.

Pemilihan persona Maya sebagai persona utama melihat pertimbangan bahwa salah satu masalah terbesar dari aplikasi Digital Wellbeing adalah sulitnya pengguna dalam menggunakan aplikasi secara efisien untuk mencapai tujuan-tujuannya. Masalah tersebut tercerminkan juga dari masalah-masalah yang dikeluhkan oleh persona lainnya. Dalam hal itu, solusi yang didesain untuk persona Maya diharapkan dapat menyelesaikan masalah-masalah yang dialami persona lain.

% $ =====================================================
% $   +  +  +  +  +  +  +  +  +  +  +  +  +  +  +  +  +
% $ =====================================================


\subsection{Analisis Fungsionalitas Aplikasi}
\label{subsec:analisis_fungsionalitas}

Dalam mengidentifikasi konteks penggunaan produk, penting untuk mengerti lingkungan sistem dari produk tersebut untuk membantu mengidentifikasi kebutuhan pengguna. Untuk tugas akhir ini, perlu diidentifikasi fungsionalitas dari aplikasi Digital Wellbeing, yang dilakukan dengan mengobservasi aplikasinya langsung, serta fungsionalitas dari aplikasi pencegah distraksi pada umumnya. Dalam Tabel \ref{tab:daftar_fungsionalitas_app_dw} terdapat fungsionalitas dari aplikasi Digital Wellbeing yang sudah diterapkan, sedangkan dalam Tabel \ref{tab:daftar_fungsionalitas_app_umum} terdapat fungsionalitas dari aplikasi-aplikasi pencegah distraksi pada umumnya dengan mengacu pada penelitian yang telah dipelajari sebagai bagian dari studi literatur Bab \ref{sec:penelitian_terkait}.

\RaggedLeft
\begin{footnotesize}
\begin{longtable}[c]{|W{c}{0.07\textwidth}|>{\ccnormspacing}m{0.86\textwidth}|}
  \caption{Daftar Fungsionalitas Aplikasi Digital Wellbeing}
  \label{tab:daftar_fungsionalitas_app_dw} \\
  \hline \rowcolor[HTML]{A3E5F5} \textbf{ID} & \multicolumn{1}{|c|}{\textbf{Fungsionalitas Digital Wellbeing}} \\ \hline \endfirsthead
  \hline \rowcolor[HTML]{A3E5F5} \textbf{ID} & \multicolumn{1}{|c|}{\textbf{Fungsionalitas Digital Wellbeing}} \\ \hline \endhead
  
  \hline \endfoot
  
  FD-01  &  Aplikasi dapat melacak penggunaan harian \textit{smartphone} dari sisi durasi penggunaan, jumlah pembukaan, dan notifikasi yang diterima \\ \hline
  FD-02  &  Aplikasi dapat melacak penggunaan harian aplikasi-aplikasi, dari sisi durasi penggunaan, jumlah pembukaan, dan notifikasi yang dikirim \\ \hline
  FD-03  &  Aplikasi dapat menampilkan laporan penggunaan \textit{smartphone} atau aplikasi-aplikasi dalam bentuk grafik \\ \hline
  FD-04  &  Aplikasi dapat menampilkan laporan penggunaan \textit{smartphone} atau aplikasi-aplikasi dalam bentuk daftar \\ \hline
  FD-05  &  Pengguna dapat mengatur batas waktu per hari akses suatu aplikasi \\ \hline
  FD-06  &  Aplikasi dapat memberikan notifikasi terkait batas waktu akses suatu aplikasi \\ \hline
  FD-07  &  Aplikasi dapat memunculkan jendela pengingat batas waktu akses aplikasi \\ \hline
  FD-08  &  Aplikasi dapat memblokir akses terhadap suatu aplikasi \\ \hline
  FD-09  &  Aplikasi dapat memblokir notifikasi yang dikirim suatu aplikasi \\ \hline
  FD-10  &  Pengguna dapat mengatur jadwal aktivasi pemblokiran akses aplikasi \\ \hline
  FD-11  &  Pengguna dapat menunda pemblokiran akses sebuah aplikasi secara sementara \\ \hline
  FD-12  &  Pengguna dapat mengatur jadwal aktivasi mode waktu tidur \textit{smartphone} \\ \hline
  FD-13  &  Pengguna dapat menunda sementara aktivasi mode waktu tidur \textit{smartphone} \\ \hline
  FD-14  &  Pengguna dapat membuka pengaturan notifikasi aplikasi \\ \hline
  FD-15  &  Aplikasi dapat mengubah warna layar menjadi berskala abu-abu \\ \hline
  FD-16  &  Aplikasi dapat mendeskripsikan sebuah fitur baik melalui penjelasan atau ilustrasi \\ \hline
\end{longtable}
\end{footnotesize}
\justifying
\FloatBarrier

\RaggedLeft
\begin{footnotesize}
\begin{longtable}[c]{|W{c}{0.07\textwidth}|>{\ccnormspacing}m{0.86\textwidth}|}
  \caption{Daftar Fungsionalitas Aplikasi Pencegah Distraksi Pada Umumnya}
  \label{tab:daftar_fungsionalitas_app_umum} \\
  \hline \rowcolor[HTML]{A3E5F5} \textbf{ID} & \multicolumn{1}{|c|}{\textbf{Fungsionalitas Umum}} \\ \hline \endfirsthead
  \hline \rowcolor[HTML]{A3E5F5} \textbf{ID} & \multicolumn{1}{|c|}{\textbf{Fungsionalitas Umum}} \\ \hline \endhead
  
  \hline \endfoot
  
  FU-01  &  Aplikasi dapat melacak penggunaan harian \textit{smartphone}, dari sisi durasi penggunaan dan jumlah pembukaan \\ \hline
  FU-02  &  Aplikasi dapat melacak penggunaan harian aplikasi-aplikasi, dari sisi durasi penggunaan dan jumlah pembukaan \\ \hline
  FU-03  &  Aplikasi dapat mempresentasikan data penggunaan \textit{smartphone} harian berupa ringkasan pemakaian \textit{smartphone} dan aplikasi-aplikasi \\ \hline
  FU-04  &  Aplikasi dapat menampilkan data penggunaan \textit{smartphone} dalam bentuk grafik \\ \hline
  FU-05  &  Aplikasi dapat menampilkan data penggunaan \textit{smartphone} dalam bentuk daftar \\ \hline
  FU-06  &  Aplikasi dapat menampilkan data penggunaan \textit{smartphone} dalam sebuah \textit{widget} \\ \hline
  FU-07  &  Aplikasi dapat menampilkan data penggunaan \textit{smartphone} dalam sebuah notifikasi \\ \hline
  FU-08  &  Pengguna dapat mengatur batas waktu per hari akses \textit{smartphone} \\ \hline
  FU-09  &  Pengguna dapat mengatur pemblokiran terhadap akses \textit{smartphone} untuk durasi waktu yang ditentukan \\ \hline
  FU-10  &  Pengguna dapat mengheningkan suara dari \textit{smartphone} dan menguncinya untuk waktu yang ditentukan \\ \hline
  FU-11  &  Pengguna dapat mengatur batas waktu per hari akses suatu aplikasi \\ \hline
  FU-12  &  Pengguna dapat mengatur pemblokiran terhadap suatu aplikasi untuk durasi waktu yang ditentukan \\ \hline
  FU-13  &  Pengguna dapat memasang pemblokiran terhadap \textit{smartphone} atau suatu aplikasi jika memenuhi konteks penggunaan yang diatur pengguna, seperti saat melakukan aktivitas, atau di lokasi tertentu \\ \hline
  FU-14  &  Pengguna dapat menunda pemblokiran akses \textit{smartphone} secara sementara \\ \hline
  FU-15  &  Pengguna dapat menghapus pemblokiran akses \textit{smartphone}\\ \hline
  FU-16  &  Pengguna dapat menunda pemblokiran akses sebuah aplikasi secara sementara \\ \hline
  FU-17  &  Pengguna dapat menghapus pemblokiran akses sebuah aplikasi \\ \hline
  FU-18  &  Aplikasi dapat memunculkan jendela pengingat batas waktu penggunaan \textit{smartphone} atau aplikasi \\ \hline
  
  
\end{longtable}
\end{footnotesize}
\justifying
\FloatBarrier

\subsection{Analisis Kebutuhan dan Tujuan Pengguna}
\label{subsec:analisis_kebutuhan_tujuan}

Bagian dari identifikasi konteks penggunaan ini akan menganalisis tentang kebutuhan serta tujuan dari pengguna mengenai aplikasi Digital Wellbeing dan aplikasi pencegah distraksi secara umum, dibantu dengan data dari perilaku dan persona pengguna, masalah pengguna, dan fungsionalitas aplikasi. Keduanya akan berguna dalam merancang perbaikan desain interaksi yang diperlukan dari aplikasi Digital Wellbeing, terutama untuk menyusun \textit{usability} dan \textit{user experience goals}.

\subsubsection{Kebutuhan Pengguna}
\label{subsubsec:kebutuhan_pengguna}

Kebutuhan pengguna adalah hal apa saja yang diperlukan pengguna untuk mencapai tujuannya dalam memakai aplikasi Digital Wellbeing. Analisis kebutuhan pengguna melibatkan masalah pengguna serta perilaku pengguna yang telah dibahas sebelumnya. Penjelasan kebutuhan pengguna dirangkum di dalam Tabel \ref{tab:daftar_kebutuhan}.

% \newpage

\RaggedLeft
\begin{footnotesize}
\begin{longtable}[c]{|W{c}{0.07\textwidth}|>{\ccnormspacing}m{0.55\textwidth}|>{\ccnormspacingcenter}m{0.125\textwidth}|>{\ccnormspacingcenter}m{0.125\textwidth}|}
  \caption{Daftar Kebutuhan Pengguna}
  \label{tab:daftar_kebutuhan} \\
  \hline \rowcolor[HTML]{A3E5F5}
  \textbf{ID} & \centering\textbf{Kebutuhan Pengguna} & \textbf{Masalah Pengguna} & \textbf{Perilaku Pengguna} \\ \hline \endfirsthead
  \hline \rowcolor[HTML]{A3E5F5}
  \textbf{ID} & \centering\textbf{Kebutuhan Pengguna} & \textbf{Masalah Pengguna} & \textbf{Perilaku Pengguna} \\ \hline \endhead

  \hline \endfoot

  UN-01  & Pengalaman pengaturan fitur yang lebih efisien dengan kemampuan seperti pencarian dan pengelompokan aplikasi & MP-01 & PP-18, PP-19 \\ \hline
  UN-02  & \textit{Widget} untuk mengakses data serta melakukan pengaturan terhadap fitur-fitur lewat Homescreen & MP-01, MP-02, MP-07 & PP-13, PP-19 \\ \hline
  UN-03  & Fitur rekomendasi untuk memberikan informasi tentang kebiasaan digital yang baik dan aksi yang dapat dilakukan & MP-02, MP-05 & PP-12, PP-13 \\ \hline
  UN-04  & Laporan penggunaan \textit{smartphone} dengan rentang waktu lebih banyak dan ringkasan informasi seperti rata-rata penggunaan & MP-02 & PP-13 \\ \hline
  UN-05  & Fitur penjadwalan dengan kemampuan menambah lebih dari satu jadwal aktivasi fitur & MP-04 & PP-18  \\ \hline
  UN-06  & Kemampuan penundaan pada restriksi yang diterapkan oleh fitur-fitur & MP-04 & PP-08 \\ \hline
  UN-07  & Kemampuan pengaturan pesan untuk melakukan personalisasi pesan-pesan pengingat dari fitur-fitur & MP-05 & PP-14 \\ \hline
  UN-08  & Tampilan aplikasi dan fitur yang lebih mudah dipelajari & MP-06 & PP-19 \\ \hline
\end{longtable}
\end{footnotesize}
\justifying
\FloatBarrier

\subsubsection{Tujuan dan Kegiatan Pengguna}
\label{subsubsec:tujuan_kegiatan_pengguna}

Dengan ditentukannya kebutuhan pengguna, maka dapat dianalisis tujuan yang ingin dicapai oleh pengguna. Dalam mencapai tujuan-tujuan tersebut, maka perlu ditentukan kegiatan yang harus dilakukan oleh pengguna, atau disebut sebagai \textit{user task}. Analisis tujuan dan kegiatan pengguna dilakukan dengan mengkaitkan kebutuhan pengguna dan perilaku pengguna. Hasil analisis dirangkum pada Tabel \ref{tab:daftar_tujuan_kegiatan}.

% \newpage

\newlength{\cccolid}
\setlength{\cccolid}{0.08\textwidth}

\newlength{\cccolgoal}
\setlength{\cccolgoal}{0.2\textwidth}

\newlength{\cccolneed}
\setlength{\cccolneed}{0.13\textwidth}

\newcommand{\ccid}[2]{\multirow{#1}{\cccolid}{\centering\linespread{1}\selectfont #2}}
\newcommand{\ccgoal}[2]{\multirow{#1}{\cccolgoal}{\linespread{1}\selectfont #2}}
\newcommand{\ccneed}[2]{\multirow{#1}{\cccolneed}{\centering\linespread{1}\selectfont #2}}
\newcommand{\ccline}{\hhline{|~|~|-|-|~|}}


\RaggedLeft
\begin{footnotesize}
\begin{longtable}[c]{|>{\ccnormspacing}m{\cccolid}|>{\ccnormspacing}m{\cccolgoal}|>{\ccnormspacing}m{0.11\textwidth}|>{\ccnormspacing}m{0.325\textwidth}|>{\ccnormspacingcenter}m{\cccolneed}|}
  \caption{Daftar Tujuan dan Kegiatan Pengguna}
  \label{tab:daftar_tujuan_kegiatan} \\
  \hline \rowcolor[HTML]{A3E5F5}
  \centering\textbf{ID Tujuan} & \centering\textbf{Tujuan Pengguna} & \centering\textbf{ID Kegiatan} & \centering\textbf{Kegiatan Pengguna} & \textbf{Perilaku \& Kebutuhan Pengguna} \\ \hline \endfirsthead
  \hline \rowcolor[HTML]{A3E5F5}
  \centering\textbf{ID Tujuan} & \centering\textbf{Tujuan Pengguna} & \centering\textbf{ID Kegiatan} & \centering\textbf{Kegiatan Pengguna} & \textbf{Perilaku \& Kebutuhan Pengguna} \\ \hline \endhead

  \hline \endfoot

   & & \centering{UT-01} & Membuat jadwal aktivasi fitur pemblokiran akses aplikasi & \\ \ccline
   & & \centering{UT-02} & Menyetel pengaturan notifikasi & \\ \ccline
   \ccid{-4}{UG-01} & \ccgoal{-4}{Mencegah distraksi dari \textit{smartphone} di waktu tertentu} & \centering{UT-03} & Memilih aplikasi yang diblokir aksesnya & \ccneed{-4}{PP-07, PP-11, PP-17, UN-01, UN-05}\\ \hline
  
   & & \centering{UT-04} & Memasang batas waktu penggunaan harian aplikasi & \\ \ccline
   & & \centering{UT-05} & Melihat sisa waktu penggunaan aplikasi lewat \textit{widget} & \\ \ccline
   \ccid{-4.6}{UG-02}& \ccgoal{-4.6}{Membatasi waktu penggunaan \textit{smartphone} harian} & \centering{UT-06} & Melihat total waktu penggunaan \textit{smartphone} lewat \textit{widget} & \ccneed{-4.6}{PP-12, PP-16, PP-18, UN-01, UN-02}\\ \hline
  
   & & \centering{UT-07} & Mengakses laporan penggunaan \textit{smartphone} & \\ \ccline
   & & \centering{UT-08} & Mengelompokkan aplikasi ke dalam sebuah kategori & \\ \ccline
   & & \centering{UT-09} & Memilih rentang waktu laporan penggunaan \textit{smartphone} & \\ \ccline
   \ccid{-6.5}{UG-03}& \ccgoal{-6.5}{Menganalisis kebiasaan penggunaan \textit{smartphone}} & \centering{UT-10} & Melihat rekomendasi kebiasaan penggunaan \textit{smartphone} yang baik & \ccneed{-6.5}{PP-13, UN-01, UN-03, UN-04}\\ \hline
  
   & & \centering{UT-11} & Memasang pesan pengingat aktivitas atau target harian & \\ \ccline
   \ccid{-3.2}{UG-04}& \ccgoal{-3.2}{Mengingatkan diri terhadap aktivitas utama yang seharusnya dilakukan} & \centering{UT-12} & Mengatur frekuensi notifikasi dari fitur-fitur yang mengirimkan notifikasi  & \ccneed{-3.2}{PP-14, UN-07}\\ \hline

   & & \centering{UT-13} & Mengatur jadwal aktivasi fitur Bedtime Mode & \\ \ccline
   \ccid{-2.7}{UG-05}& \ccgoal{-2.7}{Membantu mengatur kebiasaan tidur yang sehat} & \centering{UT-14} & Membatasi aplikasi yang dapat diakses di jadwal tidur & \ccneed{-2.7}{PP-15}\\ \hline
  
   & & \centering{UT-15} & Mengambil waktu istirahat dari Focus Mode & \\ \ccline
   & & \centering{UT-16} & Menunda aktivasi Bedtime Mode & \\ \ccline
   \ccid{-4.4}{UG-06}& \ccgoal{-4.4}{Mengambil istirahat sejenak dari restriksi aplikasi} & \centering{UT-17} & Memperpanjang waktu penggunaan aplikasi yang diblokir oleh App Timer & \ccneed{-4.4}{UN-06}\\ \hline

\end{longtable}
\end{footnotesize}
\justifying
\FloatBarrier

\subsubsection{Skenario Pengguna}
\label{subsubsec:skenario_pengguna}
Berdasarkan persona yang dipilih serta analisis terhadap tujuan dan kegiatan pengguna, dapat dibuat skenario pengguna untuk persona Maya. Skenario pengguna mengembangkan persona lebih lanjut dengan memberikan gambaran tentang bagaimana persona tersebut menggunakan aplikasi untuk mencapai tujuan-tujuannya. Penulisan skenario dalam bentuk narasi adalah alat bantu desain yang efektif untuk mengkomunikasikan ide dari desain tersebut \parencite{cooper2014face}. Skenario pengguna juga akan berguna dalam menentukan skenario pengujian dari prototipe aplikasi. Skenario-skenario pengguna dari persona Maya yang mendeskripsikan contoh kasus penggunaan solusi desain untuk aplikasi Digital Wellbeing terdapat pada Tabel \ref{tab:daftar_skenario}.

% Berdasarkan persona yang dipilih serta analisis terhadap tujuan dan kegiatan pengguna, dapat dibuat skenario pengguna untuk persona Maya. Skenario pengguna mengembangkan persona lebih lanjut dengan memberikan gambaran tentang bagaimana persona tersebut menggunakan aplikasi untuk mencapai tujuan-tujuannya. Penulisan skenario dalam bentuk narasi adalah alat bantu desain yang efektif untuk mengkomunikasikan ide dari desain tersebut \parencite{cooper2014face}. Skenario pengguna juga akan berguna dalam menentukan %$ \textit{user flow}  dari prototipe aplikasi. Skenario-skenario pengguna dari persona Maya yang mendeskripsikan contoh kasus penggunaan solusi desain untuk aplikasi Digital Wellbeing terdapat pada Tabel \ref{tab:daftar_skenario}.

\RaggedLeft
\begin{footnotesize}
\begin{longtable}[c]{|W{c}{0.07\textwidth}|>{\ccnormspacing}m{0.6\textwidth}|>{\ccnormspacingcenter}m{0.24\textwidth}|}
  \caption{Daftar Skenario Pengguna}
  \label{tab:daftar_skenario} \\
  \hline \rowcolor[HTML]{A3E5F5}
  \textbf{ID} & \centering\textbf{Skenario Pengguna} & \textbf{Kegiatan Pengguna} \\ \hline \endfirsthead
  \hline \rowcolor[HTML]{A3E5F5}
  \textbf{ID} & \centering\textbf{Skenario Pengguna} & \textbf{Kegiatan Pengguna} \\ \hline \endhead

  \hline \endfoot
  
  SP-01 & Maya sedang bekerja menggunakan laptopnya, saat ia sedang merasa sedikit bosan dan tiba-tiba mendapat perasaan untuk memeriksa media sosial lewat \textit{smartphone}-nya. Tanpa sadar, Maya sudah bermain media sosial selama setengah jam ketika rekannya bertanya tentang pekerjaannya. Bagaimana Maya dapat memanfaatkan aplikasi Digital Wellbeing untuk mengingatkan pekerjaannya saat menggunakan \textit{smartphone} di jam kerja? & UT-05, UT-11, UT-12 \\ \hline
  
  SP-02 & Maya ingin melihat seberapa sering ia menggunakan aplikasi-aplikasi favoritnya, dan menilai apakah kebiasaan digitalnya masih baik atau perlu diperbaiki. Bagaimana Maya dapat memanfaatkan aplikasi Digital Wellbeing untuk menganalisis penggunaan \textit{smartphone}-nya dengan lengkap serta mendapatkan sugesti tentang penggunaan \textit{smartphone} yang wajar? & UT-07, UT-08, UT-09, UT-10 \\ \hline
  
  SP-03 & Setelah menganalisis kebiasaannya, Maya menemukan bahwa ia menggunakan aplikasi media sosial seperti Instagram, Twitter, dan \textit{game} Minesweeper terlalu sering setiap harinya. Maya ingin membatasi waktu penggunaan media sosial dan aplikasi-aplikasi yang dinilai mendistraksinya dari kegiatan sehari-hari. Bagaimana Maya dapat memanfaatkan aplikasi Digital Wellbeing untuk mengurangi waktu penggunaan aplikasi-aplikasi? & UT-03, UT-04, UT-08 \\ \hline

  SP-04  & Maya sedang memulai hari kerjanya, dan ia ingat untuk mengatur jadwal sesi fokusnya. Bagaimana cara Maya bisa mengatur jadwalnya agar bisa sesuai dengan jam kerjanya yang bervariasi? & UT-01, UT-02, UT-03 \\ \hline
  
  SP-05  & Mendekati akhir hari kerjanya, Maya menyelesaikan tugasnya lebih cepat dari biasanya dan ingin bermain dengan \textit{smartphone}-nya. Namun ia telah memasang restriksi sebelumnya yang akan diangkat setelah jam kerjanya selesai. Bagaimana Maya dapat mengatur aplikasi Digital Wellbeing agar dapat memberikan leluasa tanpa menghapus pengaturan yang telah dipasang? & UT-15, UT-17 \\ \hline
  
  SP-06  & Maya ingin beristirahat untuk mengakhiri harinya. Namun Maya sering lupa waktu ketika sedang bermain dengan \textit{smartphone}-nya sehingga tak jarang ia melewati jam tidurnya selama satu jam. Maya ingin memasang pengingat pada \textit{smartphone}-nya untuk menepati jam tidurnya agar setiap pagi ia bisa bangun dengan tepat waktu. Bagaimana Maya dapat memanfaatkan aplikasi Digital Wellbeing untuk mencapai tujuannya? & UT-13, UT-14, UT-16 \\ \hline
\end{longtable}
\end{footnotesize}
\justifying
\FloatBarrier
