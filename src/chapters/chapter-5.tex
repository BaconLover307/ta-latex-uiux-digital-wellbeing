\chapter{Kesimpulan dan Saran}

Bab Kesimpulan dan Saran menjelaskan tentang bagian akhir dari penelitian dan merupakan penutup dari laporan tugas akhir ini. Bab ini membahas tentang kesimpulan yang berisi ketercapaian tujuan penelitian terkait dengan rumusan masalah yang diselesaikan pada tugas akhir. Bab ini juga membahas tentang saran mengenai hal-hal yang dapat dilakukan untuk pengembangan selanjutnya.

\section{Kesimpulan}
Penelitian tugas akhir ini menghasilkan sebuah prototipe \textit{high-fidelity} aplikasi pencegah distraksi. Berdasarkan hasil analisis yang didapatkan, berikut adalah kesimpulan yang diambil

\begin{enumerate}
  \item Desain interaksi yang baik untuk sebuah aplikasi pencegah distraksi memprioritaskan \textit{usability goals} efisien untuk digunakan (\textit{efficiency}) dan mudah untuk dipelajari (\textit{learnability}), serta memiliki \textit{user experience goals} \textit{helpful} dan \textit{motivating}. Prototipe \textit{high-fidelity} aplikasi dinilai sudah baik dalam mencapai \textit{goals} tersebut, dilihat dari hasil analisis masukan pengguna pada \textit{usability testing}.
    \begin{enumerate}[label=\alph*.]
      \item Pengguna merasa bahwa aplikasi sudah cukup efisien untuk digunakan, melihat penilaian untuk metrik pengukuran \textit{System Usability Scale} (SUS) memiliki skor rata-rata 91,5 dari nilai maksimal 100, yang termasuk ke dalam kategori A atau di atas \textit{Excellent}. 
      
      \item Pengguna merasa bahwa aplikasi mudah untuk dipelajari, melihat penilaian untuk metrik pengukuran \textit{Single Ease Question} (SEQ) memiliki skor rata-rata 6,86 dari nilai maksimal 7.
      
      \item Pengguna merasa terbantu dalam menggunakan aplikasi, melihat penilaian untuk metrik pengukuran \textit{Intrinsic Motivation Inventory} (IMI) subskala \textit{Value/Usefulness} memiliki skor rata-rata 6,86 dari nilai maksimal 7.
      
      \item Pengguna merasa termotivasi dari menggunakan aplikasi, melihat penilaian untuk metrik pengukuran \textit{Intrinsic Motivation Inventory} (IMI) subskala \textit{Interest/Enjoyment} memiliki skor rata-rata 6,2 dari nilai maksimal 7, serta untuk subskala \textit{Pressure/Tension} memiliki skor rata-rata 1,36.
        
    \end{enumerate}
    
  \item kesimpulan
   
\end{enumerate}

\section{Saran}
Pada implementasi prototipe \textit{high-fidelity} aplikasi pencegah distraksi ini, masih banyak hal yang dapat ditingkatkan dan dikembangkan lebih lanjut. Maka dari itu, berikut adalah beberapa saran yang dapat dilakukan dalam pengembangan selanjutnya.

\begin{enumerate}
  \item Proses pengumpulan data untuk mengidentifikasi masalah pada tugas akhir ini masih menggunakan data dari ulasan aplikasi dari Google Play Store. Maka dari itu, pengumpulan data dapat ditingkatkan dengan melakukan penyebaran form kepada orang-orang yang termasuk ke dalam lingkup pengguna.
  \item Skenario pengujian dapat dilengkapi dengan menyertakan pengujian terhadap komponen-komponen yang hanya dapat diinteraksi dari prototipe aplikasi yang dapat berjalan, seperti notifikasi aplikasi serta pemblokiran penggunaan aplikasi yang dinilai mendistraksi.
\end{enumerate}
