\chapter{Kesimpulan dan Saran}

Bab Kesimpulan dan Saran menjelaskan tentang bagian akhir dari penelitian dan merupakan penutup dari laporan tugas akhir ini. Bab ini membahas tentang kesimpulan yang berisi ketercapaian tujuan penelitian terkait dengan rumusan masalah yang diselesaikan pada tugas akhir. Bab ini juga membahas tentang saran mengenai hal-hal yang dapat dilakukan untuk pengembangan selanjutnya.

\section{Kesimpulan}
Penelitian tugas akhir ini menghasilkan sebuah prototipe \textit{high-fidelity} aplikasi pencegah distraksi. Berdasarkan hasil analisis yang didapatkan, berikut adalah kesimpulan yang diambil

\begin{enumerate}
  \item Desain interaksi yang baik untuk sebuah aplikasi pencegah distraksi memprioritaskan \textit{usability goals} efisien untuk digunakan (\textit{efficiency}) dan mudah untuk dipelajari (\textit{learnability}), serta memiliki \textit{user experience goals} \textit{helpful} dan \textit{motivating}. Prototipe \textit{high-fidelity} aplikasi dinilai sudah baik dalam mencapai \textit{goals} tersebut, dilihat dari hasil analisis masukan pengguna pada \textit{usability testing}.
    \begin{enumerate}[label=\alph*.]
      \item Pengguna merasa bahwa aplikasi sudah cukup efisien untuk digunakan, melihat penilaian untuk metrik pengukuran \textit{System Usability Scale} (SUS) memiliki skor rata-rata 91,5 dari skala 100, yang termasuk ke dalam kategori A atau di atas \textit{Excellent}. 
      
      \item Pengguna merasa bahwa aplikasi mudah untuk dipelajari, melihat penilaian untuk metrik pengukuran \textit{Single Ease Question} (SEQ) memiliki skor rata-rata 6,88 dari skala 7.
      
      \item Pengguna merasa terbantu dalam menggunakan aplikasi, melihat penilaian untuk metrik pengukuran \textit{Intrinsic Motivation Inventory} (IMI) subskala \textit{Value/Usefulness} memiliki skor rata-rata 6,89 dari skala 7.
      
      \item Pengguna merasa termotivasi dari menggunakan aplikasi, melihat penilaian untuk metrik pengukuran \textit{Intrinsic Motivation Inventory} (IMI) subskala \textit{Interest/Enjoyment} memiliki skor rata-rata 6,2 dari skala 7, serta untuk subskala \textit{Pressure/Tension} memiliki skor rata-rata 1,36.
        
    \end{enumerate}
    
  \item Rancangan desain interaksi aplikasi pencegah distraksi yang tepat untuk menyelesaikan masalah-masalah dari aplikasi Digital Wellbeing memiliki tipe interaksi \textit{Instructing} dan \textit{Responding}, dengan fitur unggulan yaitu fitur Search bar, App Group, dan Daftar Jadwal Aktivasi untuk membantu meningkatkan efisiensi pengguna dalam mencapai tujuannya, serta fitur Daily Goal untuk meningkatkan motivasi pengguna dalam mencegah distraksi. Fitur Rekomendasi Aksi dan Smartphone Usage Evaluation ditambahkan sebagai pelengkap fitur yang telah disebutkan. Adapun fitur Date range selector, App Timer, Focus Mode, Take a break, Turn off for now, dan Pengaturan notifikasi perlu dimodifikasi untuk memenuhi kebutuhan dan tujuan pengguna. Perancangan prototipe menerapkan prinsip desain dari \textit{Digital Wellbeing} menurut Google serta prinsip desain interaksi menurut Preece.  
   
\end{enumerate}

\section{Saran}
Pada implementasi prototipe \textit{high-fidelity} aplikasi pencegah distraksi ini, masih banyak hal yang dapat ditingkatkan dan dikembangkan lebih lanjut. Maka dari itu, berikut adalah beberapa saran yang dapat dilakukan dalam pengembangan selanjutnya.

\begin{enumerate}
  % \item Proses pengumpulan data untuk mengidentifikasi masalah pada tugas akhir ini masih menggunakan data dari ulasan aplikasi dari Google Play Store. Maka dari itu, pengumpulan data dapat ditingkatkan dengan melakukan penyebaran form kepada orang-orang yang termasuk ke dalam lingkup pengguna.
  \item Pengumpulan data dapat ditingkatkan dengan melakukan penyebaran form kepada orang-orang yang termasuk ke dalam lingkup pengguna.
  \item Skenario pengujian dapat dilengkapi dengan menyertakan komponen-komponen yang hanya dapat diinteraksi dari prototipe aplikasi yang dapat berjalan, seperti notifikasi aplikasi serta pemblokiran penggunaan aplikasi yang dinilai mendistraksi.
  \item Menambahkan \textit{dark mode} sebagai tema desain selain yang diterapkan pada prototipe yaitu \textit{light mode}.
  \item Membuat tampilan prototipe dalam bahasa Indonesia.
\end{enumerate}
