\chapter{Kesimpulan dan Saran}

Bab Kesimpulan dan Saran menjelaskan tentang bagian akhir dari penelitian dan merupakan penutup dari laporan tugas akhir ini. Bab ini membahas tentang kesimpulan yang berisi ketercapaian tujuan penelitian terkait dengan rumusan masalah yang diselesaikan pada tugas akhir. Bab ini juga membahas tentang saran mengenai hal-hal yang dapat dilakukan untuk pengembangan selanjutnya.

\section{Kesimpulan}
Penelitian tugas akhir ini menghasilkan sebuah prototipe \textit{high-fidelity} aplikasi Google Digital Wellbeing. Berdasarkan hasil analisis yang didapatkan, berikut adalah kesimpulan yang diambil

\begin{enumerate}
  \item Desain interaksi yang baik untuk aplikasi Google Digital Wellbeing memprioritaskan \textit{usability goals} memiliki utilitas yang baik (\textit{utility}) dan mudah untuk dipelajari (\textit{learnability}), serta memiliki \textit{user experience goals} \textit{helpful} dan \textit{motivating}. Prototipe \textit{high-fidelity} aplikasi dinilai sudah baik dalam mencapai \textit{goals} tersebut, dilihat dari hasil analisis masukan pengguna pada \textit{usability testing}.
    \begin{enumerate}[label=\alph*.]
      \item Pengguna merasa bahwa prototipe aplikasi sudah memiliki \textit{overall usability} yang cukup baik, melihat 100\% pengguna memberikan penilaian untuk metrik pengukuran \textit{System Usability Scale} (SUS) di atas ambang rata-rata 68, dengan skor terendah 85. 
      
      \item Pengguna merasa bahwa prototipe aplikasi sudah memiliki utilitas yang baik, melihat jawaban dari metrik pengukuran \textit{System Usability Scale} (SUS) 100\% pengguna setuju fungsi-fungsi di dalam prototipe terintegrasi dengan baik, tidak setuju bahwa terdapat banyak hal yang tidak konsisten di dalam prototipe, serta sangat tidak setuju bahwa prototipe tidak praktis untuk dipakai. 
      
      \item Pengguna merasa bahwa prototipe aplikasi mudah untuk dipelajari, melihat penilaian untuk metrik pengukuran \textit{Single Ease Question} (SEQ) memiliki skor rata-rata 6,88 dari skala 7.
      
      \item Pengguna merasa terbantu dalam menggunakan prototipe aplikasi, melihat 100\% pengguna memberikan penilaian untuk metrik pengukuran \textit{Intrinsic Motivation Inventory} (IMI) subskala \textit{Value/Usefulness} di atas nilai ambang batas 6, dengan skor terendah sebesar 6,43 dari skala 7.
      
      \item Pengguna merasa cukup termotivasi dari menggunakan prototipe aplikasi, melihat 60\% pengguna memberikan penilaian untuk metrik pengukuran \textit{Intrinsic Motivation Inventory} (IMI) subskala \textit{Interest/Enjoyment} di atas ambang batas 6, dengan skor terendah 5,14 dari skala 7, serta 80\% pengguna memberikan penilaian muntuk subskala \textit{Pressure/Tension} di bawah ambang batas 2, dengan skor tertinggi 2,20 dari 7.
        
    \end{enumerate}
    
  \item Rancangan desain interaksi aplikasi Digital Wellbeing yang tepat untuk menyelesaikan masalah-masalah dari aplikasi Google Digital Wellbeing memiliki tipe interaksi \textit{Instructing} dan \textit{Responding}, dengan fitur unggulan yaitu fitur Search bar, App Group, dan Daftar Jadwal Aktivasi untuk membantu meningkatkan utilitas dari aplikasi, serta fitur Daily Goal untuk meningkatkan motivasi pengguna dalam mencegah distraksi. Fitur Rekomendasi Aksi dan Smartphone Usage Evaluation ditambahkan sebagai pelengkap fitur yang telah disebutkan. Adapun fitur Date range selector, App Timer, Focus Mode, Take a break, Turn off for now, dan Pengaturan notifikasi perlu dimodifikasi untuk memenuhi kebutuhan dan tujuan pengguna. Perancangan prototipe menerapkan prinsip desain dari \textit{Digital Wellbeing} menurut Google serta prinsip desain interaksi menurut Preece.  
   
\end{enumerate}

\section{Saran}
Pada implementasi prototipe \textit{high-fidelity} aplikasi Google Digital Wellbeing ini, masih banyak hal yang dapat ditingkatkan dan dikembangkan lebih lanjut. Maka dari itu, berikut adalah beberapa saran yang dapat dilakukan dalam pengembangan selanjutnya.

\begin{enumerate}
  % \item Proses pengumpulan data untuk mengidentifikasi masalah pada tugas akhir ini masih menggunakan data dari ulasan aplikasi dari Google Play Store. Maka dari itu, pengumpulan data dapat ditingkatkan dengan melakukan penyebaran form kepada orang-orang yang termasuk ke dalam lingkup pengguna.
  \item Pengumpulan data dapat ditingkatkan dengan melakukan penyebaran form kepada orang-orang yang termasuk ke dalam lingkup pengguna.
  
  % \item Pengumpulan data sebaiknya memperhitungkan juga berapa kali pengguna membuka \textit{smartphone} secara harian, berhubung metrik pengukuran ini dapat menunjukkan seberapa terdistraksi pengguna untuk membuka \textit{smartphone}-nya.
  
  \item Skenario pengujian dapat dilengkapi dengan menyertakan komponen-komponen yang hanya dapat diinteraksi dari prototipe aplikasi yang dapat berjalan, seperti notifikasi aplikasi serta pemblokiran penggunaan aplikasi yang dinilai mendistraksi.
  
  \item \textit{Usability testing} prototipe solusi sebaiknya melibatkan lebih banyak partisipan untuk mengukur ketercapaian \textit{usability goals} dan \textit{user experience goals} dengan lebih baik, selain menemukan masalah \textit{usability} pada prototipe.
  
  \item Menambahkan \textit{dark mode} sebagai tema desain selain yang diterapkan pada prototipe yaitu \textit{light mode}.
  
  \item Membuat tampilan prototipe dalam bahasa Indonesia.
\end{enumerate}
