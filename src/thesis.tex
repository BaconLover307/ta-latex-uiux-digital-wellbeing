%--------------------------------------------------------------------%
%
% Berkas utama templat LaTeX.
%
% author Petra Barus, Peb Ruswono Aryan, Faris Rizki Ekananda
%
%--------------------------------------------------------------------%
%
% Berkas ini berisi struktur utama dokumen LaTeX yang akan dibuat.
%
%--------------------------------------------------------------------%

\documentclass[bahasa, 12pt, a4paper, onecolumn, oneside, final]{report}

\input{config/if-itb-thesis.sty}

\makeatletter

\makeatother

\addbibresource{references.bib}

\begin{document}

\sloppy

%Basic configuration
\title{Desain Interaksi Aplikasi Pencegah Distraksi Google Digital Wellbeing dengan Pendekatan \textit{User-Centered Design}}
\date{19 September 2022}
\author{
    GREGORIUS JOVAN KRESNADI \\
    NIM : 13518135
}

\pagenumbering{roman}
\setcounter{page}{1}

\clearpage
\pagestyle{empty}

\begin{center}
    \smallskip
    
    \Large \bfseries \MakeUppercase{Desain Interaksi} \\
    \Large \bfseries \MakeUppercase{Aplikasi Google Digital Wellbeing} \\
    \Large \bfseries \MakeUppercase{Dengan Pendekatan} \\
    \Large \bfseries \MakeUppercase{\textit{User-Centered Design}}
    \vfill
    
    \Large Laporan Tugas Akhir
    \vfill
    
    \large Disusun sebagai syarat kelulusan tingkat sarjana
    \vfill
    
    \large Oleh
    
    \Large \theauthor
    
    \vfill
    \begin{figure}[h]
        \centering
        \includegraphics[width=0.15\textwidth]{cover-ganesha.jpg}
    \end{figure}
    \vfill
    
    \large
    \uppercase{
        Program Studi Teknik Informatika \\
        Sekolah Teknik Elektro \& Informatika \\
        Institut Teknologi Bandung
    }
    
    September 2022
    
\end{center}

\clearpage


\titlespacing*{\chapter}{0pt}{0pt}{1pc}
\clearpage
\pagestyle{empty}

\begin{center}
    \smallskip

    \Large \bfseries \MakeUppercase{\thetitle}
    \vfill

    \Large Laporan Tugas Akhir
    \vfill

    \large Oleh

    \Large \theauthor

    \large Program Studi Teknik Informatika \\

    \normalsize \normalfont
    Sekolah Teknik Elektro dan Informatika \\
    Institut Teknologi Bandung \\

    \vfill
    \normalsize \normalfont
    Telah disetujui dan disahkan sebagai draft Laporan Tugas Akhir \\
    di Bandung, 08 Agustus 2022 \\
    Mengetahui,

    \vspace{0.5cm}
    Pembimbing,

    \vfill
    \underline{Adi Mulyanto, S.T, M.T.} \\
    NIP. 19631126 198803 1 002

\end{center}
\clearpage
 % 1 PEMBIMBING PAKAI INI
% \input{preamble/approval-2} % 2 PEMBIMBING PAKAI INI
\input{preamble/statement}

\pagestyle{plain}

\titlespacing*{\chapter}{0pt}{0pt}{0pc}
\clearpage
\chapter*{ABSTRAK}
\addcontentsline{toc}{chapter}{ABSTRAK}

\begin{center}
  \textbf{\MakeUppercase{\thetitle}} \\[1em]
  
  Oleh: \\
  \MakeUppercase{\theauthor} \\
\end{center}

\begin{singlespace}
  % $ latar belakang
  Penggunaan \textit{smartphone} yang semakin tinggi mempengaruhi kesejahteraan digital dari penggunanya secara negatif.
  Penggunaan \textit{smartphone} yang terlalu tinggi dapat membuat pengguna memiliki ketergantungan terhadap \textit{smartphone} hingga mencapai tingkat adiksi.
  Kasus ini memunculkan urgensi atas penelitian di bidang \textit{Human Computer Interaction} tentang kesengajaan untuk tidak menggunakan teknologi.
  Penelitian ini memunculkan konsep \textit{Digital Wellbeing} yang diadopsi Google untuk mengembangkan sebuah aplikasi berkonsep tersebut.
  Namun ditemukan bahwa aplikasi tersebut memiliki beberapa masalah yang tercerminkan pada banyaknya keluhan pada ulasan aplikasi tentang keterbatasan utilitas.
  
  % $ proses penelitian
  Oleh karena itu, diperlukan sebuah desain interaksi aplikasi yang dapat mengatasi masalah-masalah dari aplikasi Digital Wellbeing.
  Proses perancangan menggunakan metodologi \textit{User-Centered Design}.
  Pengumpulan data diawali dengan menganalisis ulasan dari aplikasi Digital Wellbeing, kemudian dilengkapi dengan wawancara kepada pengguna aplikasi.
  Hasil tugas akhir berupa prototipe \textit{high-fidelity} aplikasi untuk tampilan perangkat \textit{mobile} Android.
  Desain interaksi memprioritaskan \textit{usability goals utility} dan \textit{learnability}, serta mengarahkan kepada \textit{user experience goals helpful} dan \textit{motivating}.
  
  % $ hasil penelitian
  Pengujian dilakukan dengan \textit{usability testing} kepada target pengguna yang sesuai dengan persona yang ditentukan, menggunakan metrik pengukuran SUS, SEQ, dan IMI untuk subskala \textit{Value/Usefulness}, \textit{Interest/Enjoyment}, dan \textit{Pressure/Tension} untuk mengukur ketercapaian \textit{goals}.
  Hasil pengujian menunjukkan bahwa prototipe berhasil menyelesaikan masalah-masalah yang ditemukan, serta mencapai \textit{goals} yang diharapkan.
  Dari pengujian disimpulkan bahwa fitur Search bar, App Group, dan Daftar Jadwal Aktivasi dari prototipe berperan besar dalam meningkatkan utilitas aplikasi, serta fitur Daily Goal menjadi fitur unggulan dalam meningkatkan motivasi pengguna dalam memperbaiki kebiasaan digitalnya.

\noindent \textbf{Kata kunci:} \textit{Digital Wellbeing}, desain interaksi, \textit{user-centered design}, prototipe, \textit{widget}
\end{singlespace}
\clearpage
\chapter*{ABSTRACT}
\addcontentsline{toc}{chapter}{Abstract}

\begin{center}
  \textbf{\MakeUppercase{Interaction Design of Google Digital Wellbeing Application using User-Centered Design Approach}} \\[1em]
  
  By: \\
  \MakeUppercase{\theauthor} \\

\end{center}

\begin{singlespace}
  % $ latar belakang
  The increasing use of smartphones is negatively affecting the digital well-being of its users. Excessive use of smartphone can lead users to have high dependency to the point of addiction. This case raises the urgency for research in the field of Human Computer Interaction on the intentionality of not using technology. This research led to the concept of Digital Wellbeing, which Google has adopted to develop an app with such concept. However, it was found that the app had several problems according to complaints in the app reviews regarding the lack of utility provided.

  % $ proses penelitian
  Therefore, it is necessary to develop the interaction design for an application that can overcome the problems of the Google Digital Wellbeing application. The design process uses the User-Centered Design methodology. Data collection begins with analyzing reviews of the Digital Wellbeing application, then completed with interviews with application users. The result of the final project is a high-fidelity prototype of the application for Android mobile device display. The interaction design is designed by prioritizing usability goals of utility and learnability, and directing to user experience goals of helpful and motivating.

  % $ hasil penelitian
  Usability testing is done with targeted users in accordance with the specified personas, using measurement metrics such as SUS, SEQ, and IMI with Value/Usefulness, Interest/Enjoyment, and Pressure/Tension subscales to measure the achievement of goals. The test results showed that the prototype successfully solved the problems found and achieved the expected goals. It was concluded that the Search bar, App Group, and Activation Schedule features of the prototype play a major role in improving the utility of the application, and the Daily Goal feature is a leading feature in increasing user motivation to improve their digital habits.

  \noindent \textbf{Keywords:} Digital Wellbeing, interaction design, user-centered design, prototype, widget

\end{singlespace}

\titlespacing*{\chapter}{0pt}{0pt}{1pc}
\chapter*{\MakeUppercase{Kata Pengantar}}
\addcontentsline{toc}{chapter}{KATA PENGANTAR}

Puji syukur penulis ucapkan kepada Tuhan Yang Maha Esa atas berkat dan rahmat-Nya, sehingga penulis dapat menyelesaikan laporan penelitian tugas akhir ini yang berjudul "{\thetitle}". Keberhasilan penulis dalam menyelesaikan laporan ini tidak terlepas dari dukungan dan bantuan dari berbagai pihak. Untuk itu, penulis mengucapkan terima kasih kepada:

\begin{enumerate}
  \item Bapak Adi Mulyanto, S.T, M.T., selaku dosen pembimbing tugas akhir atas bimbingan dan masukan yang diberikan selama masa pengerjaan tugas akhir.
   
  % \item ... selaku dosen penguji tugas akhir yang telah memberikan saran, kritik, dan rekomendasi 
   
  \item Bapak Dicky Prima Satya, S.T., M.T., Bapak Adi Mulyanto, S.T., M.T., Ibu Latifa Dwiyanti, S.T., M.T., dan Bapak Nugraha Priya Utama, S.T., M.A., Ph.D., selaku koordinator tim tugas akhir yang memberikan arahan dalam pengerjaan tugas akhir ini,

  \item Ibu Dessi Puji Lestari, S.T., M.Eng., Ph.D., selaku Ketua Program Studi Teknik Informatika Institut Teknologi Bandung,
   
  \item Bapak Dr. Techn. Muhammad Zuhri Catur Candra, S.T, M.T. dan Ibu Ginar Santika Niwanputri, S.T, M.Sc., selaku dosen wali yang telah membimbing penulis selama berkuliah tiga tahun di Teknik Informatika,
  
  \item Bapak Yudistira Dwi Wardhana Asnar, S.T., Ph.D. yang telah memberikan ide dan inspirasi mengenai topik dari tugas akhir yang dikerjakan, 
  
  \item Para karyawan dan staff dari TU STEI yang telah membantu kebutuhan administrasi dalam menyelesaikan laporan Tugas Akhir,
  
  \item Orang tua, kakak, dan keluarga penulis yang senantiasa memberikan dukungan dan semangat selama menjalani perkuliahan di Teknik Informatika,

  \item Hollyana Puteri Haryono, selaku rekan terdekat penulis yang bersedia menemani serta memberikan semangat dan inspirasi dalam mengerjakan tugas akhir ini,
  
  % \item Teman-teman dari grup "Bunker": Naufal "Bapak" Prima, Michel "Peng" Fang, Junho Choi "Oppa" Hedyatmo, Jonathan "Jojo" Yudi, Matthew "Mek" Kevin, Kamal "Mastree" Shafi, Garry "Geri" Kuwanto, Morgen "Koh" Sudyanto, Muhammad "Euy" Hasan, Fabian "God" Zhafransyah, Reyvan "Berayfun" Rizky, Mario "Margun" Gunawan, Vincent "Lie" Lienardo, Faris "KissShot" Kautsar, Farras Hibban, Nafkhan "Camcam" Alzamzami, Fauzan "Kakek" Rafi, dan Naufal Dean yang telah memberi dukungan, bantuan, dan kata-kata pedas kepada penulis serta memberi hiburan  dengan menjadi teman-teman terbaik dalam masa perkuliahan.
  
  \item Teman-teman dari grup "Bunker": Naufal Prima, Michel Fang, Junho Choi, Jonathan Yudi, Matthew Kevin, Kamal Shafi, Garry Kuwanto, Morgen Sudyanto, Muhammad Hasan, Fabian Zhafransyah, Reyvan Rizky, Mario Gunawan, Vincentius Lienardo, Faris Kautsar, Farras Hibban, Camcam, Fauzan Rafi, dan Naufal Dean, yang telah memberi dukungan, bantuan, dan kata-kata pedas kepada penulis untuk mendorong penulis dalam mengerjakan tugas akhir, serta memberi hiburan dengan menjadi teman-teman terbaik dalam masa perkuliahan,
  
  \item Teman sebimbingan yang bersedia untuk saling menyemangati dan memberikan pendapat, kritik, saran terkait tugas akhir penulis,
  
  \item Teman-teman dari grup "Latex TA pipel", terutama Faris Rizki Ekananda, yang telah membantu penulis dalam mengatur format dokumen laporan tugas akhir,
  
  \item Gitta, Gian, dan teman-teman dari FNFBandung yang senantiasa menjadi penghibur dan penjaga kesehatan fisik maupun mental penulis selama pengerjaan tugas akhir,
  
  \item Para narasumber yang bersedia untuk membantu penulis sebagai partisipan pengumpulan data dan pengujian,

  \item Teman-teman mahasiswa Program Studi Teknik Informatika ITB 2018 yang senantiasa memberikan dukungan dan semangat dalam pengerjaan Tugas akhir,
   
  \item Pihak-pihak lain yang tidak dapat disebutkan satu per satu yang turut serta membantu penulis untuk menyelesaikan Tugas Akhir.
   
\end{enumerate}

Akhir kata, semoga penelitian tugas akhir ini dapat bermanfaat bagi semua pihak yang membutuhkannya

\vspace{1mm}

\begin{flushright}
  Bandung, 02 September 2022 \\
  \vspace{2.5cm}
  Penulis
\end{flushright}
\vfill


\titleformat*{\section}{\centering\bfseries\Large\MakeUpperCase}

% Setting judul toc, lot, lof, bib
\renewcommand{\contentsname}{DAFTAR ISI}
\renewcommand{\listfigurename}{DAFTAR GAMBAR}
\renewcommand{\listtablename}{DAFTAR TABEL}
\renewcommand{\bibname}{DAFTAR PUSTAKA}

\tableofcontents
\listofappendices
\listoffigures
\listoftables

\newpage

\titleformat*{\section}{\bfseries\large}
\pagenumbering{arabic}

%----------------------------------------------------------------%
% Konfigurasi Bab
%----------------------------------------------------------------%
\setcounter{page}{1}
\renewcommand{\chaptername}{BAB}
\renewcommand{\thechapter}{\Roman{chapter}}
%----------------------------------------------------------------%

%----------------------------------------------------------------%
% Dafter Bab
% Untuk menambahkan daftar bab, buat berkas bab misalnya `chapter-6` di direktori `chapters`, dan masukkan ke sini.
%----------------------------------------------------------------%
% \pagenumbering{}
\chapter{Pendahuluan}

Bab Pendahuluan secara umum menceritakan landasan kerja dan arah kerja penulis tugas akhir, yang berfungsi untuk mengantar pembaca dalam memahami dan menganalisis laporan tugas akhir secara keseluruhan. Bab ini terdiri dari latar belakang, rumusan masalah, tujuan, batasan masalah, metodologi, dan jadwal pelaksanaan tugas akhir.

\section{Latar Belakang}
\label{sec:latarbelakang}

Teknologi dan internet adalah kebutuhan yang penting dalam kehidupan manusia di abad ke-21 ini. Salah satu bentuk teknologi modern paling berguna yang mudah diakses manusia adalah \emph{smartphone}, di mana akses internet sudah termasuk ke dalamnya. Berdasarkan penelitian yang dikumpulkan oleh \textcite{turner2022howmanysmartphones} pada bankmycell.com, pada tahun 2021 terdapat 6.37 miliar pengguna \emph{smartphone} di dunia atau sekitar 80.68\% dari populasi dunia, di antaranya 160.23 juta adalah penduduk Indonesia.

Banyak pekerjaan manusia yang dapat dilakukan dengan menggunakan berbagai jenis aplikasi yang dapat ditemukan dalam sebuah \emph{smartphone}. Menurut penelitian yang dilakukan oleh BusinessofApps, jumlah aplikasi pada iOS App Store ketika pertama kali diluncurkan pada tahun 2008 adalah 500, pada bulan November tahun 2021 angka tersebut melonjak menjadi 1.85 juta aplikasi. Pengguna Android memiliki lebih banyak pilihan dengan total 2.56 juta aplikasi di Google Play Store. Dengan banyaknya aplikasi tersebut, pada tahun 2020 pengguna \emph{smartphone} telah mengunduh aplikasi sebanyak 142.9 miliar kali, di antaranya 56.1 miliar unduhan adalah untuk aplikasi permainan \emph{mobile}.

Teknologi informasi yang semakin berkembang dapat meningkatkan kesejahteraaan digital, atau digital wellbeing, dari pengguna teknologi tersebut, beberapa caranya adalah meningkatkan koneksi sosial, mendukung kesehatan mental, dan mendukung kebiasaan sehat dengan fleksibilitas dalam praktik kerja. Namun teknologi informasi juga dapat menimbulkan akibat buruk yang memunculkan pola penggunaan teknologi yang tidak sehat. Beberapa contoh dampak buruk yang dimunculkan adalah mudahnya seseorang untuk kehilangan fokus, perasaan \emph{fear of missing out} (FoMO), serta kecanduan digital. Sifat-sifat buruk tersebut dapat lebih sering muncul ketika waktu yang dihabiskan pada dunia maya tidak diimbangi dengan waktu di dunia nyata. \parencite{ALMOURAD2021101778}

Pengaruh buruk seperti yang telah disebutkan di atas serta perilaku adiksi pada \textit{smartphone} telah memunculkan perhatian bagi peneliti di bidang \textit{Human Computer Interaction} (HCI) untuk melakukan studi terhadap kesengajaan untuk tidak menggunakan teknologi. Studi tersebut memunculkan sebuah konsep "Digital Wellbeing" yang berfokus dalam peningkatan kesejahteraan pengguna dalam pemakaian media digital. \parencite{unesco2015dwconference}. Pada saat ini, terdapat banyak perangkat lunak pada \textit{smartphone} yang menerapkan konsep tersebut untuk mengubah perilaku penggunanya, salah satunya dibuat oleh Google. \parencite{CHI2019SOCIALIZE} Untuk membantu penggunanya, Google merilis aplikasi dengan nama yang sama, Digital Wellbeing, agar pengguna dapat memanfaatkan teknologi untuk meningkatkan kualitas kehidupannya melainkan menjadi distraksi. Pada websitenya tentang Digital Wellbeing, \textcite{google2019digitalwellbeing} mengatakan: "We’re committed to giving everyone the tools they need to develop their own sense of digital wellbeing. So that life, not the technology in it, stays front and center." 


% Salah satu fitur yang terdapat pada aplikasi Digital Wellbeing adalah Focus Mode, sebuah fitur yang dapat membantu mengurangi distraksi dari \emph{smartphone} berbasis Android dengan cara memblokir sementara aplikasi yang dinilai sebagai distraksi. Focus Mode juga akan mendiamkan notifikasi yang masuk hingga waktu yang ditentukan. Waktu yang diterapkan untuk Focus Mode dapat diatur jadwalnya oleh pengguna. \parencite{android2019digitalwellbeing} Selama Focus Mode aktif, akan terdapat sebuah menu pada bagian notifikasi \emph{smartphone} dengan 2 buah fitur. Fitur "Take a break" yang berbentuk tombol ini dapat memberikan pengguna kembali akses untuk aplikasi yang diblok dengan pilihan waktu 5 menit, 15 menit, atau 30 menit. Tombol ini dapat digunakan jika pengguna ingin beristirahat dan menggunakan aplikasi yang diblok. Fitur "Turn off for now" yang juga berbentuk tombol dapat digunakan untuk memberhentikan Focus Mode hanya untuk hari tersebut. Tombol ini dapat digunakan jika pengguna merasa tugas yang perlu difokuskannya sudah selesai dan ingin menggunakan aplikasi yang diblok untuk sisa harinya. Masalah yang ditemukan dari Focus Mode terletak pada kedua fitur ini. Walau memerlukan interaksi lebih untuk "beristirahat", terdapat kemungkinan bagi pengguna untuk terus menerus mengakses aplikasi, dengan mengambil istirahat setiap beberapa menit sekali. Selain itu, fitur "Turn off for now" dapat disalahgunakan untuk menghindari Focus Mode secara keseluruhan.

\section{Rumusan Masalah}

Penggunaan tinggi \textit{smartphone} dapat mempengaruhi kesejahteraan digital dari penggunanya dalam cara yang negatif. Salah satu bentuk dari hubungan yang tidak baik antara \textit{smartphone} dan penggunanya adalah tingginya tingkat ketergantungan terhadap \textit{smartphone} hingga dapat mencapai tingkat adiksi. Penurunan kualitas hubungan tersebut dapat disebabkan oleh gangguan distraksi dari aplikasi pada \textit{smartphone} atau keinginan pengguna sendiri untuk menggunakannya. Dalam upaya memperbaiki hubungan tersebut, Google merilis aplikasi Digital Wellbeing yang bertugas untuk membantu pengguna \textit{smartphone} berbasis Android untuk mengatur kesehatan penggunaannya. Namun, dengan banyaknya aplikasi dengan tujuan serupa di \textit{marketplace} Google Play Store, dapat dilihat bahwa terdapat beberapa batasan pada aplikasi Digital Wellbeing. Maka dari itu, dapat disimpulkan rumusan masalah yang akan didalami dalam Tugas Akhir ini adalah sebagai berikut

\begin{enumerate}
  \item Apa \textit{usability goals} dan \textit{user experience goals} yang tepat untuk sebuah aplikasi pencegah distraksi?
  \item Bagaimana rancangan desain interaksi yang tepat untuk menyelesaikan masalah yang terdapat pada aplikasi pencegah distraksi Digital Wellbeing yang disusun menggunakan pendekatan \textit{user-centered design}?
  % \item Bagaimana rancangan desain interaksi yang tepat untuk menyelesaikan masalah yang terdapat pada aplikasi pencegah distraksi Digital Wellbeing?
  % \item Bagaimana prototipe aplikasi pencegah distraksi Digital Wellbeing dengan desain interaksi menggunakan pendekatan \emph{user-centered design}?
\end{enumerate}

\section{Tujuan}

Berdasarkan latar belakang dan rumusan masalah di atas, tujuan dari Tugas Akhir ini adalah untuk membuat sebuah prototipe aplikasi pencegah distraksi. Prototipe aplikasi tersebut memiliki desain interaksi yang dapat menyelesaikan masalah-masalah desain interaksi yang ditemukan pada alat Digital Wellbeing milik Google.


\section{Batasan Masalah}

Batasan masalah untuk implementasi solusi Tugas Akhir ini adalah sebagai berikut
\begin{enumerate}
  % \item Hasil akhir dari studi adalah solusi dalam bentuk prototipe aplikasi untuk \textit{smartphone} berbasis Android dengan versi OS lebih besar sama dengan versi Android 9.0.
  \item Responden penelitian adalah masyarakat Indonesia, dengan rentang umur 18 hingga 30 tahun.
  \item Hasil akhir dari studi adalah solusi dalam bentuk prototipe aplikasi yang didesain untuk \textit{smartphone} berbasis Android.
  % \item Pengujian akan dilakukan dengan membandingkan penggunaan \textit{smartphone} yang disertai bantuan aplikasi Digital Wellbeing dengan prototipe aplikasi solusi.
\end{enumerate}

\section{Metodologi}
\label{sec:metodologi}

Metodologi dalam perancangan solusi Tugas Akhir menggunakan pendekatan \textit{user-centered design} (UCD) yang mengikuti standar alur kerja dari ISO (\textit{International Organization for Standardization}) 9241-210:2010. Berikut adalah penjelasan dari setiap tahap UCD


\begin{figure}[h]
  \centering
  \includegraphics[width=0.9\textwidth]{chapter-1-method.png}
  \caption{Diagram alur pengerjaan \textit{User-Centered Design} (ISO 9241-210, 2010)}
  \label{fig:diagram_iso1}
\end{figure}

\begin{enumerate}
  \item Perencanaan proses desain
  \subitem Proses UCD diawali dengan tahap persiapan, yaitu proses perancangan terhadap lingkup aplikasi yang dibuat serta perencanaan untuk pengambilan data. Lingkup dari aplikasi termasuk jenis antarmuka dan \textit{platform} yang dipilih untuk implementasi, lingkup fungsionalitas aplikasi, serta target pengguna aplikasi. Sedangkan perencanaan pengambilan data dilakukan dengan cara analisis ulasan aplikasi dan wawancara pengguna.

  \item Identifikasi konteks penggunaan
  \subitem Pada tahap ini dilakukan pengumpulan data pengguna melalui analisis ulasan pengguna dan wawancara sesuai dengan kebutuhan. Data yang didapatkan akan dianalisis untuk mengungkapkan perilaku dan permasalahan pengguna tentang aplikasi. Fungsionalitas dari aplikasi juga penting untuk mengerti konteks penggunaannya. Berdasarkan data riset tersebut akan dibentuk persona pengguna, yang kemudian akan dianalisis untuk mendapatkan kebutuhan, tujuan dan kegiatan pengguna, serta skenario pengguna.
   
  \item Penentuan kebutuhan perangkat lunak
  \subitem Tahap ini terdiri dari analisis tipe interaksi, analisis fitur, analisis prinsip desain, serta \textit{user experience goals} dan \textit{usability goals}. Fitur-fitur yang terkumpul akan menjadi bahan implementasi pada tahap selanjutnya.
  
  \item Perancangan prototipe perangkat lunak
  \subitem Rancangan kebutuhan pengguna yang sudah dikumpulkan pada tahap sebelumnya akan kemudian diimplementasikan, mulai dari \textit{low-fidelity prototype}, dilanjut dengan \textit{high-fidelity prototype} berupa prototipe aplikasi. Proses ini dilakukan secara iteratif bersamaan dengan tahap evaluasi.
  
  \item Evaluasi prototipe perangkat lunak
  \subitem Tahap evaluasi dilakukan untuk menguji kemampuan solusi desain yang telah dirancang dalam memenuhi kebutuhan pengguna dan \textit{user experience goals} serta \textit{usability goals} yang ditargetkan. Hasil evaluasi akan menentukan apakah desain yang diuji perlu diperbaiki dalam proses iterasi.
  
\end{enumerate}


\section{Sistematika Pembahasan}

\begin{enumerate}
  \item Bab I Pendahuluan
  \subitem Bab I adalah pendahuluan yang berisi latar belakang dari pengerjaan tugas akhir, rumusan masalah yang ingin diselesaikan, tujuan yang ingin dicapai, batasan masalah yang ditentukan, serta metode yang digunakan selama proses pengerjaan, serta sistematika penulisan tugas akhir.
   
  \item Bab II Studi Literatur
  \subitem Bab II membahas tentang studi literatur sebagai landasan teori untuk bab III dan bab IV. Pada bab ini, dituliskan hasil studi literatur terhadap adiksi smartphone, Digital Wellbeing, desain interaksi, \textit{usability testing}, serta penelitian terkait tugas akhir.
 
  \item Bab III Identifikasi Masalah dan Rancangan Solusi
  \subitem Bab III membahas proses identifikasi masalah dan perancangan solusi yang akan dibentuk. Bab ini memuat tiga tahap pertama dari metodologi \textit{user-centered design}, yaitu perancangan proses desain, identifikasi konteks masalah, dan penentuan kebutuhan perangkay lunak.
 
  \item Bab IV Implementasi dan Pengujian Prototipe
  \subitem Bab IV memuat tahap lanjutan dari \textit{user-centered design}, yaitu perancangan prototipe perangkat lunak serta evaluasi prototipe perangkat lunak. Pada bab ini, kedua tahap tersebut mengalami iterasi, diawali dengan implementasi dan pengujian prototipe \textit{low-fidelity}, implementasi dan pengujian prototipe \textit{high-fidelity} iterasi pertama, serta implementasi dan pengujian prototipe \textit{high-fidelity} iterasi kedua yang ditetapkan sebagai solusi desain yang memenuhi tujuan tugas akhir.
 
  \item Bab V Kesimpulan dan Saran
  \subitem Bab V memaparkan kesimpulan dari proses perancangan prototipe desain solusi tugas akhir. Bab ini juga memuat saran mengenai hal yang dapat dilakukan ke depannya untuk memperbaiki solusi yang dibuat pada tugas akhir ini.
\end{enumerate}
\chapter{Studi Literatur}


\newcommand{\cbnormspacing}{\baselineskip=12pt}

Bab Studi Literatur digunakan untuk membahas kajian literatur yang terkait dengan persoalan tugas akhir ini. Pembahasan meliputi Adiksi \textit{Smartphone}, Digital Wellbeing, dan Desain Interaksi.

% * =======================================================================
% *   ||  ||  ||  ||  ||  ||  ||  ||  ||  ||  ||  ||  ||  ||  ||  ||  ||
% * =======================================================================

\section{Adiksi \textit{Smartphone}}
Seperti yang telah disebutkan pada subbab \ref{sec:latarbelakang}, \textit{smartphone} yang telah menjadi bagian dari kehidupan sehari-hari manusia memiliki banyak fungsionalitas yang dapat meningkatkan kualitas hidup manusia, tapi di sisi lain dapat memberikan pengaruh buruk. Efek negatif yang timbul dari pengaruh-pengaruh buruk tersebut menunjukkan kemiripan pada pola-pola perilaku korban adiksi. Menurut penelitian tentang \textit{Smartphone Addiction Scale} oleh \textcite{10.1371/journal.pone.0083558}, walaupun \textit{smartphone addiction} belum terdaftar sebagai \textit{behavioral addiction} dalam DSM-5 (\textit{Diagnostic and Statistical Manual of Mental Disorders}), sebuah standar klasifikasi terhadap penyakit mental yang digunakan oleh ahli kesehatan mental di Amerika Serikat, adiksi untuk aktivitas yang dapat dilakukan pada internet melalui \textit{smartphone}, seperti bermain gim, \textit{chatting}, dan pornografi menunjukkan tingkat adiksi yang sama dengan korban adiksi narkotika dan alkohol.

Menurut penelitian oleh \textcite{CHI2019SOCIALIZE}, penggunaan \textit{smartphone} yang berlebihan menimbulkan pengaruh negatif terhadap kesehatan mental dan interaksi sosial. Hal ini terlihat pada kualitas interaksi sesama secara langsung, yang biasanya membutuhkan usaha dan komitmen untuk menjalin hubungan baik, terpengaruh oleh konsep interaksi tidak langsung melalui \textit{smartphone}, di mana hubungan lebih menyebar dengan lebih sedikit interaksi.

Roffarello dan De Russis melanjutkan \textit{smartphone} juga sering berperan sebagai sumber distraksi yang mengalihkan perhatian dari kegiatan penting. Distraksi tersebut dapat berasal dari stimuli eksternal seperti notifikasi pada \textit{smartphone}, namun dapat juga dari stimuli internal seperti keinginan untuk memeriksa email atau bermain gim. Pengguna yang merasakan gangguan internal dan eksternal rutin dan tidak dapat diprediksi ini cenderung merasa tidak produktif dan lebih sering stress.


% * =======================================================================
% *   ||  ||  ||  ||  ||  ||  ||  ||  ||  ||  ||  ||  ||  ||  ||  ||  ||
% * =======================================================================

\section{\textit{Digital Wellbeing}}
\label{sec:digital_wellbeing}

Untuk menanggapi permasalahan pada penggunaan \textit{smartphone} yang berlebihan, peneliti di bidang HCI mulai gencar untuk melakukan studi terhadap kesengajaan untuk tidak menggunakan teknologi. Sebagai jawabannya, peneliti, perusahaan, dan organisasi dunia muncul dengan istilah \textit{Digital Wellbeing} untuk menyatakan kesehatan hubungan antara pengguna dan teknologinya. 

Menurut Forum for Well-being in Digital Media yang ada di bawah \textcite{unesco2015dwconference}, \textit{Digital Wellbeing} adalah peningkatan kesejahteraan pengguna dalam pemakaian media digital. Kesejahteraan yang dimaksud adalah aset psikologis berharga seseorang untuk bertahan hidup dan merasakan pengalaman positif yang berkelanjutan. Sebuah lingkungan atau media digital untuk dapat memberikan pengembangan kesejahteraan jangka panjang bagi penggunanya perlu memenuhi syarat-syarat berikut:

\begin{enumerate}
  \item Membawakan rasa sambut dan empati di antara penggunanya,
  \item Mendorong pengembangan rasa kompetensi untuk penggunanya,
  \item Memungkinkan penggunanya untuk bertingkah sesuai hati nuraninya, dan
  \item Mendorong penggunanya untuk bereksplorasi pada bidang yang diminati untuk memicu pengembangan diri.
\end{enumerate}

Salah satu perusahaan yang memunculkan komitmen untuk menanggulangi permasalahan \textit{smartphone addiction} adalah Google. Pada bulan Mei tahun 2018 dalam konferensi Google I/O, Google meluncurkan langkah \textit{Digital Wellbeing}, sebuah filosofi desain yang akan dipakai dalam produk-produknya yang bertujuan untuk memberikan hubungan yang lebih baik antara pengguna dan teknologi yang dipakainya. Dalam sebuah \textit{online course} yang disediakan Google dijelaskan bahwa \textit{Digital Wellbeing} adalah tentang membuat sebuah hubungan yang sehat dengan teknologi dan menjaga kesehatan hubungan tersebut. Google menyadari bahwa teknologi yang berkembang dengan pesat telah menjadi sebuah tantangan dalam menjaga keseimbangan waktu yang dihabiskan antara dunia nyata dan dunia maya. Maka dari itu konsep \textit{Digital Wellbeing} yang dibawakan oleh Google mengajak penggunanya untuk mengambil kendali teknologi agar dapat memberikan manfaat dan potensial semaksimal mungkin serta membantu mencapai tujuan, melainkan menjadi pengganggu, distraksi, atau rintangan \parencite{google2019dwcourse}.

\subsection{Manfaat \textit{Digital Wellbeing}}

Melalui bantuan konsep \textit{Digital Wellbeing}, membuat sebuah kebiasaan yang sehat dalam menggunakan teknologi dapat memberikan penggunanya beberapa manfaat. Menurut Google, manfaat yang didapatkan adalah sebagai berikut:

\begin{enumerate}
  \item Meningkatkan fokus untuk digunakan pada kegiatan utama
  \item Menjaga atau memperbaiki hubungan sesama berkat tersedianya perhatian penuh untuk lawan bicara
  \item Meningkatkan produktifitas serta efektifitas dalam pekerjaan
  \item Meningkatkan keterlibatan serta kesadaran diri atas lingkungannya
\end{enumerate}

\subsection{Cara Penerapan \textit{Digital Wellbeing}}

Untuk mencapai manfaat-manfaat yang telah disebutkan sebelumnya, berikut adalah beberapa cara yang perlu dilakukan menurut Google:

\begin{enumerate}
  \item Meningkatkan kesadaran diri terhadap kebiasaan digital di dunia maya
  \subitem Untuk mengubah pola pikir dan perilaku, langkah pertama terbaik yang sebaiknya dilakukan dapat dimulai dari diri sendiri. Merefleksikan berapa banyak waktu yang dihabiskan di dunia maya dapat menyadarkan terhadap kebiasaan digital yang dilakukan sehari-hari. Dari sana, seseorang dapat menilai apakah mereka puas akan kebiasaan-kebiasaan tersebut. Hal ini juga perlu dilakukan sendiri karena penggunaan teknologi antarindividu dipastikan berbeda.
  
  \item Menyadari ulang tujuan utama dari pemakaian teknologi digital
  \subitem Terkadang seseorang dapat terjebak dalam kebiasaan digitalnya sehingga mereka hanya melakukan atau memakai teknologi tanpa menyadari apa yang ingin dicapai. Mengambil langkah mundur untuk berefleksi dapat menyadarkan diri akan tujuan utama dari pemakaian teknologi digital dan mengadakan kemungkinan untuk mengubah pola pemakaian tersebut.
  
  \item Meminta pertolongan eksternal untuk menilai kebiasaan digital diri
  \subitem Mendapatkan perspektif lain adalah cara yang baik dalam melakukan refleksi karena terkadang penilaian diri dapat bersifat subjektif atau terjadi estimasi yang tidak akurat dari nilai aslinya. Dengan meminta bantuan teman, rekan kerja, atau keluarga yang dapat membantu memantau kebiasaan digital diri dapat membuka perspektif baru untuk mengkonfirmasi penilaian diri sendiri.
  
  \item Memantau penggunaan teknologi digital dengan bantuan alat
  \subitem Keberadaan data yang jelas tentang penggunaan teknologi digital, seperti waktu penggunaan aplikasi dan jumlah notifikasi yang diterima dapat membantu memberi gambaran saat melakukan refleksi. Aplikasi-aplikasi untuk memantau aktivitas tersebut dapat didapatkan dengan mudah di \textit{mobile appstore} pada masing-masing platform atau di \textit{website} untuk PC.
  
  \item Membuat perubahan kecil untuk membentuk kebiasaan baru
  \subitem Setelah mendapatkan tujuan dan bayangan akan bagaimana kebiasaan yang ingin dicapai, langkah-langkah dapat diambil untuk membentuk kebiasaan lama menjadi yang lebih sehat dan bermanfaat. Langkah-langkah yang diambil dapat bertahap dan tidak terlalu besar, hal ini ditujukan agar tidak memaksa diri terlalu jauh dan memicu stress yang tidak diinginkan dari perubahan yang terlalu besar.
  
\end{enumerate}

  \subsection{Panduan Penerapan \textit{Digital Wellbeing}}
  
  Google menyadari bahwa manfaat-manfaat dari \textit{Digital Wellbeing} tidak dapat dicapai dengan cara yang sama untuk semua orang. Oleh karena itu, terdapat beberapa opsi yang disarankan oleh Google untuk menerapkan \textit{Digital Wellbeing} yang terbagi ke dalam 2 kategori panduan, yaitu panduan digital dan panduan fisik.
  
  \subsubsection{Panduan Digital}
  
  Panduan digital adalah kumpulan aplikasi serta teknologi yang didesain untuk membantu pengguna teknologi digital untuk mengambil alih kendali terhadap teknologi yang dipakai. Berikut adalah beberapa panduan digital yang disarankan oleh Google:
  
  \begin{enumerate}
    \item Meminimalisir masuknya notifikasi
    \item Mengubah warna tampilan \textit{smartphone} menjadi berskala abu-abu
    \item Mengatur \textit{smartphone} ke dalam mode Do Not Disturb
    \item Membatasi jumlah aplikasi atau alat pada layar utama
  \end{enumerate}

  \subsubsection{Panduan Fisik}

  Panduan fisik adalah panduan penerapan \textit{Digital Wellbeing} yang memandang dari segi lingkungan atau ruang personal di sekitar diri. Panduan ini dapat dilakukan dengan atau tanpa bantuan teknologi, namun diutamakan untuk kondisi ketidakberadaannya teknologi. Berikut adalah beberapa panduan fisik yang disarankan oleh Google:

  \begin{enumerate}
    \item Menghabiskan waktu sebanyak mungkin di luar ruangan
    \item Memulai dan mengakhiri hari tanpa menggunakan \textit{smartphone}
    \item Melakukan pertemuan atau percakapan tanpa melibatkan perangkat digital
    \item Membedakan perangkat yang digunakan untuk keperluan pekerjaan dengan keperluan hidup sehari-hari
    \item Meletakkan \textit{smartphone} di lokasi yang berbeda dari tempat bekerja
    \item Menjadwalkan akses terhadap \textit{e-mail} 
  \end{enumerate}

\subsection{Prinsip Desain \textit{Digital Wellbeing}}
\label{subsec:prinsip_desain_dw}

Dalam upaya membantu \textit{developer-developer} lain mendukung konsep dari \textit{Digital Wellbeing}, \textcite{google2021dwframework} telah menyusun sebuah kakas desain berisi prinsip-prinsip desain interaksi untuk diterapkan kepada produk yang dibuat. Tujuan Google membuat kakas ini adalah agar \textit{developer} dapat membuat produk atau aplikasi yang mampu meningkatkan \textit{digital wellbeing} penggunanya dengan mendukung intensi baik dalam menggunakan teknologi. Berikut adalah penjelasan dari prinsip-prinsip desain tersebut

\begin{enumerate}
  \item \textit{Empowerment}
  \subitem Produk sebaiknya memiliki pengaturan \textit{default} yang dapat mendukung pengguna untuk memperbaiki perilakunya. Ketika terdapat pilihan terhadap pengaturan, sebaiknya tampilan yang pertama kali dilihat pengguna adalah pengaturan \textit{default} yang memberikan bayangan terbaik untuk mendukung tujuan pengguna. Namun pengguna tetap perlu dapat mengubahnya jika diperlukan.

  \item \textit{Awareness}
  \subitem Tindakan refleksi diri membuat orang lebih sadar tentang perilakunya atau bagaimana seseorang menggunakan waktunya. Menampilkan data penggunaan dengan langsung dapat mendorong pengguna untuk merefleksi perilakunya, membantunya dalam meluruskan tujuan mereka untuk memperbaiki kebiasaan serta menentukan langkah yang harus diambil. Contoh pemaparan yang dapat membantu adalah tampilan \textit{dashboard}, visualisasi data, dan wawasan perilaku pengguna.

  \item \textit{Control}
  \subitem Transparansi terhadap pengaturan dapat mengantisipasi pengguna dengan kebutuhan, kemampuan, dan latar belakang yang beragam. Pemberian pengaturan yang mendalam dapat membantu banyak pengguna mencapai tujuan spesifik mereka. Hal ini harus didukung dengan penjelasan yang jelas dan transparansi tentang cara kerja atau fungsi dari pengaturannya, termasuk bagaimana data pengguna dikumpulkan dan digunakan.

  \item \textit{Adaptability}
  \subitem Perlu dipertimbangkan kemampuan fitur untuk beradaptasi terhadap beragam konteks pengguna. Dengan mengintegrasikan pengalaman penggunaan aplikasi dengan konteks pengguna seperti waktu, lokasi, atau perangkat yang sedang digunakan dapat mengurangi beban navigasi pengguna.

\end{enumerate}

\subsection{Aplikasi Google Digital Wellbeing}

Aplikasi Google Digital Wellbeing yang telah disebutkan pada awal subbab \ref{sec:digital_wellbeing} adalah bagian dari langkah \textit{Digital Wellbeing} yang diluncurkan pada konferensi Google I/O. Aplikasi ini sudah terintegrasi pada sistem operasi Android sejak versi Android 9.0 \parencite{google2021dwsupport}. Menurut \textcite{8976353}, aplikasi ini berperan sebagai alat untuk membantu mengoptimisasi penggunaan \textit{smartphone}, didesain untuk membantu penggunanya hidup berdampingan dengan teknologi digital yang selalu menarik perhatian dan menyita waktu.

Fitur-fitur yang terdapat di aplikasi ini didesain untuk membantu penggunanya menerapkan konsep \textit{Digital Wellbeing} dengan menggunakan panduan penerapan konsep tersebut dalam desain aplikasinya. Berikut adalah fitur-fitur yang tersedia.

% https://lup.lub.lu.se/luur/download?func=downloadFile&recordOId=8976353&fileOId=8981518
% https://static.googleusercontent.com/media/wellbeing.google/en//static/pdf/digital-wellbeing-product-experience-toolkit.pdf
% https://experiments.withgoogle.com/collection/digitalwellbeing
% Cold Turkey

\subsubsection{Dashboard}
Dashboard adalah fitur yang berperan seperti kendali pusat dari aplikasi Google Digital Wellbeing. Dashboard dapat menampilkan jumlah waktu yang dihabiskan untuk membuka aplikasi, jumlah berapa kali pengguna membuka \textit{smartphone}, dan jumlah notifikasi yang diterima pada hari tersebut dalam sebuah grafik \parencite{android2019digitalwellbeing}. Dashboard juga memiliki kemampuan untuk menampilkan ringkasan dari data pada hari-hari sebelumnya, memungkinkan pengguna untuk memantau dan menganalisis kebiasaannya. Gambar fitur pada aplikasi dapat dilihat pada Lampiran \ref{chpt:gambar_dw}.

\subsubsection{App Timers}
App Timers adalah fitur yang memungkinkan pengguna untuk memberikan batas waktu akses pada aplikasi tertentu. Fitur ini berperan sebagai pelengkap dari fitur Dashboard yang telah disebutkan. Jika pengguna telah mengakses aplikasi yang diatur melebihi dari batas waktu yang ditentukan, maka semua notifikasinya akan diheningkan dan pengguna tidak dapat mengakses aplikasinya lagi untuk hari tersebut \parencite{android2019digitalwellbeing}. Ikon aplikasi yang diblokir akan memiliki warna berskala abu-abu, serta jika ditekan akan ada pengingat bahwa penggunaan aplikasi tersebut telah mencapai batas waktu sehingga dapat dilanjutkan esok hari. Gambar fitur pada aplikasi dapat dilihat pada Lampiran \ref{chpt:gambar_dw}.

\subsubsection{Bedtime Mode}
Bedtime Mode adalah fitur yang bertujuan untuk membantu penggunanya menjaga jam tidur yang sehat. Fitur Bedtime Mode akan mengubah warna tampilan \textit{smartphone} menjadi berskala abu-abu, dan menghambat notifikasi yang masuk dengan bantuan fitur Do Not Disturb. Bedtime Mode memungkinkan pengguna untuk mengatur jadwal tidurnya dari jam mulai tidur, jam bangun, serta hari apa saja fitur tersebut akan menyala. Bedtime Mode juga dapat diatur untuk menyala hanya saat pengisian baterai \textit{smartphone} \parencite{android2019digitalwellbeing}. Bedtime Mode juga memiliki kemampuan untuk memberikan notifikasi kepada pengguna untuk mengingatkan bahwa mode akan segera aktif. Gambar fitur pada aplikasi dapat dilihat pada Lampiran \ref{chpt:gambar_dw}.

\subsubsection{Focus Mode}
Focus Mode adalah fitur yang bertujuan untuk membantu penggunanya memblokir distraksi dari \textit{smartphone} dan memfokuskan diri untuk bekerja. Focus Mode memungkinkan pengguna untuk memilih aplikasi yang dinilai dapat menjadi distraksi, kemudian memblokir notifikasi dari aplikasi tersebut serta memblokir akses untuk membuka aplikasi tersebut. Pengguna juga dapat mengatur jadwal menyalanya fitur Focus Mode \parencite{android2019digitalwellbeing}.

Pada saat Focus Mode aktif, ikon aplikasi pada halaman utama serta \textit{app drawer} akan berskala abu-abu. Ketika ikon diklik maka akan muncul sebuah pesan yang mengingatkan bahwa aplikasi tersebut dinilai sebagai distraksi dan sedang diblokir sementara. Kemudian pengguna memiliki pilihan untuk menutupnya atau menggunakan aplikasi tersebut selama 5 menit, hal ini bertujuan agar pengguna dapat menggunakan aplikasi tersebut dalam kondisi darurat. Selain itu, status aktif Focus Mode akan ditampilkan pada bar notifikasi disertai 2 tombol, "Take a break" dan "Turn off for now". Tombol "Take a break" dapat memberikan pilihan kepada pengguna untuk mematikan Focus Mode selama 5 menit, 15 menit, atau 30 menit. Tombol ini bertujuan agar pengguna dapat beristirahat sejenak dari sesi pekerjaannya dan menggunakan aplikasi-aplikasi yang dinilai sebagai distraksi. Tombol "Turn off for now" dapat mematikan Focus Mode pada hari tersebut walaupun jadwal yang ditentukan belum terpenuhi. Gambar fitur pada aplikasi dapat dilihat pada Lampiran \ref{chpt:gambar_dw}.


\subsubsection{Do Not Disturb}
Do Not Disturb adalah fitur yang bertujuan untuk membantu penggunanya memblokir gangguan dari notifikasi pada \textit{smartphone}. Fitur Do Not Disturb akan mematikan suara dari \textit{smartphone} sehingga notifikasi atau panggilan yang masuk tidak mampu mengeluarkan suara. Selain itu, saat Do Not Disturb aktif maka layar \textit{smartphone} tidak akan menyala saat adanya notifikasi atau panggilan yang masuk \parencite{android2019digitalwellbeing}. Selain dari aplikasi Digital Wellbeing, fitur ini juga dapat diaktifkan dari \textit{control center}.

\subsubsection{Customize Notifications}
Customize Notifications adalah fitur yang memungkinkan penggunanya untuk mengatur notifikasi yang diterima. Pengguna dapat mengatur notifikasi dari aplikasi apa saja yang dapat diterima oleh \textit{smartphone} serta bagaimana bentuk notifikasi yang ingin diterima. Pengaturan notifikasi juga dapat diatur untuk fitur spesifik dari sebuah aplikasi, jika aplikasi tersebut memberikan izin bagi pengguna untuk melakukan pengaturan tersebut. 


% * =======================================================================
% *   ||  ||  ||  ||  ||  ||  ||  ||  ||  ||  ||  ||  ||  ||  ||  ||  ||
% * =======================================================================



\section{Desain Interaksi}

Menurut \textcite{PreeceRogersSharp15}, desain interaksi adalah proses mendesain suatu produk yang interaktif untuk menciptakan sebuah pengalaman yang meningkatkan kualitas dari cara kerja, komunikasi, dan interaksi dari pengguna produk. Untuk memberi konteks tentang apa yang didesain, beberapa aspek yang seringkali ditegaskan adalah \textit{user interface} (UI), rekayasa perangkat lunak, \textit{user-centered design}, dan desain produk. Desain interaksi juga dapat dilihat sebagai basis yang fundamental dalam beberapa displin, bidang, dan pendekatan yang berhubungan dengan proses penelitian dan desain sistem berbasis komputer. Maka dari itu, seringkali beberapa aspek dari pendekatan-pendekatan yang memakai pedoman desain interaksi seringkali bertumpang tindih. Preece dkk. mengilustrasikannya pada Gambar \ref{fig:desain_interaksi}.


\begin{figure}[h]
  \centering
  \includegraphics[width=\textwidth]{chapter-2-interaction-design.png}
  \caption{Hubungan studi antardisiplin terkait desain interaksi (Panah dua arah berarti saling tumpang tindih) \textcite{PreeceRogersSharp15}}
  \label{fig:desain_interaksi}
\end{figure}

\FloatBarrier

\subsection{Pendekatan Desain Interaksi}
\label{subsec:pendekatan_id}

Dalam desain interaksi terdapat beberapa pendekatan utama yang dapat digunakan untuk menyusun solusi permasalahan, yaitu \textit{user-centered design} (UCD), \textit{activity-centered design}, \textit{systems design}, dan \textit{genius design} \parencite{saffer2010designing}.

\begin{enumerate}
  \item \textit{User-Centered Design} (UCD)
  \subitem Konsep utama dari UCD adalah mendesain seputar kebutuhan penggunanya. Seorang desainer perlu mendefinisikan tujuan utama dari produk yang dibuat seputar apa yang ingin dicapai oleh penggunanya. Seringkali pengguna pun dilibatkan dalam tahap-tahap pengembangan, seperti pembuatan konsep, pengumpulan data, serta proses pengujian. Hal ini bertujuan untuk menjauhkan produk akhir dari preferensi desainer dan mendekatkan pada preferensi penggunanya sendiri.
   
  \item \textit{Activity-Centered Design}
  \subitem Berbeda dari UCD, pendekatan dengan \textit{activity-centered design} akan memfokuskan seputar kegiatan tertentu. Kegiatan yang dimaksud dapat didefinisikan sebagai kumpulan tugas yang dilakukan untuk mencapai tujuan tertentu. \textit{Activity-centered design} mengharuskan desainer untuk membuat solusi di seputar kegiatan dan menopang kegiatan tersebut, melainkan tujuan dari kegiatan tersebut. Desainer juga harus membedakan maksud dari sebuah aktivitas dengan tujuannya, di mana mereka harus menitikberatkan fokus desain pada maksud dari aktivitas.
 
  \item \textit{Systems Design}
  \subitem \textit{Systems design} adalah pendekatan yang memfokuskan permasalahan pada keseluruhan sebuah sistem dalam proses desainnya. Sistem yang dimaksud bisa terdiri dari banyak komponen pendukung seperti manusia, perangkat keras dan lunak, mesin, dan objek lain sehingga permasalahan yang dihadapi cenderung lebih kompleks.
 
  \item \textit{Genius Design}
  \subitem Pendekatan dengan \textit{Genius Design} mengandalkan sepenuhnya terhadap pengalaman, keahlian, serta preferensi dari desainernya sendiri dalam membuat keputusan desain. Keterlibatan pengguna sangat jarang dan biasanya hanya ada untuk memvalidasi apakah desain yang diprediksi sesuai dengan yang desainer inginkan. Pendekatan ini dinilai memiliki resiko-resiko yang cukup besar namun terkadang dilakukan karena alasan yang lebih kuat daripada resiko tersebut.
 
\end{enumerate}

\subsection{\textit{User-Centered Design} (UCD)}
Seperti yang telah dijelaskan pada subsubbab \ref{subsec:pendekatan_id}, pendekatan UCD memusatkan perhatian proses desain pada pengguna. UCD sendiri merupakan turunan dari cabang ilmu HCI (\textit{Human-Computer Interaction}), yaitu metodologi rekayasa perangkat lunak bagi pengembang yang ditujukan agar perangkat lunak dapat memenuhi kebutuhan penggunanya \parencite{lowdermilk2013user}. Namun pendekatan ini tidak semerta-merta menanyakan langsung keingingan pengguna untuk produknya, karena hal ini dapat membuat produk yang dibuat bias ke pihak tertentu. UCD memiliki tahap dan panduan di mana seorang desainer atau ahli UCD akan mengidentifikasi profil dari penggunanya serta perilaku dan preferensi terhadap aspek-aspek sebuah produk. Informasi yang didapat akan kemudian digunakan dalam proses desain \parencite{10.1145/1621995.1621997}.

Dalam pendekatan UCD, seorang desainer juga tidak hanya membuat desain dengan tampilan antarmuka yang bagus. Desainer harus memastikan agar desain yang dibuatnya menyelesaikan permasalahan awal sesuai dengan riset yang telah dilakukan dan data yang telah diambil dari pengguna. Desainer bertanggung jawab untuk melakukan evaluasi dengan \textit{user} untuk memastikan desain yang telah dibuatnya tepat sasaran \parencite{lowdermilk2013user}.

Dalam penerapannya, pendekatan UCD harus mengikuti prinsip-prinsip tertentu. Menurut \textcite{iso9241-210:2010}, berikut adalah prinsip-prinsip standar yang perlu diikuti

\begin{enumerate}
  \item Desain dibuat berdasarkan pemahaman jelas atas pengguna, tugas-tugas, dan lingkungannya.
  \item Pengguna dilibatkan keseluruhan proses desain dan perkembangannya.
  \item Desain akan diubah dan diperbaiki secara terus-menerus sesuai dengan evaluasi dari pengguna.
  \item Proses UCD dilakukan secara berulang (iteratif).
  \item Desain meliputi keseluruhan \textit{user experience}.
  \item Pembuatan desain melibatkan berbagai perspektif dan kemampuan multidisipliner.
\end{enumerate}

\subsubsection{Proses-Proses dalam UCD}

Dalam melakukan perancangan dengan menerapkan pendekatan UCD, terdapat beberapa standar alur pekerjaan yang dapat diikuti. Salah satu yang umum digunakan adalah alur kerja standar ISO 9241-210. Berikut adalah alur kerja UCD sesuai dengan standar ISO 9241-210 yang tercantum pada Gambar II.2.

Terlihat pada diagram alur kerja pada Gambar \ref{fig:diagram_iso2}, terdapat beberapa kegiatan yang dilakukan secara iteratif. Kegiatan-kegiatan yang dilakukan secara iteratif tersebut merupakan komponen utama dalam kerangka UCD. Berikut adalah penjelasan mengenai kegiatan-kegiatan tersebut:

\begin{figure}[h]
  \centering
  \includegraphics[width=0.9\textwidth]{chapter-2-ucd-figure.png}
  \caption{Alur kerja \textit{User-Centered Design} (\parencite{iso9241-210:2010})}
  \label{fig:diagram_iso2}
\end{figure}
\FloatBarrier

\begin{enumerate}
  \item Memahami dan merincikan konteks penggunaan produk
  \subitem Pada tahap ini dilakukan pengumpulan dan analisa informasi mengenai konteks pemakaian produk. Hal ini bertujuan untuk mengungkapkan adanya kebutuhan, permasalahan, serta batasan dari produk yang penting untuk pengembangan solusi yang akan dibuat. Konteks pemakaian ini perlu mencakup informasi mengenai keseluruhan \textit{stakeholder}, karakteristik target pengguna, tujuan dan kegiatan dari pengguna, serta lingkungan di mana sistem akan dibuat atau dikembangkan. Jika UCD diterapkan pada produk yang sudah dibuat, maka beberapa informasi yang sudah tersedia dapat digunakan untuk melakukan modifikasi atau meningkatkan kualitas produk.
   
  \item Merincikan kebutuhan pengguna
  \subitem Pada tahap ini dilakukan identifikasi dan analisa lebih lanjut terhadap data-data yang telah dikumpulkan pada tahap sebelumnya. Dari hasil analisa akan didapatkan kebutuhan pengguna yang perlu dipenuhi dalam desain yang akan dibuat nanti. Kebutuhan pengguna yang didapat juga perlu mempertimbangkan batasan yang perlu diikuti sesuai dengan konteks pemakaian produk.
  
  \item Membuat solusi desain yang memenuhi kebutuhan pengguna
  \subitem Tahap selanjutnya adalah membuat prototipe desain sesuai dengan kebutuhan pengguna yang telah diidentifikasi pada fase sebelumnya. Prototipe desain yang dimaksud adalah produk implementasi desain aplikasi yang sudah menyerupai produk akhir, tanpa adanya implementasi unsur-unsur teknikal aplikasi tersebut. Hal ini bertujuan agar pengguna dapat memahami interaksi dan antarmuka dari desain aplikasi untuk dievaluasi pada tahap selanjutnya sebelum menjalani implementasi akhir.
  
  \item Mengevaluasi desain yang dibuat terhadap kebutuhan
  \subitem Proses evaluasi adalah proses untuk menentukan apakah desain aplikasi yang dibuat telah menyelesaikan permasalahan atau memenuhi kebutuhan yang diidentifikasi pada tahap-tahap sebelumnya. Proses evaluasi juga dilakukan untuk mengetahui apakah desain aplikasi sesuai dengan \textit{user experience goals} dan \textit{usability goals} yang diharapkan. Proses evaluasi tidak menutup kemungkinan adanya wawasan baru mengenai kebutuhan pengguna atau desain aplikasinya sendiri. Hasil dari evaluasi akan menentukan apakah desain tersebut layak dilanjutkan ke tahap implementasi atau diperlukan iterasi untuk menjalani perbaikan.

\end{enumerate}



\subsection{\textit{Usability Goals} dan \textit{User Experience Goals}}
\label{subsec:goals}
Untuk mendesain produk yang tepat bagi pengguna, seorang desainer perlu mengerti kebutuhan pengguna dengan menentukan tujuan yang jelas dari pengembangan produk interaktif tersebut. \textcite{PreeceRogersSharp15} menyebutkan bahwa untuk mencapainya, tujuan dapat diklasifikasikan sesuai dengan \textit{usability goals} dan \textit{user experience goals}.

% http://bpm.umg.ac.id/aset/images/download/M4-Standar-Rujuka-BA(1-8-2017).pdf


\textit{Usability goals} mengarahkan produk untuk mencapai kriteria \textit{usability} tertentu. \textit{Usability goals} mencakup bagaimana cara untuk mengoptimalisasi interaksi pengguna dengan produk untuk melakukan kegiatannya. \textit{Usability goals} dapat diuraikan menjadi 6 tujuan berikut
\begin{enumerate}
  \item Efektif untuk digunakan (\textit{effectiveness}) adalah tujuan yang menunjukkan apakah suatu produk sukses dalam menjalankan tugasnya.
  \item Efisien untuk digunakan (\textit{efficiency}) adalah tujuan yang menunjukkan bagaimana sebuah produk dapat membantu pengguna dalam mencapai tujuannya. Sebuah produk dapat dikatakan efisien jika penggunanya dapat melakukan suatu kegiatan dalam langkah-langkah yang sederhana dan tidak menuntut pengguna untuk mempelajari langkah-langkah tersebut terlalu lama.
  \item Aman untuk digunakan (\textit{safety}) adalah tujuan yang menunjukkan bagaimana sebuah produk dapat melindungi penggunanya dari situasi yang berbahaya atau tidak diinginkan, atau melakukan hal-hal yang bersifat destruktif. Tujuan ini dapat dicapai dengan meminimalisir resiko yang dapat ditemui pengguna atau memberikan opsi untuk membatalkan aksinya. Tujuan ini juga menunjukkan bagaimana sebuah produk membantu penggunanya mengeksplorasi produk secara percaya diri.
  \item Memiliki utilitas yang baik (\textit{utility}) adalah tujuan yang menunjukkan bagaimana sebuah produk menyediakan fungsionalitas yang baik untuk membantu pengguna melakukan hal yang dibutuhkan atau diinginkan.
  \item Mudah untuk dipelajari (\textit{learnability}) adalah tujuan yang menunjukkan seberapa mudah sebuah produk untuk dipelajari hingga pengguna dapat menggunakannya dengan benar.
  \item Mudah untuk mengingat penggunaan (\textit{memorability}) adalah tujuan yang menunjukkan seberapa mudah bagi pengguna untuk mengingat bagaimana cara menggunakan sebuah produk setelah mempelajarinya.
\end{enumerate}

\textit{User experience goals} lebih bersifat subjektif dibandingkan \textit{usability goals} karena mencakup berbagai jenis emosi dan pengalaman yang dirasakan oleh pengguna saat berinteraksi dengan produk. \textit{User experience goals} juga berhubungan erat dengan estetika dari produk. \textit{User experience goals} terdiri dari 2 jenis, yaitu tujuan yang diharapkan dan tujuan yang tidak diharapkan \parencite{PreeceRogersSharp15}. Pembagiannya dapat dilihat pada Tabel \ref{tab:ux_goals}.

\newpage

\RaggedLeft
\begin{footnotesize}
\begin{longtable}[c]{|>{\cbnormspacing}m{0.44\textwidth}|>{\cbnormspacing}m{0.49\textwidth}|}
  \caption{\textit{User experience goals} yang diharapkan dan tidak diharapkan}
  \label{tab:ux_goals} \\
  \hline \rowcolor[HTML]{A3E5F5}
  \centering\textbf{\textit{User experience goals} yang diharapkan} & \textbf{\textit{User experience goals} yang tidak diharapkan} \\ \hline \endfirsthead
  \hline \rowcolor[HTML]{A3E5F5}
  \centering\textbf{\textit{User experience goals} yang diharapkan} & \textbf{\textit{User experience goals} yang tidak diharapkan} \\ \hline \endhead
  
  \hline \endfoot
  
  1.	Satisfying              & 1.  Boring                  \\
  2.	Helpful                 & 2.  Unpleasant              \\
  3.	Fun                     & 3.  Frustrating             \\
  4.	Enjoyable               & 4.  Patronizing             \\
  5.	Motivating              & 5.  Making one feel guilty  \\
  6.	Provocative             & 6.  Making one feel stupid  \\
  7.	Engaging                & 7.  Annoying                \\
  8.	Challenging             & 8.  Cutesy                  \\
  9.	Surprising              & 9.  Childish                \\
  10.	Pleasurable             & 10. Gimmicky                \\
  11.	Enhancing sociability   &                             \\
  12.	Rewarding               &                             \\
  13.	Exciting                &                             \\
  14.	Supporting creativity   &                             \\
  15.	Emotionally fulfilling  &                             \\
  16.	Entertaining            &                             \\
  17.	Cognitively stimulating &                             \\
  18.	Experiencing flow       &                             \\
\end{longtable}
\end{footnotesize}
\justifying
\FloatBarrier


\subsection{Prinsip Desain Interaksi}
\label{subsec:prinsip_interaksi}
Prinsip desain interaksi adalah abstraksi umum untuk mengarahkan desainer dalam berpikir dari aspek-aspek yang berbeda dalam mendesain pengalaman pengguna. Prinsip desain dihasilkan dari gabungan antara pengetahuan berteori, pengalaman desainer, dan nalar wajar manusia. Prinsip desain ditujukan untuk membantu desainer dalam menjelaskan dan mengembangkan desain mereka secara konseptual, namun bukan sebagai cara mendesain sebuah antarmuka dari awal. Menurut \textcite{PreeceRogersSharp15}, terdapat banyak prinsip desain yang sudah disetujui secara umum, namun berikut adalah beberapa prinsip paling umum yang berkaitan dengan bagaimana pengguna melihat dan memanfaatkan suatu desain dalam mencapai tujuannya

\newpage

\begin{enumerate}
  \item \textit{Visibility}
  \subitem Visibilitas dari sebuah fitur mempengaruhi jumlah informasi yang diperlihatkan kepada pengguna. Tingginya tingkat visibilitas akan membantu pengguna dalam menentukan aksi yang dapat dilakukan. Posisi dari fitur terhadap bagaimana fitur tersebut digunakan juga dapat menentukan tingkat kemudahan untuk menggunakan fitur tersebut.
   
  \item \textit{Feedback}
  \subitem \textit{Feedback} atau umpan balik berhubungan dengan prinsip \textit{visibility}. Sebuah aksi layaknya dipasangi dengan sebuah reaksi untuk memberikan pengguna konfirmasi bahwa aksinya berhasil dilakukan atau tidak, membantunya dalam melanjutkan aktivitasnya. Informasi yang dikirimkan sebagai reaksi dapat bersifat taktil, verbal, visual, auditori, atau kombinasinya.
   
  \item \textit{Constraints}
  \subitem Sifat \textit{constraints} menunjuk pada interaksi apa saja yang bisa dilakukan pengguna di saat tersebut, tanpa menghilangkan aksi yang tidak bisa dilakukan. Umumnya, hal ini dapat dicapai dengan menonaktifkan sebuah opsi dalam kelompok pilihan. \textit{Constraints} juga membatasi persepsi pengguna dalam merepresentasikan informasi hanya dengan cara yang diharapkan desainer. Hal ini dapat dicapai dengan menunjukkan relasi yang jelas antarobjek.
   
  \item \textit{Consistency}
  \subitem Konsistensi dalam antarmuka menunjukkan bahwa fitur-fitur dengan elemen yang mirip akan melakukan aksi yang mirip juga atau menghasilkan reaksi yang mirip. Konsistensi akan meningkatkan tingkat kemudahan sebuah fitur untuk dipelajari, dengan cukup mempelajari satu fitur pengguna dapat mengerti fitur lain dengan elemen yang mirip, walaupun tiap fitur memiliki fungsinya sendiri. Contohnya adalah fitur di sebelah tombol \textit{toggle} menunjukkan bahwa pengaturannya bisa dinyalakan / dimatikan.
   
  \item \textit{Affordance}
  \subitem \textit{Affordance} adalah sifat dari objek yang menandakan cara penggunaannya. Contohnya, sebuah \textit{mouse} digunakan dengan ditekan dengan melihat bagaimana tombol \textit{mouse} itu didesain. Semakin tinggi tingkat \textit{affordance} atau semakin jelas bagaimana cara suatu objek digunakan, maka semakin mudah untuk menggunakannya atau belajar menggunakannya.
   
\end{enumerate}

\subsection{Tipe Interaksi}
\label{subsec:tipe_interaksi}

Dalam menyusun konsep desain, penentuan cara pengguna berinteraksi dengan produk / aplikasi akan didasarkan pada tipe interaksi dari desainnya. Menurut \textcite{PreeceRogersSharp15}, ada 5 tipe utama, yaitu \textit{instructing}, \textit{conversing}, \textit{manipulating}, \textit{exploring}, dan \textit{responding}. Menentukan tipe interaksi dari desain dapat membantu untuk menyusun model konseptual sebelum menentukan tipe antarmuka dari desainnya. Suatu sistem tidak terkekang dalam memiliki satu jenis tipe interaksi, pengguna dapat merasakan pengalaman yang berbeda di bagian-bagian sistem dengan tipe interaksinya masing-masing. Berikut adalah penjelasan tentang tipe-tipe interaksi.

\begin{enumerate}
  \item \textit{Instructing}
  \subitem Pada desain yang menerapkan tipe interaksi \textit{instructing} pengguna akan melakukan aktivitasnya dengan cara memberikan instruksi kepada sistem. Pengguna dapat memberikan instruksi dengan mengetikan perintah, memilih opsi dari menu, mengatakan perintahnya ke mikrofon, memberikan gestur, atau sesederhana menekan satu atau kombinasi tombol. Keuntungan dalam memilih tipe interaksi ini adalah kecepatan dan efisiensi interaksi, cocok untuk aksi yang perlu dilakukan berulang kali. Contoh kasus yang menggunakan tipe interaksi \textit{instructing} adalah sebuah \textit{vending machine}, di mana konsumen memilih makanan yang ingin dibeli baik dengan mengetikkan nomor label makanan.

  \item \textit{Conversing}
  \subitem Tipe interaksi \textit{conversing} berdasar pada konsep di mana pengguna akan melakukan percakapan dengan sistem. Sistem didesain untuk membalas pengguna sebagaimana manusia biasa akan menjawab, berbeda dengan tipe interaksi \textit{instructing} di mana sistem hanya menuruti perintah yang diberikan. Pengguna mampu mendapatkan saran, jawaban lebih pribadi, atau berdiskusi dengan sistem. Pada umumnya, sistem akan memanfaatkan teknologi \textit{voice-recognition}, AI, atau teknologi berbasis \textit{natural-language} lainnya dalam menerima input, dan menggunakan pembangkitan suara atau cukup kalimat tertulis untuk memberikan balasan. Tipe interaksi ini dapat ditemukan pada sistem penasehat, fasilitas bantuan, serta \textit{chatbot}.
  
  \item \textit{Manipulating}
  \subitem Tipe interaksi \textit{manipulating} melibatkan tindakan memanipulasi sebuah objek layaknya di dunia nyata, seperti menggerakan (\textit{dragging}), membuka, dan menutup. Pengguna juga dapat melakukan aksi yang mungkin tidak dapat dilakukan di dunia nyata, seperti \textit{stretching}, \textit{shrinking}, memperbesar, dan memperkecil objek. Tujuan yang ingin dicapai adalah untuk memberikan pengguna perasaan senyata mungkin bahwa mereka sedang beinteraksi langsung dengan objek digital. Prinsip dari tipe interaksi ini adalah objek yang ada di layar tetap tampak selama pengguna memanipulasinya, dan memberikan \textit{feedback} yang langsung ketika dimanipulasi untuk mempertahankan perasaan nyatanya. Walaupun tipe interaksi ini dapat membuat pengguna mudah berinteraksi dengan objek, terdapat beberapa aksi yang sebaiknya menggunakan tipe interaksi lain seperti \textit{instructing}. Contohnya adalah ketika memperbaiki \textit{typo} yang berulang di suatu dokumen, melainkan mengubahnya dengan mencari dan menggantinya satu per satu, lebih baik untuk memberikan instruksi kepada sistem untuk mencari \textit{typo} tersebut di seluruh dokumen dan langsung mengganti semuanya.
  
  \item \textit{Exploring}
  \subitem Sistem yang menggunakan tipe interaksi \textit{exploring} melibatkan penggunanya untuk bergerak di sebuah lingkungan baik virtual maupun fisik. Lingkungan sistem dapat mendeteksi ketika pengguna melakukan aksi tertentu dan memicu suatu peristiwa yang dapat dirasakan dan diinteraksi oleh pengguna. Lingkungan sistem tidak selalu harus berbentuk dunia virtual 3D, tetapi kebanyakan sistem yang mengadopsi tipe interaksi \textit{exploring} memilikinya, seperti eksibisi virtual yang diakses menggunakan perangkat VR atau sebuah \textit{game}.
  
  \item \textit{Responding}
  \subitem Sistem dengan tipe interaksi \textit{responding} akan menginisiasi interaksinya dengan memberikan, menunjukkan, atau mendeskripsikan sesuatu kepada pengguna, dan menunggu sebuah tanggapan. Interaksi yang diberikan sistem bergantung pada konteks yang sedang dialami pengguna atau dideteksi sistem.  Contohnya sebuah aplikasi dapat memberikan rekomendasi restoran yang berada di dekat pengguna ketika melewati suatu jalan, atau sebuah \textit{smartwatch} memberi notifikasi bahwa penggunanya sudah berhasil berjalan sebanyak 10.000 langkah. Tantangan desain yang menggunakan tipe interaksi ini adalah mengetahui kapan pengguna merasa informasi yang diberikan akan merasa berguna, dan menghindari interaksi yang dirasa dapat mengganggu pengguna.

\end{enumerate}


% * =======================================================================
% *   ||  ||  ||  ||  ||  ||  ||  ||  ||  ||  ||  ||  ||  ||  ||  ||  ||
% * =======================================================================



\section{Penelitian Terkait}
\label{sec:penelitian_terkait}

Untuk membantu menyusun Tugas Akhir ini, digunakan beberapa penelitian yang pernah dilakukan sebagai sumber referensi. Referensi berikut digunakan sebagai pertimbangan dalam penyusunan desain solusi, atau untuk mengerti konteks dari tema yang dibahas.

\subsection{Socialize}

Socialize adalah sebuah aplikasi yang dibuat untuk membantu penelitian tentang aplikasi yang menerapkan konsep-konsep dari Digital Wellbeing. Dalam penelitian yang berjudul “The Race Towards Digital Wellbeing: Issues and Opportunities”, \textcite{CHI2019SOCIALIZE} meneliti aplikasi-aplikasi berbeda yang berkaitan dengan konsep Digital Wellbeing untuk menemukan fitur-fitur yang umum ditemukan.

Dari 42 aplikasi yang diteliti, ditemukan dua fungsi paling utama dari aplikasi berkonsep Digital Wellbeing adalah melacak dan memvisualisasi data penggunaan serta melakukan intervensi untuk mengurangi ketergantungan. Dalam upaya pelacakan data, 57\% aplikasi memiliki fitur menampilkan semacam visualisasi ringkasan dari penggunaan \textit{smartphone} pengguna, dan 50\% dapat menampilkan penggunaan per aplikasi. Dalam menampilkan ringkasan, aplikasi biasanya menggunakan grafik untuk memberikan visualisasi data (60\% aplikasi) atau \textit{widgets} yang lebih mudah diakses pengguna dari homescreen \textit{smartphone} (38\%).

Untuk melakukan intervensi, aplikasi-aplikasi menyediakan fitur untuk memberikan batas waktu penggunaan pada aplikasi atau disebut app timer (31\%), fitur yang memblokir akses ke aplikasi (26\%) atau masuknya notifikasi (19\%), fitur yang membatasi waktu penggunaan \textit{smartphone} (26\%) atau memblokirnya secara keseluruhan (15\%), atau fitur untuk beristirahat dari \textit{smartphone}-nya secara sementara (15\%). Sebagai fitur tambahan, beberapa aplikasi membantu pengguna dengan memberikan kutipan motivasi (12\%) atau suatu jenis penghargaan jika berhasil menyelesaikan suatu tantangan terkait Digital Wellbeing. Informasi lebih lengkap dari penemuan Roffarello dan De Russis dapat dilihat pada Gambar \ref{img:ruf_table} dan Gambar \ref{img:ruf_chart}.

\begin{figure}[h]
  \centering
  \includegraphics[width=0.5\textwidth]{chapter-2-ruf_table.png}
  \caption{Fitur-fitur umum pada aplikasi-aplikasi berkonsep Digital Wellbeing \parencite{CHI2019SOCIALIZE}}
  \label{img:ruf_table}
\end{figure}

\begin{figure}[h]
  \centering
  \includegraphics[width=0.7\textwidth]{chapter-2-ruf_chart.png}
  \caption{Distribusi fitur pada aplikasi-aplikasi berkonsep Digital Wellbeing \parencite{CHI2019SOCIALIZE}}
  \label{img:ruf_chart}
\end{figure}

Setelah melakukan penelitian terhadap fitur-fitur aplikasi, dilakukan pula analisis tematik terhadap 1.128 ulasan pengguna untuk aplikasi-aplikasi tersebut yang mencakup ulasan antara tahun 2015 dan 2018, baik ulasan positif maupun negatif. Hal ini dilakukan untuk mengetahui fitur apa saja yang disukai atau tidak oleh penggunanya. Roffarello dan De Russis menemukan bahwa banyak pengguna yang memang menyukai ide dari konsep Digital Wellbeing untuk memonitor dan memperbaiki kebiasaan penggunaan \textit{smartphone}, terbantu dengan mudahnya menggunakan aplikasinya. Pengguna menyukai banyaknya fitur yang tersedia, seperti pembatasan waktu dan pemblokiran akses, pelacakan dan statistik data, serta pemberian motivasi melalui penghargaan dan kutipan motivasi, dapat membantu memperbaiki kebiasaan penggunaan \textit{smartphone} atau kebiasaan lainnya secara efektif. Tak hanya itu, pengguna juga menemukan bahwa aplikasi sangat berguna untuk kasus penggunaan seperti pembatasan penggunaan \textit{smartphone} di saat belajar atau bekerja, pemantauan dan pengendalian perangkat milik anak kecil oleh orang tua, serta bantuan untuk memperbaiki jadwal tidur.

Di sisi lain, banyak pengguna yang mengeluhkan aplikasi-aplikasi kurang bersifat membatasi dan mudah untuk dikelabui, membuatnya tidak efektif dalam membatasi orang-orang yang kecanduan. Ada juga pengguna yang mengkhawatirkan tentang privasinya selama menggunakan fitur pelacakan, menilai aplikasi terasa intrusif. Selain itu, banyak ulasan pengguna yang mengeluhkan bug dan kecacatan desain hingga membuat aplikasi tidak berguna. Seluruh penemuan dirangkum menjadi kumpulan kata kunci yang dapat dilihat pada Gambar \ref{img:ruf_review}.

\begin{figure}[h]
  \centering
  \includegraphics[width=0.5\textwidth]{chapter-2-ruf_review.png}
  \caption{Kata kunci pada analisis ulasan aplikasi-aplikasi berkonsep Digital Wellbeing \parencite{CHI2019SOCIALIZE}}
  \label{img:ruf_review}
\end{figure}

Dari analisis fitur umum dan ulasan, Roffarello dan De Russis membuat sebuah aplikasi Android bernama Socialize yang mencakup fitur-fitur paling umum dari aplikasi-aplikasi tersebut, yaitu fitur yang muncul pada setidaknya 15\% dari aplikasi-aplikasi yang dieksplorasi. Fitur-fitur yang diimplementasi dapat ditemukan pada Gambar \ref{img:ruf_fitur}, sedangkan tampilannya dapat dilihat pada Gambar \ref{img:ruf_app}. Lalu dilakukan pengujian aplikasi Socialize terhadap 38 orang untuk membandingkan penggunaan \textit{smartphone} sebelum dan sesudah memakai aplikasi. Secara kualitatif, ditemukan bahwa seluruh partisipan bersedia untuk menggunakan aplikasi Socialize lagi jika dirilis menjadi aplikasi sesungguhnya, dan memberikan masukan konstruktif tentang fitur-fiturnya. Beberapa partisipan juga antusias saat melihat statistik penggunaan \textit{smartphone}-nya. Partisipan juga mampu meningkatkan kualitas dari kebiasaan penggunaan \textit{smartphone}-nya dengan menggunakan fitur pembatasan dari Socialize. Namun di sisi lain, beberapa pengguna mengeluhkan bahwa Socialize membuat \textit{smartphone} mereka lebih boros dalam penggunaan baterai, atau mengganggu performa dari \textit{smartphone} mereka.

\begin{figure}[h]
  \centering
  \includegraphics[width=0.55\textwidth]{chapter-2-ruf_fitur.png}
  \caption{Fitur-fitur yang diimplementasi pada aplikasi Socialize \parencite{CHI2019SOCIALIZE}}
  \label{img:ruf_fitur}
\end{figure}

\begin{figure}[h]
  \centering
  \includegraphics[width=\textwidth]{chapter-2-ruf_app.png}
  \caption{Tampilan aplikasi Socialize \parencite{CHI2019SOCIALIZE}}
  \label{img:ruf_app}
\end{figure}

Secara keseluruhan pada penelitian aplikasi Socialize, pengguna menyukai ide dari sebuah aplikasi Digital Wellbeing untuk membantu meningkatkan kualitas dari kebiasaan digitalnya. Namun solusi yang diberikan terkadang tidak cukup untuk menyelesaikan masalah yang ada. Masih perlu dibutuhkan tekad dari penggunanya untuk memanfaatkan fitur-fitur yang ada. Adapun beberapa masukan dari pengguna seperti kurang ketatnya pembatasan yang diberikan, dan diperlukannya interaksi sosial untuk membantu pengguna dalam mendapat bayangan dari kebiasaan digital yang baik. Penemuan yang cukup besar juga mengungkapkan bahwa aplikasi-aplikasi berkonsep Digital Wellbeing hanya membantu penggunanya dalam melakukan pemantauan mandiri dan menghentikan kebiasaan lama, tanpa membantu penggunanya mengembangkan kebiasaan baru yang lebih baik. Fitur yang dapat membantu hal tersebut, seperti pemberian kutipan motivasi dan penghargaan serta adanya interaksi sosial antarpengguna, cukup jarang ditemukan dari aplikasi-aplikasi Digital Wellbeing yang ada.

% * =======================================================================
% *   ||  ||  ||  ||  ||  ||  ||  ||  ||  ||  ||  ||  ||  ||  ||  ||  ||
% * =======================================================================

\section{\textit{Usability Testing}}
\textit{Usability Testing} adalah sebuah proses pengujian produk yang sedang dikembangkan untuk mengetahui apakah produk tersebut dapat digunakan dengan benar oleh pengguna yang terpilih, serta untuk menguji kepuasan pengguna dalam menggunakan produk. Dengan melakukan \textit{usability testing} di sebuah lingkungan yang terkendali, desainer atau peneliti dapat mengendalikan pengaruh lingkungan yang dapat mempengaruhi performa pengguna. \parencite{PreeceRogersSharp15} Menurut \textcite{nielsengrouptesting}, terdapat 3 elemen penting dalam sebuah \textit{usability testing}, yaitu fasilitator, \textit{tasks}, dan partisipan. Fasilitator adalah pihak yang menuntun partisipan dalam pengujian dengan memberikan \textit{tasks}, mengobservasi perilaku, serta menerima masukan. \textit{Tasks} adalah kegiatan yang menyerupai skenario realistis yang dilakukan untuk menguji produk tersebut. Susunan kata yang tepat sangat penting untuk menghindari kebingungan atau hasil yang salah saat mengerjakan \textit{tasks}. Partisipan adalah seorang \textit{user} atau calon \textit{pengguna produk} yang sesuai dengan target pengguna produk yang diuji. 

\subsection{Tipe \textit{Usability Testing}}
Berdasarkan data yang diambil, \textit{usability testing} dapat dibagi menjadi 2 tipe, yaitu kualitatif dan kuantitatif. Berikut adalah penjelasannya menurut \textcite{nielsengrouptesting}

\begin{enumerate}
  \item Kualitatif
  \subitem Pengujian kualitatif berfokus dalam mengumpulkan wawasan tentang bagaimana \textit{user} menggunakan produk atau jasanya. Pengujian kualitatif adalah tipe pengujian paling tepat dalam menemukan masalah dari pengalaman pengguna.

  \item Kuantitatif
  \subitem Pengujian kuantitatif berfokus dalam mengumpulkan metrik-metrik yang dapat menjelaskan tentang pengalaman pengguna. Pengujian kuantitatif adalah tipe pengujian paling tepat untuk mengumpulkan tolok ukur. 
\end{enumerate}


\subsection{Metrik Pengujian}
Dalam pengujian kuantitatif, dikumpulkan metrik-metrik pengujian untuk mengukur ketercapaian \textit{usability goals} dan \textit{user experience goals} yang telah ditentukan. Berikut adalah beberapa metrik pengujian yang digunakan dalam pengerjaan tugas akhir


\subsubsection{\textit{Likert scale}}
\label{subsubsec:likert_scale}
\textit{Likert scale} adalah sebuah teknik psikometrik untuk mengukur sikap seseorang terhadap suatu kondisi atau permasalahan. Skala ini mengukur tingkat persetujuan partisipan tentang kondisi tersebut, dari sangat tidak setuju hingga sangat setuju, dalam bentuk skala metrik. \parencite{likert2015joshi} Skala metrik yang digunakan terdapat 2 jenis yaitu \textit{5-point scale} dan \textit{7-point scale}. Banyak kuesioner mengadopsi teknik ini dengan memberikan beberapa pernyataan untuk sikap spesifik terkait suatu masalah agar kumpulan pernyataan tersebut dapat terikat satu sama lain. Metrik pengukuran seperti \textit{Single Ease Question}, \textit{System Usability Scale}, dan \textit{Intrinsic Motivation Inventory} mengadopsi \textit{likert scale} dalam kuesionernya. 

\subsubsection{\textit{System Usability Scale} (SUS)}
\label{subsubsec:sus}
\textit{System Usability Scale} atau SUS adalah metrik pengujian yang digunakan untuk mengukur \textit{usability} dari sebuah produk atau sistem secara menyeluruh \parencite{sus1995brooke}. Metrik SUS berupa \textit{post-test questionnaire}, diberikan kepada partisipan di akhir \textit{usability testing}. Metrik SUS berisi kuesioner yang terdiri dari 10 pertanyaan dengan jawaban \textit{5-point likert scale}, mulai dari sangat tidak setuju untuk nilai 1 dan sangat setuju untuk nilai 5. Nilai akhir dari SUS adalah skala 0 hingga 100. Menurut \textcite{sus2008bangor}, setelah dilakukan pengujian kepada 500 penelitian ditemukan bahwa penilaian dengan metrik SUS dapat dibagi menjadi beberapa persentil, dengan nilai di atas 68 dapat disebut sebagai di atas rata-rata. Pengukuran untuk menentukan apakah \textit{usability} sebuah produk dapat dibilang baik atau belum mengacu kepada skala SUS yang dapat dilihat pada Gambar \ref{img:sus_scale}.

\begin{figure}[h]
  \centering
  \includegraphics[width=0.8\textwidth]{chapter-2-sus_scale.jpg}
  \caption{Skala persentil pengukuran \textit{System Usability Scale} \parencite{sus2008bangor}}
  \label{img:sus_scale}
\end{figure}
\FloatBarrier

\subsubsection{\textit{Single Ease Question} (SEQ)}
\label{subsubsec:seq}
\textit{Single Ease Question} atau SEQ adalah metrik pengujian yang digunakan untuk mengukur tingkat kemudahan dari sebuah \textit{task} yang dikerjakan partisipan. Metrik SEQ berupa \textit{post-task questionnaire}, diberikan kepada partisipan setelah mengerjakan sebuah \textit{task}. Metrik SEQ menggunakan satu buah pertanyaan dengan jawaban \textit{7-point likert scale}, mulai dari sangat tidak setuju untuk nilai 1 dan sangat setuju untuk nilai 7. Pengukuran dengan SEQ selalu menggunakan satu buah pertanyaan yang sama yaitu tentang bagaimana tingkat kemudahan dari task yang dikerjakan. Menurut \textcite{seq2012sauro}, penelitian terhadap 400 \textit{task} yang diberikan kepada 10.000 partisipan menemukan bahwa nilai rata-rata dari pengukuran SEQ berkisar antara 5.3 sampai 5.6. \textcite{seq2018sauro} menambahkan, pengukuran dengan SEQ dapat dikorelasikan dengan \textit{completion rate} dari \textit{task}, semakin tinggi SEQ maka semakin tinggi \textit{completion rate}-nya. Adapaun korelasi antara penilaian SEQ dengan waktu pengerjaan dari \textit{task}, semakin tinggi SEQ maka semakin cepat waktu pengerjaannya, namun hal ini perlu dikaitkan dengan konteks dari \textit{task} yang dikerjakan.

\subsubsection{\textit{Intrinsic Motivation Inventory} (IMI)}
\label{subsubsec:imi}
\textit{Intrinsic Motivation Inventory} adalah sebuah skala multidimensional yang digunakan untuk mengukur pengalaman subjektif partisipan terkait aktivitas yang diteliti. \parencite{imisdtorg} IMI pertama dibuat oleh \textcite{RYANDECI2000SDT} sebagai bagian dari penelitian terhadap motivasi intrinsik manusia dibantu dengan \textit{self-determination theory}.

Pengukuran dengan IMI dilakukan melalui 7 subskala, yaitu \textit{interest/enjoyment}, \textit{perceived competence}, \textit{effort}, \textit{value/usefulness}, \textit{felt pressure and tension}, \textit{perceived choice}, dan \textit{relatedness}. Masing-masing subskala terdiri dari sejumlah pertanyaan dengan \textit{7-point likert scale} sebagai jawabannya, mulai dari sangat tidak setuju untuk nilai 1 hingga sangat setuju untuk nilai 7. Tabel \ref{tab:imi_questions} berisi daftar pertanyaan untuk setiap subskala dalam bahasa Inggris. Penafsiran hasil dari sebuah subskala IMI perlu melakukan langkah-langkah berikut

\begin{enumerate}
  \item Nilai pada pertanyaan \textit{reverse}, yang ditandai dengan huruf (R), diambil dari mengurangi angka 8 dengan jawaban pertanyaan. Selain itu, nilai diambil apa adanya sesuai jawaban pertanyaan.
  \item Nilai subskala didapat dari hasil rata-rata seluruh nilai dari pertanyaan.
  \item Semakin tinggi nilai yang didapat, maka semakin cocok pengalaman yang dirasakan partisipan dengan subskala tersebut.
\end{enumerate}

\RaggedLeft
\begin{footnotesize}
\begin{longtable}[c]{|W{c}{0.04\textwidth}|p{0.8\textwidth}|}
  \caption{Daftar Pertanyaan \textit{Intrinsic Motivation Inventory} \parencite{imisdtorg}}
  \label{tab:imi_questions} \\
  \hline \rowcolor[HTML]{A3E5F5} \textbf{No.} & \textbf{Pertanyaan} \\ \hline \endfirsthead
  \hline \rowcolor[HTML]{A3E5F5} \textbf{No.} & \textbf{Pertanyaan} \\ \hline \endhead

  \hline \endfoot
  
  \rowcolor[HTML]{DCF3FC} \multicolumn{2}{|l|}{\textbf{Subskala \textit{Interest/Enjoyment}}} \\ \hline
  1 & I enjoyed doing this activity very much \\ \hline
  2 & This activity was fun to do. \\ \hline
  3 & I thought this was a boring activity. (R) \\ \hline
  4 & This activity did not hold my attention at all. (R) \\ \hline
  5 & I would describe this activity as very interesting. \\ \hline
  6 & I thought this activity was quite enjoyable. \\ \hline
  7 & While I was doing this activity, I was thinking about how much I  enjoyed it. \\ \hline
  
  \rowcolor[HTML]{DCF3FC} \multicolumn{2}{|l|}{\textbf{Subskala \textit{Perceived Competence}}} \\ \hline
  1 & I think I am pretty good at this activity. \\ \hline
  2 & I think I did pretty well at this activity, compared to other students. \\ \hline
  3 & After working at this activity for awhile, I felt pretty competent. \\ \hline
  4 & I am satisfied with my performance at this task. \\ \hline
  5 & I was pretty skilled at this activity. \\ \hline
  6 & This was an activity that I couldn't do very well. (R) \\ \hline
  
  \rowcolor[HTML]{DCF3FC} \multicolumn{2}{|l|}{\textbf{Subskala \textit{Effort/Importance}}} \\ \hline
  1 & I put a lot of effort into this. \\ \hline
  2 & I didn't try very hard to do well at this activity. (R) \\ \hline
  3 & I tried very hard on this activity. \\ \hline
  4 & It was important to me to do well at this task. \\ \hline
  5 & I didn't put much energy into this. (R) \\ \hline
  
  \rowcolor[HTML]{DCF3FC} \multicolumn{2}{|l|}{\textbf{Subskala \textit{Pressure Tension}}} \\ \hline
  1 & I did not feel nervous at all while doing this.   (R) \\ \hline
  2 & I felt very tense while doing this activity. \\ \hline
  3 & I was very relaxed in doing these. (R) \\ \hline
  4 & I was anxious while working on this task. \\ \hline
  5 & I felt pressured while doing these. \\ \hline
  
  \rowcolor[HTML]{DCF3FC} \multicolumn{2}{|l|}{\textbf{Subskala \textit{Perceived Choice}}} \\ \hline
  1 & I believe I had some choice about doing this activity. \\ \hline
  2 & I felt like it was not my own choice to do this task. (R) \\ \hline
  3 & I didn't really have a choice about doing this task. (R) \\ \hline
  4 & I felt like I had to do this. (R) \\ \hline
  5 & I did this activity because I had no choice. (R) \\ \hline
  6 & I did this activity because I wanted to. \\ \hline
  7 & I did this activity because I had to. (R) \\ \hline
   
  \rowcolor[HTML]{DCF3FC} \multicolumn{2}{|l|}{\textbf{Subskala \textit{Value/Usefulness}}} \\ \hline
  1 & I believe this activity could be of some value to me. \\ \hline
  2 & I think that doing this activity is useful for ... \\ \hline
  3 & I think this is important to do because it can ... \\ \hline
  4 & I would be willing to do this again because it has some value to me. \\ \hline
  5 & I think doing this activity could help me to ... \\ \hline
  6 & I believe doing this activity could be beneficial to me. \\ \hline
  7 & I think this is an important activity. \\ \hline

  \rowcolor[HTML]{DCF3FC} \multicolumn{2}{|l|}{\textbf{Subskala \textit{Relatedness}}} \\ \hline
  1 & I felt really distant to this person. (R) \\ \hline
  2 & I really doubt that this person and I would ever be friends. (R) \\ \hline
  3 & I felt like  I could really trust this person. \\ \hline
  4 & I'd like a chance to interact with this person more often. \\ \hline
  5 & I'd really prefer not to interact with this person in the future. (R) \\ \hline
  6 & I don't feel like I could really trust this person. (R) \\ \hline
  7 & It is likely that this person and I could become friends if we interacted a lot. \\ \hline
  8 & I feel close to this person. \\ \hline

\end{longtable}
\end{footnotesize}
\justifying
\FloatBarrier

% Pengukuran dengan IMI dilakukan melalui 6 subskala, yaitu \textit{interest/enjoyment}, \textit{perceived competence}, \textit{effort}, \textit{value/usefulness}, \textit{felt pressure and tension}, dan \textit{perceived choice}, serta satu subskala tambahan, \textit{relatedness}, yang keabsahannya belum sepenuhnya diuji. Subskala \textit{interest/enjoyment} khusus menilai motivasi intrinsik partisipan, walaupun keseluruhan kuesioner dinamai \textit{Intrinsic Motivation Inventory}. Masing-masing subskala terdiri dari sejumlah pertanyaan yang menggunakan \textit{7-point likert scale}. 


% * =======================================================================
% *   ||  ||  ||  ||  ||  ||  ||  ||  ||  ||  ||  ||  ||  ||  ||  ||  ||
% * =======================================================================



\chapter{Analisis Masalah dan Rancangan Solusi}

% Pada bab Analisis Masalah dan Rancangan Solusi akan diuraikan tentang identifikasi permasalahan yang menjadi dasar dari tugas akhir ini, serta analisis dan rancangan solusi yang ingin diajukan untuk menyelesaikan permasalahan tersebut. Secara garis besar, proses perancangan prototipe aplikasi akan menggunakan pendekatan \textit{user-centered design} (UCD).

% Pada bab ini akan dilakukan analisis permasalahan yang menjadi dasar dari tugas akhir ini, analisis solusi yang ingin diajukan, serta rencana pengerjaan selanjutnya untuk Tugas Akhir II. Secara garis besar, proses perancangan prototipe aplikasi akan menggunakan pendekatan \textit{user-centered design} (UCD).

\section{Analisis Masalah}
\label{sec:analisis_masalah}

Sebagaimana yang telah dibahas pada subbab \ref{sec:latarbelakang}, aplikasi Digital Wellbeing yang terdapat pada smartphone berbasis Android memiliki beberapa masalah pada desain interaksinya. Pada dasarnya, inti masalah dari desain interaksi aplikasi Digital Wellbeing adalah kurang efektif dalam menghambat pengguna mengakses aplikasi-aplikasi yang menurunkan tingkat produktivitas pengguna, serta kurang efektif dalam mempromosikan kebiasaan-kebiasaan yang lebih baik dalam penggunaan gawai dalam kehidupan sehari-hari. Daftar permasalahan desain interaksi yang terdapat pada aplikasi Digital Wellbeing dicantumkan pada Tabel \ref{tab:daftar_permasalahan}. Rincian mengenai permasalahan dapat dilihat pada Lampiran \ref{chpt:rincian_analisis_permasalahan}.

\begin{table}[ht]
  \centering
  \fontsize{10}{12}
  \caption{Daftar Permasalahan}
  \label{tab:daftar_permasalahan}
  \vspace{0.2cm}
  \begin{tabular}{|p{0.12\textwidth}|p{0.85\textwidth}|}
  \hline
  Kode  & Permasalahan                                                                                                                    \\ \hline
  M-001 & Fitur "Take a break" dari Focus Mode dapat disalahgunakan untuk mengambil istirahat terus menerus                               \\ \hline
  M-002 & Fitur "Turn off for now" dari Focus Mode dapat disalahgunakan untuk mematikan Focus Mode sebelum tenggat waktu yang ditentukan  \\ \hline
  M-003 & Kurangnya elemen yang memotivasi pengguna dalam memperbaiki pola penggunaan aplikasi                                            \\ \hline
  \end{tabular}
\end{table}


\section{Analisis Solusi}
\label{sec:analisis_solusi}

Dari permasalahan yang telah diuraikan pada subbab \ref{sec:analisis_masalah}, terdapat beberapa solusi yang dapat diimplementasikan ke dalam aplikasi pencegah distraksi untuk mencapai tujuan dari aplikasi Digital Wellbeing dengan lebih baik yaitu menghambat akses pengguna terhadap aplikasi distraksi secara efektif dan memotivasi pengguna untuk mengubah pola penggunaan aplikasi secara general. Setiap solusi akan dijelaskan tentang permasalahan apa yang akan diselesaikan. Solusi yang ditawarkan untuk menyelesaikan masalah yang telah dianalisis dicantumkan pada Tabel \ref{tab:daftar_solusi}. Rincian mengenai solusi tersebut dapat dilihat pada Lampiran \ref{chpt:rincian_analisis_solusi}.


\begin{table}[h]
  \centering
  \fontsize{10}{12}
  \caption{Daftar Solusi}
  \label{tab:daftar_solusi}
  \vspace{0.2cm}
  \begin{tabular}{|p{0.15\textwidth}|p{0.18\textwidth}|p{0.6\textwidth}|}
  \hline
  Kode Solusi & Kode Masalah & Solusi \\ \hline
  S-002 & M-001 & Memberikan langkah tambahan setiap kali pengguna mengakses fungsionalitas "Take a break" \\ \hline
  S-001 & M-002 & Memindahkan fungsionalitas dari tombol "Turn off for now" ke pengaturan aplikasi \\ \hline
  S-003 & M-003 & Memberikan sugesti atas langkah yang dapat diambil untuk meningkatkan kualitas pola penggunaan aplikasi \\ \hline
  \end{tabular}
\end{table}



% \blindtext

\section{Rencana Pengerjaan Solusi}

% Ketiga solusi yang telah diuraikan pada subbab \ref{sec:analisis_solusi} akan diimplementasikan dalam prototipe aplikasi, beserta fitur-fitur lain pada Digital Wellbeing yang akan mendukung solusi tersebut. Prototipe aplikasi ini akan diimpementasikan pada \textit{platform} Android. Secara garis besar, proses perancangan prototipe aplikasi akan menggunakan pendekatan \textit{user-centered design} (UCD).

Seperti yang telah disebutkan pada subbab \ref{sec:metodologi}, metodologi yang digunakan dalam pengerjaan Tugas Akhir ini akan menggunakan pendekatan UCD. Dengan maksud mengikuti prosesnya, maka langkah selanjutnya yang akan dilakukan adalah mengumpulkan data. Pengumpulan data akan dilakukan dengan menyebarkan form secara online serta melakukan wawancara dengan responden yang bersedia untuk bekerja sama lebih lanjut. Proses ini akan dilaksanakan pada periode pengerjaan Tugas Akhir 2. Pengumpulan data ini bertujuan untuk melakukan validasi terhadap permasalahan yang sudah dianalisis, dan juga tidak menutup kemungkinan untuk menemukan permasalahan desain interaksi lain dari masukan pengguna.

Setelah melakukan pengumpulan data, akan dilakukan analisis terhadap masukan yang didapat untuk mejadi kebutuhan perangkat lunak. Hasil analisis juga akan memvalidasi analisis masalah dan solusi yang didapat dari observasi penulis pada subbab \ref{sec:analisis_masalah} dan \ref{sec:analisis_solusi}.

Kebutuhan perangkat lunak yang telah disusun akan diimplementasi dalam bentuk prototipe \textit{low-fidelity} terlebih dahulu. Setelah dilakukan evaluasi, maka implementasi akan dilanjutkan dalam bentuk prototipe \textit{high-fidelity}. Setelah menjalani evaluasi, maka perancangan prototipe aplikasi akan dikerjakan. Prototipe aplikasi diharapkan akan menghasilkan data dengan kualitas yang lebih tinggi pada saat evaluasi dibandingkan saat menggunakan prototipe \textit{low-fidelity} atau \textit{high-fidelity}. Hasil evaluasi juga akan menentukan apakah aplikasi akan menjalani proses iterasi atau diimplementasi lebih lanjut.

% \blindtext
\chapter{Implementasi dan Pengujian Prototipe}

Bab Implementasi dan Pengujian Prototipe berisi tentang lanjutan dari metodologi \textit{User-Centered Design}, yaitu tahap perancangan serta evaluasi prototipe perangkat lunak. Kedua tahap tersebut dilakukan iterasi sesuai dengan kebutuhan. Perancangan dan evaluasi akan dilakukan pada \textit{low-fidelity prototype} berbentuk \textit{wireframe} dan \textit{high-fidelity prototype} berbentuk prototipe \textit{mobile application} pada perangkat Android. Bagian perancangan prototipe menjelaskan tentang proses implementasi dari prototipe, sedangkan bagian evaluasi akan menjelaskan tentang proses pengujian prototipe yang berisi skenario dan hasil pengujian yang telah dicocokan dengan \textit{usability goals} dan \textit{user experience goals} yang sudah ditentukan sebelumnya. 

\section{Pengembangan Prototipe \textit{Low-Fidelity}}
\blindtext

\section{Pengujian Prototipe \textit{Low-Fidelity}}
\blindtext

\section{Pengembangan Prototipe \textit{High-Fidelity}}
\blindtext

\section{Pengujian Prototipe \textit{High-Fidelity}}
\blindtext
\chapter{Kesimpulan dan Saran}

Bab Kesimpulan dan Saran menjelaskan tentang bagian akhir dari penelitian dan merupakan penutup dari laporan tugas akhir ini. Bab ini membahas tentang kesimpulan yang berisi ketercapaian tujuan penelitian terkait dengan rumusan masalah yang diselesaikan pada tugas akhir. Bab ini juga membahas tentang saran mengenai hal-hal yang dapat dilakukan untuk pengembangan selanjutnya.

\section{Kesimpulan}
Penelitian tugas akhir ini menghasilkan sebuah prototipe \textit{high-fidelity} aplikasi Google Digital Wellbeing. Berdasarkan hasil analisis yang didapatkan, berikut adalah kesimpulan yang diambil

\begin{enumerate}
  \item Desain interaksi yang baik untuk aplikasi Google Digital Wellbeing memprioritaskan \textit{usability goals} memiliki utilitas yang baik (\textit{utility}) dan mudah untuk dipelajari (\textit{learnability}), serta memiliki \textit{user experience goals} \textit{helpful} dan \textit{motivating}. Prototipe \textit{high-fidelity} aplikasi dinilai sudah baik dalam mencapai \textit{goals} tersebut, dilihat dari hasil analisis masukan pengguna pada \textit{usability testing}.
    \begin{enumerate}[label=\alph*.]
      \item Pengguna merasa bahwa prototipe aplikasi sudah memiliki \textit{overall usability} yang cukup baik, melihat 100\% pengguna memberikan penilaian untuk metrik pengukuran \textit{System Usability Scale} (SUS) di atas ambang rata-rata 68, dengan skor terendah 85. 
      
      \item Pengguna merasa bahwa prototipe aplikasi sudah memiliki utilitas yang baik, melihat jawaban dari metrik pengukuran \textit{System Usability Scale} (SUS) 100\% pengguna setuju fungsi-fungsi di dalam prototipe terintegrasi dengan baik, tidak setuju bahwa terdapat banyak hal yang tidak konsisten di dalam prototipe, serta sangat tidak setuju bahwa prototipe tidak praktis untuk dipakai. 
      
      \item Pengguna merasa bahwa prototipe aplikasi mudah untuk dipelajari, melihat penilaian untuk metrik pengukuran \textit{Single Ease Question} (SEQ) memiliki skor rata-rata 6,88 dari skala 7.
      
      \item Pengguna merasa terbantu dalam menggunakan prototipe aplikasi, melihat 100\% pengguna memberikan penilaian untuk metrik pengukuran \textit{Intrinsic Motivation Inventory} (IMI) subskala \textit{Value/Usefulness} di atas nilai ambang batas 6, dengan skor terendah sebesar 6,43 dari skala 7.
      
      \item Pengguna merasa cukup termotivasi dari menggunakan prototipe aplikasi, melihat 60\% pengguna memberikan penilaian untuk metrik pengukuran \textit{Intrinsic Motivation Inventory} (IMI) subskala \textit{Interest/Enjoyment} di atas ambang batas 6, dengan skor terendah 5,14 dari skala 7, serta 80\% pengguna memberikan penilaian muntuk subskala \textit{Pressure/Tension} di bawah ambang batas 2, dengan skor tertinggi 2,20 dari 7.
        
    \end{enumerate}
    
  \item Rancangan desain interaksi aplikasi Digital Wellbeing yang tepat untuk menyelesaikan masalah-masalah dari aplikasi Google Digital Wellbeing memiliki tipe interaksi \textit{Instructing} dan \textit{Responding}, dengan fitur unggulan yaitu fitur Search bar, App Group, dan Daftar Jadwal Aktivasi untuk membantu meningkatkan utilitas dari aplikasi, serta fitur Daily Goal untuk meningkatkan motivasi pengguna dalam mencegah distraksi. Fitur Rekomendasi Aksi dan Smartphone Usage Evaluation ditambahkan sebagai pelengkap fitur yang telah disebutkan. Adapun fitur Date range selector, App Timer, Focus Mode, Take a break, Turn off for now, dan Pengaturan notifikasi perlu dimodifikasi untuk memenuhi kebutuhan dan tujuan pengguna. Perancangan prototipe menerapkan prinsip desain dari \textit{Digital Wellbeing} menurut Google serta prinsip desain interaksi menurut Preece.  
   
\end{enumerate}

\section{Saran}
Pada implementasi prototipe \textit{high-fidelity} aplikasi Google Digital Wellbeing ini, masih banyak hal yang dapat ditingkatkan dan dikembangkan lebih lanjut. Maka dari itu, berikut adalah beberapa saran yang dapat dilakukan dalam pengembangan selanjutnya.

\begin{enumerate}
  % \item Proses pengumpulan data untuk mengidentifikasi masalah pada tugas akhir ini masih menggunakan data dari ulasan aplikasi dari Google Play Store. Maka dari itu, pengumpulan data dapat ditingkatkan dengan melakukan penyebaran form kepada orang-orang yang termasuk ke dalam lingkup pengguna.
  \item Pengumpulan data dapat ditingkatkan dengan melakukan penyebaran form kepada orang-orang yang termasuk ke dalam lingkup pengguna.
  
  % \item Pengumpulan data sebaiknya memperhitungkan juga berapa kali pengguna membuka \textit{smartphone} secara harian, berhubung metrik pengukuran ini dapat menunjukkan seberapa terdistraksi pengguna untuk membuka \textit{smartphone}-nya.
  
  \item Skenario pengujian dapat dilengkapi dengan menyertakan komponen-komponen yang hanya dapat diinteraksi dari prototipe aplikasi yang dapat berjalan, seperti notifikasi aplikasi serta pemblokiran penggunaan aplikasi yang dinilai mendistraksi.
  
  \item \textit{Usability testing} prototipe solusi sebaiknya melibatkan lebih banyak partisipan untuk mengukur ketercapaian \textit{usability goals} dan \textit{user experience goals} dengan lebih baik, selain menemukan masalah \textit{usability} pada prototipe.
  
  \item Menambahkan \textit{dark mode} sebagai tema desain selain yang diterapkan pada prototipe yaitu \textit{light mode}.
  
  \item Membuat tampilan prototipe dalam bahasa Indonesia.
\end{enumerate}

%----------------------------------------------------------------%

% Daftar pustaka
\printbibliography
% \blankpage

% Setting judul lampiran
\titlespacing*{\chapter}{0pt}{0pt}{0pt}
\titlespacing*{\section}{0pt}{0pt}{*1}

% Setting judul anak lampiran
\titleformat*{\section}{\bfseries}
\titleformat*{\subsection}{\bfseries}

% Index
\appendix
\chapter{Tampilan Aplikasi Google Digital Wellbeing}
\label{chpt:gambar_dw}

\begin{figure}[h]
  \centering
  \includegraphics[width=0.25\textwidth]{appendix-a-dashboard.png}
  \caption{Fitur Dashboard pada aplikasi Google Digital Wellbeing (Android, 2019)}
\end{figure}

\begin{figure}[h]
  \centering
  \includegraphics[width=0.25\textwidth]{appendix-a-bedtime-mode.png}
  \caption{Fitur Bedtime Mode pada aplikasi Google Digital Wellbeing (Android, 2019)}
\end{figure}

\begin{figure}[h]
  \centering
  \includegraphics[width=0.25\textwidth]{appendix-a-focus-mode-1.png}
  \caption{Fitur Focus Mode pada aplikasi Google Digital Wellbeing, halaman pengaturan (Android, 2019)}
\end{figure}

\begin{figure}[h]
  \centering
  \includegraphics[width=0.25\textwidth]{appendix-a-focus-mode-2.png}
  \caption{Fitur Focus Mode pada aplikasi Google Digital Wellbeing, mengubah warna ikon aplikasi menjadi berskala abu-abu (Android, 2019)}
\end{figure}

\begin{figure}[h]
  \centering
  \includegraphics[width=0.25\textwidth]{appendix-a-focus-mode-3.jpg}
  \caption{Fitur Focus Mode pada aplikasi Google Digital Wellbeing, notifikasi status aktif}
\end{figure}

\begin{figure}[h]
  \centering
  \includegraphics[width=0.25\textwidth]{appendix-a-app-timer.png}
  \caption{Fitur App Timers pada aplikasi Google Digital Wellbeing (Android, 2019)}
\end{figure}

\chapter{Rincian Analisis Permasalahan}
\label{chpt:rincian_analisis_permasalahan}

Analisis terhadap permasalahan yang telah disebutkan pada subbab \ref{sec:analisis_masalah} adalah sebagai berikut:

\begin{enumerate}
  \item M-001: Fitur "Take a break" dari Focus Mode dapat disalahgunakan untuk mengambil istirahat terus menerus
  \subitem Fitur ”Take a break” yang berbentuk tombol ini dapat memberikan pengguna kembali akses untuk aplikasi yang diblok dengan pilihan waktu 5 menit, 15 menit, atau 30 menit. Tombol ini dapat digunakan jika pengguna ingin beristirahat dan menggunakan aplikasi yang diblok. Namun, fitur ini berpotensi untuk disalahgunakan oleh pengguna untuk mengambil istirahat terus menerus. Pengguna cukup mengambil pilihan istirahat selama 30 menit, dan mengambil lagi tepat setelah waktunya habis untuk memperpanjang waktu istirahatnya.
  
  \item M-002: Fitur "Turn off for now" dari Focus Mode dapat disalahgunakan untuk mematikan Focus Mode sebelum tenggat waktu yang ditentukan
  \subitem Fitur ”Turn off for now” yang berbentuk tombol dapat digunakan untuk memberhentikan Focus Mode hanya untuk hari tersebut. Tombol ini dapat digunakan jika pengguna merasa kegiatannya sudah selesai lebih awal dari yang telah dijadwalkan dan ingin menggunakan sisa harinya untuk mengakses kembali aplikasi yang diblokir. Namun fitur ini dapat disalahgunakan untuk menghindari Focus Mode secara keseluruhan.
   
  \item M-003: Memberikan sugesti atas langkah yang dapat diambil untuk meningkatkan kualitas Digital Wellbeing
  \subitem Aplikasi Digital Wellbeing telah memiliki fungsionalitas yang melaksanakan salah satu cara penerapan konsep Digital Wellbeing, yaitu memberikan pantauan terhadap penggunaan teknologi digital. Hal ini direpresentasikan dengan sebuah ringkasan dari penggunaan aplikasi per harinya dengan metrik jumlah waktu penggunaan aplikasi, jumlah notifikasi yang diterima, dan jumlah pembukaan aplikasi. Namun ringkasan tersebut hanya memberikan informasi kepada pengguna tanpa adanya penilaian apakah performa dari pengguna terbilang baik atau buruk. Hal ini kurang memotivasi pengguna untuk memulai membuat evaluasi terhadap kebiasaannya karena pengguna belum tentu dapat menilai kesehatan dari kebiasaan digitalnya.
   
\end{enumerate}
\chapter{Rincian Analisis Solusi}
\label{chpt:rincian_analisis_solusi}

Analisis terhadap solusi yang telah disebutkan pada subbab \ref{sec:analisis_solusi} adalah sebagai berikut:

\begin{enumerate}
  \item S-001: Memberikan langkah tambahan setiap kali pengguna mengakses fungsionalitas "Take a break"
  \subitem
  Tujuan utama dari tombol "Take a break" adalah untuk pengguna dapat mengambil istirahat dari kegiatan utamanya dan mengakses kembali aplikasi yang diblokir, atau jika sekali-kali pengguna butuh menggunakan aplikasi tersebut. Namun pengaksesan berkali-kali terhadap fungsionalitas ini dapat berarti pengguna sedang menyalahgunakannya. Maka dari itu, jika diberikan sebuah langkah tambahan setiap kali pengguna mengambil istirahat maka diharapkan pengguna akan semakin sulit mengaksesnya sehingga menghindari aplikasi distraksi yang diblokir. Hal ini memanfaatkan \textit{user experience goal} yang tidak diharapkan, \textit{frustrating}, untuk menjauhkan pengguna dari mengakses kembali aplikasi distraksi.
  
  \item S-002: Memindahkan fungsionalitas dari tombol "Turn off for now" ke pengaturan aplikasi
  \subitem
  Tujuan utama dari tombol "Turn off for now" adalah untuk pengguna dapat mengakses kembali aplikasi-aplikasi distraksi lebih awal dari tenggat waktu yang telah ditentukan. Namun, lokasi tombol tersebut yang mudah diakses yaitu pada Notification Bar membuat fitur tersebut mudah disalahgunakan pengguna. Maka dari itu dengan memindahkan fungsionalitas ke halaman aplikasi yang lebih dalam membuat pengguna perlu langkah lebih banyak untuk mematikan fitur Focus Mode. Seperti sebelumnya, hal ini memanfaatkan \textit{user experience goal} yang tidak diharapkan, \textit{frustrating}, untuk menjauhkan pengguna dari mengakses kembali aplikasi distraksi. 
  
  \item S-003: Memberikan sugesti atas langkah yang dapat diambil untuk meningkatkan kualitas pola penggunaan aplikasi
  \subitem Fungsionalitas aplikasi Digital Wellbeing yang hanya memberikan ringkasan dari penggunaan aplikasi kurang dapat memotivasi pengguna untuk melakukan evaluasi terhadap kebiasaannya. Maka dari itu, penilaian lebih atas kebiasaan pengguna dapat memberikan bayangan terhadap apa saja yang bisa ditingkatkan. Penilaian tersebut dapat berbentuk sebuah sugesti kepada pengguna untuk mengambil langkah-langkah tertentu, seperti mengutilisasi fitur lain yang terdapat pada aplikasi Digital Wellbeing. Hal ini memanfaatkan \textit{user experience goal} yang diharapkan, \textit{motivating}, untuk memotivasi pengguna untuk meningkatkan kualitas pola penggunaan aplikasi-aplikasi pada \textit{smartphone}.

\end{enumerate}
\begin{landscape}
  
\chapter{Data Hasil Wawancara}
\label{chpt:hasil_wawancara}

% Vars
\newlength{\colresp}
\setlength{\colresp}{0.26\textwidth}

\newlength{\colbag}
\setlength{\colbag}{0.20\textwidth}

% Functions
\newcommand{\apdhead}[1]{\cellcolor[HTML]{A3E5F5}\textbf{#1}}
\newcommand{\apdheadcell}[1]{\multicolumn{1}{c|}{\apdhead{#1}}}

\newcommand{\borderblue}{\arrayrulecolor[HTML]{A3E5F5}}
\newcommand{\borderblack}{\arrayrulecolor{black}}

\newcommand{\apdbag}[2]{\multirow{#1}{\colbag}{\linespread{1}\selectfont #2}}
\newcommand{\apdline}{\hhline{|-|~|*5{-}|}}

\newcommand{\apdnum}[1]{\colorbox{white}{\raisebox{7pt}{\begin{minipage}[t]{\colresp}\linespread{0.8}\selectfont \begin{enumerate}[parsep=0pt, leftmargin=*] #1 \end{enumerate} \end{minipage}}}}
\newcommand{\apditem}[1]{\colorbox{white}{\raisebox{7pt}{\begin{minipage}[t]{\colresp}\linespread{0.8}\selectfont \begin{itemize}[parsep=0pt, leftmargin=*] #1 \end{itemize} \end{minipage}}}}


\RaggedLeft
\begin{footnotesize}
\begin{longtable}[c]{|m{0.04\textwidth}|>{\baselineskip=8pt}m{\colbag}|>{\baselineskip=8pt}p{\colresp}|>{\baselineskip=8pt}p{\colresp}|>{\baselineskip=8pt}p{\colresp}|>{\baselineskip=8pt}p{\colresp}|>{\baselineskip=8pt}p{\colresp}|}
  
  \hline
  
  \apdhead{} & \apdhead{} & \multicolumn{5}{c|}{\apdhead{Jawaban}} \\ \hhline{|>{\borderblue}->{\borderblack}|>{\borderblue}->{\borderblack}|*5{-}|}
  \rowcolor[HTML]{A3E5F5} \multicolumn{1}{|c|}{\multirow{-2}{*}{\apdhead{No.}}} & \multicolumn{1}{c|}{\multirow{-2}{*}{\apdhead{Bagian}}} & \apdheadcell{Narasumber A} & \apdheadcell{Narasumber B} & \apdheadcell{Narasumber C} & \apdheadcell{Narasumber D} & \apdheadcell{Narasumber E} \\ \hline
  \endfirsthead
  
  \hline
  \apdhead{} & \apdhead{} & \multicolumn{5}{c|}{\apdhead{Jawaban}} \\ \hhline{|>{\borderblue}->{\borderblack}|>{\borderblue}->{\borderblack}|*5{-}|}  
  \rowcolor[HTML]{A3E5F5} \multicolumn{1}{|c|}{\multirow{-2}{*}{\apdhead{No.}}} & \multicolumn{1}{c|}{\multirow{-2}{*}{\apdhead{Bagian}}} & \apdheadcell{Narasumber A} & \apdheadcell{Narasumber B} & \apdheadcell{Narasumber C} & \apdheadcell{Narasumber D} & \apdheadcell{Narasumber E} \\ \hline
  \endhead
  \hline \endfoot
  
  1. &  & Perempuan & Laki-laki & Laki-laki & Perempuan & Laki-laki \\ \apdline
  2. & \apdbag{-1}{Identitas Responden} & 21 tahun & 23 tahun & 29 tahun & 22 tahun & 28 tahun \\ \apdline
  3. &  & Mahasiswi & Wirausaha, pemasaran properti & Wirusaha, koki & Mahasiswi & Pegawai swasta, desain produk \\ \hline
  4. &  & \apdnum{\item Komunikasi lewat aplikasi \textit{messenger} \item \textit{Browsing} internet \item Hiburan}
        & \apdnum{\item Komunikasi lewat aplikasi \textit{messenger} \item Belajar lewat media \item Bekerja \item Hiburan}
        & \apdnum{\item Pekerjaan, untuk mencari ide dan resep \item Komunikasi lewat aplikasi \textit{messenger} \item Media sosial \item Mengambil foto \item Bermain game}
        & \apdnum{\item Komunikasi lewat aplikasi \textit{messenger} \item Kuliah \item Hiburan \item Menggunakan jasa makan online / \textit{e-commerce}}
        & \apdnum{\item Komunikasi untuk pekerjaan lewat aplikasi \textit{messenger} \item Hiburan lewat media sosial \item Mencari informasi di media sosial}
    \\ \apdline
  5. & \apdbag{-1}{Perilaku Penggunaan \textit{Smartphone} Responden} & \apditem{\item Keinginan membuka aplikasi}
        & \apditem{\item Keinginan untuk menggunakan aplikasi \item Notifikasi dari pesan client yang masuk}
        & \apditem{\item Keinginan untuk menggunakan aplikasi \item Notifikasi berupa indikator suara dan visual}
        & \apditem{\item Keinginan untuk menggunakan aplikasi\item Kurang fokus sehingga bermain di \textit{smartphone}}
        & \apditem{\item Keinginan diri sendiri untuk mencari informasi dari media sosial \item Notifikasi dari aplikasi \textit{messenger} \item Keberadaan \textit{smartphone} di sekitar diri sendiri}
    \\ \apdline
  6. &  & 8-12 jam & 7-8 jam & 5-7 jam & 3-7 jam & 3-6 jam \\ \apdline
  7. &  & skala 5 & skala 4 & skala 3 & skala 2 & skala 4 \\ \apdline
  8. &  & skala 1 & skala 2 & skala 4 & skala 1 & skala 3 \\ \hline

  9. &  & \apditem{\item Aplikasi fokus belajar} & \apditem{\item Digital Wellbeing \item Daywise \item Task Organizer} & \apditem{\item Digital Wellbeing} & \apditem{\item Forest \item Digital Wellbeing} & \apditem{\item Digital Wellbeing} \\ \apdline
  10. &  & \apditem{\item Task Reminder \item Bedtime Helper} & \apditem{\item Task Reminder} & \apditem{\item Task Reminder \item Bedtime Helper} & \apditem{\item App Timer} & \apditem{\item Task Reminder \item App Blocker \item Usage Tracker \item Focus Timer} \\ \apdline
  11. &  & \apditem{\item Kurang bisa membangkitkan komitment \item Kurang strict} & \apditem{\item Kurang terintegrasi dengan ekosistem aplikasi \item Kurang memberikan saran untuk memperbaiki kebiasaan} & \apditem{\item Belum menemukan masalah yang cukup signifikan} & \apditem{\item Beberapa informasi penting tidak terdeteksi ketika menggunakan aplikasi pencegah distraksi \item Tidak bisa menambah waktu ketika timer aplikasi habis} & \apditem{\item Kurang tegas dalam mengingatkan penggunaan yang terlalu banyak \item Kurang banyak pengingat sebelum batas waktu aplikasi habis} \\ \apdline
  12. & \apdbag{-4}{Perilaku Responden Terkait Aplikasi Pencegah Distraksi} & \apditem{\item Kemampuan menambahkan goals} & \apditem{\item Visual yang menarik agar pengguna tertarik untuk memakai} & \apditem{\item Notifikasi untuk jam tidur \item Notifikasi jika terlalu lama menggunakan aplikasi yang mendistraksi} & \apditem{\item Ingin timer yang dapat diatur sendiri untuk memblokir akses aplikasi} & \apditem{\item Menggunakan widget untuk melihat sisa waktu penggunaan aplikasi \item Fitur pengingat dalam bentuk notifikasi atau widget \item App blocker / focus timer yang lebih mendetail, seperti memblokir akun tertentu dari aplikasi \textit{messenger}} \\ \hline

  13. &  & \apditem{\item Memakai App Timer untuk memblokir aplikasi} & \apditem{\item Memakai Bedtime Mode untuk mengingatkan jam tidur} & \apditem{\item Mengingatkan terhadap penggunaan \textit{smartphone} yang tinggi \item Lebih fokus dengan pekerjaan dengan memblokir aplikasi \item Membantu menjaga kualitas tidur} & \apditem{\item Menggunakan App Timer untuk membatasi waktu menggunakan aplikasi} & \apditem{\item App Timer untuk mengatur agar tidak menggunakan aplikasi terlalu lama \item Focus Mode untuk mengurangi distraksi di saat bekerja \item Dashboard untuk melihat aktivitas dan evaluasi penggunaan \textit{smartphone}} \\ \apdline
  14. & \apdbag{-1}{Perilaku Responden Terkait Aplikasi Digital Wellbeing} & \apditem{\item Kurang ketat dalam membatasi akses aplikasi} & \apditem{\item Tampilan terlalu kaku, kurang user friendly dan mengajak \item Kurang fitur untuk mengelompokkan aplikasi \item Kurang mengerti fi Dashboard} & \apditem{\item Belum menemukan masalah yang cukup signifikan} & \apditem{\item Jika App Timer habis, tidak dapat menambah waktu sebentar tanpa menghapus timer} & \apditem{\item Notifikasi terlalu mepet untuk App Timer \item Focus Mode bisa disalahgunakan} \\ \apdline
  15. & & \apditem{\item Kemampuan menambahkan pesan / tugas harian \item Fitur Search saat memilih aplikasi untuk dipasang App Timer / Focus Mode} & \apditem{\item Integrasi dengan aplikasi lain milik Google \item Fitur untuk menambah pesan saat ingin menggunakan fitur / membuat jadwal \item Tampilan yang lebih menarik \item Fitur Search saat memilih aplikasi untuk dipasang App Timer / Focus Mode} & \apditem{\item Memberikan notifikasi periodik untuk aplikasi yang dibatasi oleh App Timer \item Memberikan notifikasi tentang berapa lama telah menggunakan Focus Mode \item Fitur Search saat memilih aplikasi untuk dipasang App Timer / Focus Mode} & \apditem{\item Ingin kemampuan untuk menunda App Timer / Focus Mode dengan waktu yang ditentukan sendiri \item Fitur Search saat memilih aplikasi untuk dipasang App Timer / Focus Mode} & \apditem{\item Kemampuan menambah tugas yang perlu dikerjakan untuk diingatkan \item Kemampuan melihat batas waktu App Timer dengan progress bar \item Bisa memblokir kontak tertentu di aplikasi \textit{messenger} \item Fitur Search saat memilih aplikasi untuk dipasang App Timer / Focus Mode} \\ \hline

  16. &  & Tidak & Ya, untuk mengetahui batas timer" dan membatasi penggunaan app & Ya, untuk melihat batas waktu aplikasi yang dibatasi & Ya, untuk melihat batas waktu aplikasi yang dibatasi & Ya, untuk mengingatkan batas waktu aplikasi yang dibatasi, dalam bentuk progress bar \\ \apdline
  17. & \apdbag{-2}{Validasi masalah kurangnya fitur widget pada Homescreen} & Tidak & Ya, karena peletakan di notifikasi menghalangi notif lain & Tidak & Tidak & Ya, untuk memberi tahu berapa kali kesempatan untuk mengambil break. Berguna untuk orang yang kerja tanpa jadwal tetap \\ \apdline
  18. &  & Ya, untuk melihat jadwal & Hanya jadi shortcut & Tidak & Tidak & Tidak \\ \hline

  19. & \apdbag{-1}{Validasi masalah pada fitur laporan data penggunaan aplikasi} & \apditem{\item Data penggunaan tertinggi dan terendah \item Data rata-rata penggunaan mingguan \item Data penggunaan \texttt{>1} bulan \item Tampilan data mingguan} & \apditem{\item Data rata-rata penggunaan mingguan \item Garis besar aktivitas \texttt{>1} bulan (hanya ringkasan data) \item Tampilan data mingguan} & \apditem{\item Data rata-rata pengunaan mingguan \item Data penggunaan \texttt{>1} bulan \item Tampilan data mingguan} & \apditem{\item Data rata-rata penggunaan mingguan \item Data penggunaan \texttt{>1} bulan \item Indikator pembanding penggunaan dengan hari-hari sebelumnya} & \apditem{\item Data rata-rata penggunaan mingguan, bisa menyontoh grafik pasar saham \item Kemampuan untuk menentukan periode waktu yang ditampilkan pada grafik \item Data penggunaan \texttt{>1} bulan} \\ \hline

  20. &  & Dipersulit untuk mematikan & Tombol break dihilangkan & Hanya bisa dimatikan / istirahat lewat halaman utama & Ya, diatur pengguna dengan tingkat keketatan & Ya, diberi kemampuan untuk menunjukan sisa waktu yang dapat diambil untuk istirahat \\ \apdline
  21. & \apdbag{-2}{Validasi masalah pada fitur Focus Mode} & Dipersulit untuk mematikan saat aktif & Dipersulit untuk mematikan saat aktif & Cukup & Perlu dibatasi jumlah istirahat yang bisa diambil, perlu dipersulit untuk mematikan, hanya jika bisa diatur oleh pengguna & Ya, perlu dipersulit tapi ada pengaturan yang bisa diatur secara sadar \\ \hline

  22. &  & Tidak & Ya & Ya & Ya & Ya \\ \apdline
  23. &  & Ya & Ya & Ya, bisa diberi judul & Ya & Ya \\ \apdline
  24. & \apdbag{-3}{Validasi masalah untuk kemampuan penjadwalan pada fitur-fitur} & Ya & Ya & Tidak & Tidak & Tidak \\ \hline

  25. &  & Kunci dengan random string & Kunci dengan random string dan password & Kunci dengan password, random string, atau tugas sulit lainnya, tapi hanya untuk App Timer & Kunci dengan password, random string & Kunci dengan password, timer, random string, atau quiz \\ \apdline
  26. &  & Narasumber & Narasumber & Narasumber & Narasumber & Narasumber \\ \apdline
  27. & \apdbag{-3}{Validasi masalah kurangnya fitur pengaturan tingkat keketatan} & Ya, dilarang ngubah timer jika habis & Ya, ada opsi untuk dipersulit & Ya, perlu dipersulit & Cukup & Ya, dilarang untuk mengubah pengaturan jika batas waktu telah habis, untuk mengambil istirahat perlu melakukan task yang sulit \\ \hline

  28. &  & Tidak & Perlu lebih banyak, bisa memilih sendiri & Ya, bisa diingatkan secara periodik & Tidak & Ya, lebih sering diingatkan \\ \apdline
  29. & \apdbag{-1}{Validasi masalah kurangnya fitur penundaan pada App Timer} & Tidak & Hanya terbatas, kalau bisa buka aplikasi hanya sekali saja lalu langsung ditutup & Tidak & Ya, perlu menunda dengan waktu yang dipilih sendiri & Ya, bisa menunda App Timer dengan terbatas atau dengan konsekuensi melakukan sebuah task yang sulit \\ \apdline
  30. &  & Ya & Ya & Ya & Ya & Tidak \\ \hline

  31. &  & Ya, dalam paketan & Boleh & Ya, diatur sendiri & Ya & Tidak \\ \apdline
  32. &  & Ya & Ya & Tidak & Ya & Tidak \\ \apdline
  33. & \apdbag{-3}{Validasi masalah pada fitur Bedtime Mode} & Ya & Tidak & Ya, perlu dibatasi & Ya & Tidak \\ \hline

  34. &  & Cukup & Kurang jelas di Dashboard, perlu saran untuk memperbaiki wellbeing & Cukup & Cukup & Cukup \\ \apdline
  35. & \apdbag{-2}{Validasi masalah kurangnya penjelasan dan susunan kata} & Ya, ingin menambah goals ke Focus Mode, notif, Dashbaord, App Timer, supaya ingat kenapa memakai fitur tersebut & Ya & Ya & Tidak & Ya, untuk mengingatkan mengapa menggunakan fitur \\ \hline

  36. & Validasi masalah kurangnya fitur pengelompokkan aplikasi & Tidak & Ya & Ya & Tidak & Ya, kategori diberi nama sendiri \\ \hline

  37. & Validasi masalah kurangnya fitur pengaturan jam akhir hari & Ya & Ya & Ya & Ya & Ya \\ \hline

  38. & Validasi masalah kurangnya kemampuan whitelisting & Tidak & Ya & Tidak & Ya, cukup menarik & Ya, tertarik dan lebih memilih whitelisting daripada blacklisting \\ \hline
  
  

\end{longtable}
\end{footnotesize}
\justifying

\newpage

\RaggedLeft
\begin{footnotesize}
\begin{longtable}[c]{|m{0.04\textwidth}|>{\baselineskip=8pt}m{\colbag}|>{\baselineskip=8pt}p{\colresp}|>{\baselineskip=8pt}p{\colresp}|>{\baselineskip=8pt}p{\colresp}|>{\baselineskip=8pt}p{\colresp}|>{\baselineskip=8pt}p{\colresp}|}
  
  \hline
  
  \apdhead{} & \apdhead{} & \multicolumn{5}{c|}{\apdhead{Jawaban}} \\ \hhline{|>{\borderblue}->{\borderblack}|>{\borderblue}->{\borderblack}|*5{-}|}
  \rowcolor[HTML]{A3E5F5} \multicolumn{1}{|c|}{\multirow{-2}{*}{\apdhead{No.}}} & \multicolumn{1}{c|}{\multirow{-2}{*}{\apdhead{Bagian}}} & \apdheadcell{Narasumber F} & \apdheadcell{Narasumber G} & \apdheadcell{Narasumber H} & \apdheadcell{Narasumber I} & \apdheadcell{Narasumber J} \\ \hline
  \endfirsthead
  
  \hline
  \apdhead{} & \apdhead{} & \multicolumn{5}{c|}{\apdhead{Jawaban}} \\ \hhline{|>{\borderblue}->{\borderblack}|>{\borderblue}->{\borderblack}|*5{-}|}  
  \rowcolor[HTML]{A3E5F5} \multicolumn{1}{|c|}{\multirow{-2}{*}{\apdhead{No.}}} & \multicolumn{1}{c|}{\multirow{-2}{*}{\apdhead{Bagian}}} & \apdheadcell{Narasumber F} & \apdheadcell{Narasumber G} & \apdheadcell{Narasumber H} & \apdheadcell{Narasumber I} & \apdheadcell{Narasumber J} \\ \hline
  \endhead
  \hline \endfoot
  
  1. &  & Perempuan & Perempuan & Perempuan & Laki-laki & Laki-laki \\ \apdline
  2. & \apdbag{-1}{Identitas Responden} & 29 tahun & 22 tahun & 22 tahun & 22 tahun & 22 tahun \\ \apdline
  3. &  & Finance accounting, kerja kantoran & Mahasiswi & Mahasiswi & Mahasiswa & Mahasiswa \\ \hline
  4. &  & \apdnum{\item Komunikasi lewat aplikasi \textit{messenger} \item Pekerjaan \item Hiburan dengan bermain game, media sosial, menonton film}
        & \apdnum{\item Komunikasi lewat aplikasi \textit{messenger} \item Bersosialisasi lewat media sosial \item Hiburan dengan menonton YouTube, bermain game ringan, dan mendengarkan lagu \item Menggunakan jasa e-commerce}
        & \apdnum{\item Komunikasi lewat aplikasi \textit{messenger} \item Hiburan dengan menonton film, bermain game, bermain media sosial \item Utilitas}
        & \apdnum{\item Komunikasi lewat aplikasi \textit{messenger} \item Hiburan dengan menonton YouTube, bermain media sosial \item Sarana informasi \item Bermain game \item Utilitas}
        & \apdnum{\item Komunikasi lewat aplikasi \textit{messenger} \item Utilitas \item Bermain media sosial \item Hiburan dengan menonton Youtube atau film}
    \\ \apdline
  5. & \apdbag{-1}{Perilaku Penggunaan \textit{Smartphone} Responden} & \apditem{\item Keinginan untuk mengontak orang lain \item Keinginan untuk bermain game}
        & \apditem{\item Lupa akan tujuan utama saat menggunakan aplikasi \item Keinginan untuk menggunakan aplikasi di luar tujuan}
        & \apditem{\item Keinginan diri untuk menggunakan aplikasi}
        & \apditem{\item Menjadi distraksi ketika merasa bosan \item Menerima distraski ketika sedang menunggu informasi / hal lain \item Distraksi dari notifikasi aplikasi}
        & \apditem{\item Keinginan diri untuk menggunakan media sosial atau bermain game \item Notifikasi dari aplikasi \textit{e-commerce} \item Notifikasi dari media sosial \item Keberadaan \textit{smartphone} di sekitar diri sendiri}
    \\ \apdline
  6. &  & 5-8 jam & 5-11 jam & 5-7 jam & 4-5 jam & 2-6 jam \\ \apdline
  7. &  & skala 5 & skala 5 & skala 4 & skala 3 & skala 4 \\ \apdline
  8. &  & skala 3 & skala 4 & skala 3 & skala 1 & skala 3 \\ \hline

  \pagebreak

  9. &  & \apdnum{\item Digital Wellbeing \item App Timer} & \apdnum{\item Forest \item Digital Wellbeing} & \apdnum{\item Digital Wellbeing \item App Timer} & \apdnum{\item Aplikasi Game Booster untuk mencegah distraksi saat sedang bermain} & \apdnum{\item Digital Wellbeing \item Forest \item ActionDash \item Minimalist Device} \\ \apdline
  10. &  & \apdnum{\item Bedtime Helper \item Usage Tracker \item App Timer} & \apdnum{\item Task Reminder \item Focus Timer \item Usage Tracker \item Bedtime Helper} & \apdnum{\item Usage Tracker \item Task Reminder \item App Timer} & \apdnum{\item App Timer \item Task Reminder} & \apdnum{\item App Timer \item Focus Timer \item Task Reminder \item Bedtime Helper \item Usage Tracker} \\ \apdline
  11. & \apdbag{-1}{Perilaku Responden Terkait Aplikasi Pencegah Distraksi} & \apdnum{\item Sulit untuk mengakses aplikasi yang sudah habis batas waktu pemakaiannya jika ingin menggunakan di akhir hari } & \apdnum{\item Tidak bisa mengatasi distraksi yang berasal dari luar smartphone } & \apdnum{\item Tidak ada saran atau rekomendasi durasi saat menggunakan App Timer \item Harus mengatur ulang timer setiap hari jika batas waktu habis} & \apdnum{\item Aplikasi pencegah distraksi terasa terlalu ketat dengan mencegah semua notifikasi yang akan diterima} & \apdnum{\item Aplikasi kurang ketat dalam membatasi akses \item Aplikasi kurang memberikan tolok ukur penggunaan yang sehat \item Aplikasi kurang sistem reward yang memuaskan" } \\ \apdline
  12. &  & \apdnum{\item Aplikasi lebih gencar dalam memperkenalkan fitur-fiturnya kepada pengguna} & \apdnum{\item Fitur pengingat tugas yang harus dikerjakan \item Sistem penghargaan yang bersifat visual} & \apdnum{\item Sistem penjadwalan \item Notifikasi untuk memberi informasi lama penggunaan aplikasi \item Deskripsi yang lengkap untuk menjelaskan fitur-fitur aplikasi} & \apdnum{\item - Fitur untuk memilih notifikasi / aplikasi yang tidak diblokir} & \apdnum{\item Fitur pengunci terhadap pengaturan pembatasan akses \item Fitur pengingat terhadap lama penggunaan aplikasi} \\ \apdline

  \pagebreak

  13. &  & \apdnum{\item Menggunakan App Timer untuk membatasi akses ke aplikasi \item Menggunakan Bedtime Mode memperbaiki jadwal tidur} & \apdnum{\item Meningkatkan kesadaran diri tentang seberapa banyak menggunakan aplikasi di smartphone \item Mengurangi penggunaan smartphone / aplikasi yang berlebihan \item Memperbaiki kebiasaan digital} & \apdnum{\item Menyadarkan diri untuk fokus terhadap pekerjaan \item Menyadarkan diri akan jam tidur \item Melihat informasi lama penggunaan smartphone / aplikasi pada Usage Tracker} & \apdnum{\item Melihat laporan data penggunaan smartphone dari Dashboard untuk membatasi penggunaan smartphone dari diri sendiri \item Menggunakan App Timer untuk membatasi akses ke aplikasi} & \apdnum{\item Focus Mode untuk membatasi akses ke aplikasi distraksi di saat bekerja \item App Timer untuk membatasi lama akses ke aplikasi media sosial \item Bedtime Mode untuk mengingatkan jam tidur" } \\ \apdline
  14. & \apdbag{-1}{Perilaku Responden Terkait Aplikasi Digital Wellbeing} & \apdnum{\item Pada App Timer, tidak bisa menunda batas waktu untuk menyelesaikan keperluan tak terduga tanpa menghapus timer } & \apdnum{\item Kurang adanya penghargaan ketika berhasil memperbaiki kebiasaan digital } & \apdnum{\item Kebingungan dari cara kerja fitur-fitur tertentu } & \apdnum{\item Kurang ketat dalam membatasi akses \item Kurangnya sugesti / rekomendasi untuk memberikan tolok ukur antara penggunaan smartphone yang sehat atau tidak sehat} & \apdnum{\item Kurang ketat dalam membatasi akses dan mencegah pengubahan pengaturan \item Kurang ada tolok ukur penggunaan smartphone yang sehat} \\ \apdline
  15. &  & \apdnum{\item Fitur Search saat memilih aplikasi untuk dipasang App Timer / Focus Mode } & \apdnum{\item Fitur yang memberikan sense of reward jika berhasil memperbaiki kebiasaan \item Fitur untuk mencatat target yang ingin dicapai dan pengingat target tersebut} & \apdnum{\item Akses yang lebih mudah ke dalam aplikasi \item Fitur Search saat memilih aplikasi untuk dipasang App Timer / Focus Mode} & \apdnum{\item Sistem rekomendasi tentang tolok ukur pemakaian smartphone yang sehat } & \apdnum{\item Pengingat yang lebih sering untuk App Timer \item Focus Mode yang lebih ketat \item Fitur Search saat memilih aplikasi untuk dipasang App Timer / Focus Mode} \\ \hline
  
  \pagebreak

  16. &  & Tidak & Ya, untuk melihat sisa batas waktu aplikasi yang dibatasi & Ya, untuk melihat sisa batas waktu aplikasi yang dibatasi & Ya, agar lebih mudah diakses dan melihat sisa batas waktu aplikasi yang dibatasi & Ya, untuk melihat sisa batas waktu aplikasi yang dibatasi \\ \apdline
  17. & \apdbag{-2}{Validasi masalah kurangnya fitur widget pada Homescreen} & Ya, lebih memilih pengaturan dalam bentuk widget daripada notification & Ya, untuk memberi tahu sisa waktu Focus Mode & Tidak & Ya, agar lebih mudah diakses dan memilih aplikasi yang ingin dibatasi & Tidak \\ \apdline
  18. &  & Tidak & Tidak & Tidak & Tidak & Tidak \\ \hline
  19. & \apdbag{-1}{Validasi masalah pada fitur laporan data penggunaan aplikasi} & \apdnum{\item Tampilan data dalam periode mingguan \item Grafik yang lebih menarik dengan warna yang berbeda untuk menunjukkan aplikasi yang paling banyak dipakai hari di hari tersebut} & \apdnum{\item Tampilan data dalam periode mingguan dan bulanan \item Indikator target lama penggunaan aplikasi / smartphone } & \apdnum{\item Data rata-rata penggunaan mingguan \item Data penggunaan \texttt{>1} bulan \item Indikator aplikasi yang paling banyak digunakan pada grafik \item Tampilan data dalam periode yang bisa diatur} & \apdnum{\item Rekomendasi tolok ukur penggunaan smartphone yang sehat pada grafik \item Variasi tampilan / bentuk grafik} & \apdnum{\item Data rata-rata penggunaan mingguan \item Data penggunaan \texttt{>1} bulan \item Rekomendasi tolok ukur penggunaan smartphone yang sehat pada grafik \item Tampilan data dalam periode yang bisa diatur } \\ \hline
  20. &  & Ya, tombol "Turn off for now" dipindah posisinya & Ya, dipersulit untuk dimatikan & Ya, dipersulit untuk dimatikan atau istirahat & Tidak & Ya, dipersulit untuk dimatikan atau istirahat \\ \apdline
  21. & \apdbag{-2}{Validasi masalah pada fitur Focus Mode} & Tidak & Ya, dipersulit untuk dimatikan & Ya, perlu ada opsi untuk dipersulit untuk dimatikan & Tidak & Ya, perlu ada opsi untuk dipersulit untuk dimatikan \\ \hline

  22. &  & Ya & Ya & Ya & Ya & Tidak \\ \apdline
  23. & \apdbag{-2}{Validasi masalah untuk kemampuan penjadwalan pada fitur-fitur} & Ya & Ya & Ya & Ya & Ya \\ \apdline
  24. &  & Ya & Tidak & Tidak & Ya & Tidak \\ \hline


  25. &  & Tidak & Ya, namun tetap ingin fitur-fitur fleksible untuk diatur & Kunci dengan password, random string, atau quiz & Kunci dengan password atau random string & Kunci dengan password, random string, atau timer \\ \apdline
  26. &  & Ya, menarik jika ada & Ya & Ya, perlu ada pengaturan yang sulit untuk diubah jika tingkat ketat tinggi & Tidak setuju, karena jika aplikasi dibuat lebih ketat, maka kesehatan penggunaan smartphone lebih sulit untuk diperbaiki, karena harus lebih dominan dari kesadaran diri sendiri & Ya, perlu ada opsi untuk membuat fitur-fitur menjadi lebih ketat \\ \apdline
  27. & \apdbag{-6}{Validasi masalah kurangnya fitur pengaturan tingkat keketatan} & Ya, dilarang untuk dimatikan jika waktu untuk habis di hari tersebut & Ya, perlu dipersulit, namun tetap perlu pengaturan untuk tetap fleksibel & Tidak & Tidak & Ya, dilarang untuk diubah jika batas waktu telah tercapai \\ \hline

  28. &  & Ya, lebih sering diingatkan & Ya, lebih sering diingatkan & Ya, perlu pengingat yang dapat diatur sendiri & Ya, lebih sering diingatkan & Ya, lebih sering diingatkan \\ \apdline
  29. & \apdbag{-2}{Validasi masalah kurangnya fitur penundaan pada App Timer} & Tidak & Ya & Ya, butuh penundaan yang terbatas & Ya, dapat diatur aplikasi mana yang bisa ditunda & Ya, butuh penundaan yang terbatas \\ \apdline
  30. &  & Tidak & Ya & Ya & Ya & Ya \\ \hline

  31. &  & Tidak & Ya & Tidak & Ya & Ya \\ \apdline
  32. &  & Tidak & Tidak & Perlu dipersulit & Ya & Ya \\ \apdline
  33. & \apdbag{-3}{Validasi masalah pada fitur Bedtime Mode} & Ya, tidak perlu ada opsi untuk menunda & Ya & Ya & Ya & Ya \\ \hline
  
  34. &  & Cukup, perlu ditambah tombol untuk menampilkan informasi yang lebih banyak bagi fitur yang lebih kompleks & \apditem{ \item Kurang penjelasan tentang penggunaan dan tujuan fitur Dashboard \item Kurang penjelasan / sugesti pada App Timer yang dapat memberikan saran penggunaan} & Cukup & \apditem{ \item Kurang mudah dimengerti \item Tata letak fitur dari aplikasi perlu diperbaiki} & Cukup \\ \apdline
  35. & \apdbag{-5}{Validasi masalah kurangnya penjelasan dan susunan kata} & Tidak & Ya & Ya, pengingat harus bisa diatur agar terasa lebih pribadi & Ya, pengingat harus bisa diatur agar terasa lebih pribadi & Perlu \\ \hline

  36. & Validasi masalah kurangnya fitur pengelompokkan aplikasi & Tidak & Ya & Ya & Ya & Tidak \\ \hline
  
  37. & Validasi masalah kurangnya fitur pengaturan jam akhir hari & Tidak & Tidak & Ya & Ya & Ya \\ \hline
  
  38. & Validasi masalah kurangnya kemampuan whitelisting & Tidak & Ya, tapi lebih memilih blacklisting & Ya, lebih memilih whitelisting & Ya, lebih memilih whitelisting & Tidak \\ \hline
  
  

\end{longtable}
\end{footnotesize}
\justifying

\FloatBarrier
\end{landscape}

\chapter{Skenario Pengujian Prototipe \textit{Low-Fidelity}}
\label{chpt:skenario_lofi}

Setelah setiap pengerjaan skenario, partisipan akan diberikan pertanyaan mengenai skenario dan tugas yang telah mereka kerjakan. Pertanyaan ini berguna untuk mengevaluasi alur prototipe, informasi yang terdapat pada prototipe, serta mencari saran atau kritik dari partisipan. Pertanyaan-pertanyaannya adalah sebagai berikut

\begin{enumerate}
  \item Apa tanggapan Anda mengenai alur tampilan untuk menyelesaikan task tersebut?
  \item Apakah informasi pada tampilan cukup untuk membantu menyelesaikan task tersebut?
  \item Apakah ada masukan / saran mengenai tampilan dari halaman yang dilalui?
\end{enumerate}

\RaggedLeft
\begin{small}
\begin{longtable}[c]{|>{\ccnormspacing}m{0.19\textwidth}|>{\ccnormspacing}p{0.73\textwidth}|}
  
  \hline
  \rowcolor[HTML]{A3E5F5} \multicolumn{2}{|l|}{\textbf{Skenario Pengujian 1}} \\ \hline
  Kaitan Skenario Pengguna & SP-02 \\ \hline
  Tujuan & Mengukur pemahaman pengguna dalam menganalisis data penggunaan \textit{smartphone} \\ \hline
  Skenario & Pengguna ingin melihat data penggunaan \textit{smartphone} harian. \\ \hline
  Task & Lihat data penggunaan \textit{smartphone} \\ \hline
  Pra kondisi & Pengguna berada di Halaman Main Menu  \\ \hline

  \rowcolor[HTML]{A3E5F5} \multicolumn{2}{|l|}{\textbf{Skenario Pengujian 2}} \\ \hline
  Kaitan Skenario Pengguna & SP-03 \\ \hline
  Tujuan & Mengukur pemahaman pengguna dalam membuat sebuah App Group \\ \hline
  Skenario & Pengguna ingin mengelompokkan aplikasi-aplikasi ke dalam sebuah kategori. \\ \hline
  Task & Kelompokan aplikasi-aplikasi ke dalam sebuah kategori  \\ \hline
  Pra kondisi & Pengguna berada di Halaman Dashboard \\ \hline

  \rowcolor[HTML]{A3E5F5} \multicolumn{2}{|l|}{\textbf{Skenario Pengujian 3}} \\ \hline
  Kaitan Skenario Pengguna & SP-02 \\ \hline
  Tujuan & Mengukur pemahaman pengguna dalam menganalisis data penggunaan satu buah aplikasi \\ \hline
  Skenario & Pengguna sudah melihat data penggunaan \textit{smartphone}. Pengguna ingin melihat data penggunaan untuk hanya sebuah aplikasi. \\ \hline
  Task & Lihat data penggunaan dari sebuah aplikasi \\ \hline
  Pra kondisi & Pengguna berada di Halaman Dashboard \\ \hline

  \rowcolor[HTML]{A3E5F5} \multicolumn{2}{|l|}{\textbf{Skenario Pengujian 4}} \\ \hline
  Kaitan Skenario Pengguna & SP-02 \\ \hline
  Tujuan & Mengukur pemahaman pengguna dalam menganalisis data penggunaan App Group \\ \hline
  Skenario & Pengguna ingin melihat data penggunaan App Group yang sudah dibuat. \\ \hline
  Task & Lihat data penggunaan dari kategori aplikasi yang sudah dibuat \\ \hline
  Pra kondisi & Pengguna berada di Halaman Dashboard \\ \hline

  \rowcolor[HTML]{A3E5F5} \multicolumn{2}{|l|}{\textbf{Skenario Pengujian 5}} \\ \hline
  Kaitan Skenario Pengguna & SP-03 \\ \hline
  Tujuan & Mengukur pemahaman pengguna dalam memasang App Timer pada sebuah aplikasi \\ \hline
  Skenario & Pengguna ingin membatasi waktu penggunaan sebuah aplikasi. \\ \hline
  Task & Batasi waktu penggunaan dari sebuah aplikasi \\ \hline
  Pra kondisi & Pengguna berada di Halaman Main Menu \\ \hline

  \rowcolor[HTML]{A3E5F5} \multicolumn{2}{|l|}{\textbf{Skenario Pengujian 6}} \\ \hline
  Kaitan Skenario Pengguna & SP-05 \\ \hline
  Tujuan & Mengukur pemahaman pengguna dalam menunda aktivasi App Timer \\ \hline
  Skenario & Pengguna ingin membatasi waktu penggunaan sebuah App Group. \\ \hline
  Task & Tunda pemblokiran untuk aplikasi yang batas waktu penggunaannya sudah habis \\ \hline
  Pra kondisi & Pengguna berada di Halaman Main Menu \\ \hline

  \rowcolor[HTML]{A3E5F5} \multicolumn{2}{|l|}{\textbf{Skenario Pengujian 7}} \\ \hline
  Kaitan Skenario Pengguna & SP-04 \\ \hline
  Tujuan & Mengukur pemahaman pengguna dalam menambah jadwal Focus Mode \\ \hline
  Skenario & Pengguna ingin fokus pada pekerjaannya di jam kerja. Pengguna ingin memblokir beberapa aplikasi di jam kerjanya. \\ \hline
  Task & Buat jadwal pemblokiran aplikasi-aplikasi mendistraksi sesuai dengan jam kerja \\ \hline
  Pra kondisi & Pengguna berada di Halaman Main Menu \\ \hline

  \rowcolor[HTML]{A3E5F5} \multicolumn{2}{|l|}{\textbf{Skenario Pengujian 8}} \\ \hline
  Kaitan Skenario Pengguna & SP-05 \\ \hline
  Tujuan & Mengukur pemahaman pengguna dalam menunda aktivasi Fokus Mode \\ \hline
  Skenario & Pengguna ingin beristirahat dari kerjanya sejenak. Pengguna ingin menggunakan aplikasi yang diblokir secara sementara. \\ \hline
  Task & Tunda pemblokiran aplikasi yang sudah dipasang sebelumnya \\ \hline
  Pra kondisi & Pengguna berada di Halaman Main Menu \\ \hline
  
  \rowcolor[HTML]{A3E5F5} \multicolumn{2}{|l|}{\textbf{Skenario Pengujian 9}} \\ \hline
  Kaitan Skenario Pengguna & SP-01 \\ \hline
  Tujuan & Mengukur pemahaman pengguna dalam menentukan Daily Goal \\ \hline
  Skenario & Pengguna memiliki sebuah target yang harus dipenuhi hari ini. Pengguna ingin mengingatkan diri terhadap capaian tersebut. \\ \hline
  Task & Pasang pengingat untuk capaian yang harus dipenuhi hari ini \\ \hline
  Pra kondisi & Pengguna berada di Halaman Main Menu \\ \hline
  
  \rowcolor[HTML]{A3E5F5} \multicolumn{2}{|l|}{\textbf{Skenario Pengujian 10}} \\ \hline
  Kaitan Skenario Pengguna & SP-06 \\ \hline
  Tujuan & Mengukur kecukupan informasi dalam membantu pengguna mengatur jadwal Bedtime Mode \\ \hline
  Skenario & Pengguna ingin memperbaiki jam tidurnya. Pengguna ingin mengurangi penggunaan \textit{smartphone} di saat jam tidur. \\ \hline
  Task & Aturlah agar \textit{smartphone} menjadi lebih hening di saat jam tidur. \\ \hline
  Pra kondisi & Pengguna berada di Halaman Main Menu \\ \hline
  
  \rowcolor[HTML]{A3E5F5} \multicolumn{2}{|l|}{\textbf{Skenario Pengujian 11}} \\ \hline
  Kaitan Skenario Pengguna & SP-01 \\ \hline
  Tujuan & Mengukur pemahaman pengguna dalam menggunakan Daily Summary Notification \\ \hline
  Skenario & Pengguna ingin diberikan ringkasan harian tentang penggunaan \textit{smartphone}-nya pada hari tersebut. \\ \hline
  Task & Pasang pengingat di akhir hari untuk memberikan ringkasan penggunaan \textit{smartphone}  dan evaluasi. \\ \hline
  Pra kondisi & Pengguna berada di Halaman Daily Goal \\ \hline

  \rowcolor[HTML]{A3E5F5} \multicolumn{2}{|l|}{\textbf{Skenario Pengujian 12}} \\ \hline
  Kaitan Skenario Pengguna & SP-02 \\ \hline
  Tujuan & Mengukur pemahaman pengguna tentang widget Dashboard \\ \hline
  Skenario & Pengguna ingin melihat pengunaan \textit{smartphone}-nya hari ini tanpa masuk ke aplikasi Digital Wellbeing \\ \hline
  Task & Lihat data penggunaan \textit{smartphone} hari ini dari Homescreen \\ \hline
  Pra kondisi & Pengguna berada di Homescreen \textit{smartphone} \\ \hline

  \rowcolor[HTML]{A3E5F5} \multicolumn{2}{|l|}{\textbf{Skenario Pengujian 13}} \\ \hline
  Kaitan Skenario Pengguna & SP-02 \\ \hline
  Tujuan & Mengukur pemahaman pengguna tentang widget App Timer \\ \hline
  Skenario & Pengguna ingin melihat sisa waktu pengunaan aplikasi yang dibatasi App Timer tanpa masuk ke aplikasi Digital Wellbeing \\ \hline
  Task & Lihat data sisa waktu penggunaan aplikasi dari Homescreen \\ \hline
  Pra kondisi & Pengguna berada di Homescreen \textit{smartphone} \\ \hline
  
  \rowcolor[HTML]{A3E5F5} \multicolumn{2}{|l|}{\textbf{Skenario Pengujian 14}} \\ \hline
  Kaitan Skenario Pengguna & SP-02 \\ \hline
  Tujuan & Mengukur pemahaman pengguna tentang widget Focus Mode \\ \hline
  Skenario & Pengguna ingin membatasi akses ke aplikasi yang mendistraksi tanpa masuk ke aplikasi Digital Wellbeing \\ \hline
  Task & Batasi akses ke aplikasi yang mendistraski dari Homescreen \\ \hline
  Pra kondisi & Pengguna berada di Homescreen \textit{smartphone} \\ \hline

\end{longtable}
\end{small}


\chapter{Hasil Pengujian Prototipe \textit{Low-Fidelity}}
\label{chpt:hasil_test_lofi}

\RaggedLeft
\begin{footnotesize}
\begin{longtable}[c]{|>{\ccnormspacingcenter}m{0.11\textwidth}|>{\ccnormspacing}p{0.25\textwidth}|>{\ccnormspacing}p{0.25\textwidth}|>{\ccnormspacing}p{0.25\textwidth}|}

  \hline \rowcolor[HTML]{A3E5F5}
  \multicolumn{4}{|l|}{\textbf{Partisipan 1}} \\
  \hline \rowcolor[HTML]{DCF3FC}
  \textbf{Skenario Pengujian} & \multicolumn{1}{c|}{\textbf{Tanggapan Alur}} & \multicolumn{1}{c|}{\textbf{Tanggapan Informasi}} & \multicolumn{1}{c|}{\textbf{Kritik \& Saran}} \\ \hline \endfirsthead
  
  \hline \rowcolor[HTML]{A3E5F5}
  \multicolumn{4}{|l|}{\textbf{Partisipan 1}} \\
  \hline \rowcolor[HTML]{DCF3FC}
  \textbf{Skenario Pengujian} & \multicolumn{1}{c|}{\textbf{Tanggapan Alur}} & \multicolumn{1}{c|}{\textbf{Tanggapan Informasi}} & \multicolumn{1}{c|}{\textbf{Kritik \& Saran}} \\ \hline \endhead
  \hline \endfoot

  1 & Alurnya cukup mudah untuk mengakses data & Informasi cukup & - \\ \hline
  2 & Cukup mengerti untuk membuat App Group & Tampilan cukup jelas, grafiknya sangat menjelaskan & - \\ \hline
  3 & Alurnya mudah & Informasi mirip dengan Dashboard & - \\ \hline
  4 & Alurnya cukup mudah & Informasi mirip dengan Dashboard, ada tambahan daftar aplikasi & - \\ \hline
  5 & Alurnya mudah untuk dimengerti & Informasi cukup untuk mengerti aplikasi hanya bisa dipakai beberapa saat, pengaturan waktu untuk pengingat mudah dimengerti & - \\ \hline
  6 & Cukup mudah untuk menunda & Informasi sederhana dan cukup & - \\ \hline
  7 & Cukup mudah dan enak untuk membuat jadwal & Tampilan daftar jadwal sederhana, fitur penjadwalan cukup informatif & - \\ \hline
  8 & Alurnya cukup jelas & Informasinya terus terang dan sangat simple & - \\ \hline
  9 & Cukup mudah untuk menambah goal & Informasi untuk membuat goal cukup, namun fitur pengingat kurang jelas maksudnya & Perlu penjelasan lebih untuk penyematan notifikasi Daily Goal ke fitur lain \\ \hline
  10 & Alurnya bisa diikuti tapi bingung tujuannya & Kurang jelas apa yang diingatkan & Perlu penjelasan lebih tentang yang dimaksud dengan "\textit{summary}" pada Daily Summary Notification \\ \hline
  11 & Alurnya cukup simple, tapi deskripsi di Main Menu kurang mengerti & Tampilan penjadwalan cukup, namun kemampuan kurang jelas & Kemampuan yang bisa aktif saat Bedtime Mode perlu penjelasan lebih, perlu terminologi lain untuk deskripsi di Main Menu \\ \hline
  12 & Bagus bisa navigasi langsung ke aplikasi hanya dengan menekan widget & Data 3 penggunaan tertinggi sangat cukup, ditambah screen time total & - \\ \hline
  13 & Bagus bisa navigas langsung ke aplikasi hanya dengan menekan widget & Mudah untuk melihat App Timer tanpa perlu masuk ke aplikasi & - \\ \hline
  14 & Alur untuk aktivasi simple dan mudah & Tampilan minimalis tapi cukup untuk widget & - \\ \hline


\end{longtable}
\end{footnotesize}

% \hline \rowcolor[HTML]{A3E5F5}
  % \multicolumn{4}{|c|}{\textbf{Partisipan 1}} \\
  % \hline \rowcolor[HTML]{DCF3FC}
  % \apdheadcell{Skenario Pengujian} & \apdheadcell{Tanggapan Alur} & \apdheadcell{Tanggapan Informasi} & \apdheadcell{Kritik \& Saran} \\ \hline \endfirsthead
\chapter{Rancangan \textit{Usability Testing} Prototipe \textit{High-Fidelity}}
\label{chpt:testing_hifi}

% Vars
\newlength{\coln}
\setlength{\coln}{0.02\textwidth}

% Functions
\newcommand{\apghead}[1]{\cellcolor[HTML]{A3E5F5}\textbf{#1}}
\newcommand{\apgheadcell}[1]{\multicolumn{1}{c|}{\apghead{#1}}}

\newcommand{\borderblue}{\arrayrulecolor[HTML]{A3E5F5}}
\newcommand{\borderblack}{\arrayrulecolor{black}}


\section{Aktivitas Pengujian}
Skenario dan \textit{task} pengujian mengacu pada Lampiran \ref{chpt:skenario_hifi1}.

\section{\textit{Post-task Questions}}

Setiap menyelesaikan sebuah task, partisian diminta untuk memberikan kesan terhadap \textit{task} yang dilakukan. Selain itu, dikumpulkan data-data berikut

\subsection{\textit{Single Ease Question} (SEQ)}

Setiap menyelesaikan sebuah \textit{task}, diberikan sebuah pertanyaan dengan \textit{likert scale} dari 1 sampai dengan 7, di mana 1 menunjukkan \textit{task} sangat sulit dan 7 menunjukkan \textit{task} sangat mudah. Tujuan pertanyaan ini adalah untuk mengukur tingkat kemudahan fitur untuk dipelajari dan digunakan pengguna. (\textit{learnability}). Selain itu, dicatat apakah berhasil dalam menyelesaikan \textit{task}-nya atau tidak. Pertanyaan yang ditanyakan adalah:

\begin{itemize}
  \item Bagaimana tingkat kemudahan yang Anda rasakan dalam melakukan task ini?
\end{itemize}


% \subsection{Pengukuran waktu pengerjaan task}
% Untuk setiap \textit{task}, dilakukan pengukuran waktu pengerjaan dari partisipan. Hal ini dilakukan untuk mengukur \textit{efficiency} dari fitur-fitur prototipe aplikasi. 


\section{\textit{Post-test questions}}

\subsection{\textit{System Usability Scale} (SUS)}
\label{subsec:sus}
Beri tanda centang pada nilai 1 sampai dengan 5. Nilai 1 menunjukkan Anda sangat tidak setuju dengan pernyataan, sedangkan nilai 5 menunjukkan Anda sangat setuju. Tujuan dari pertanyaan ini adalah untuk menguji \textit{usability} dari aplikasi.

\RaggedLeft
\begin{footnotesize}
\begin{longtable}[c]{|m{0.04\textwidth}|>{\baselineskip=8pt}m{0.57\textwidth}|>{\baselineskip=8pt}p{\coln}|>{\baselineskip=8pt}p{\coln}|>{\baselineskip=8pt}p{\coln}|>{\baselineskip=8pt}p{\coln}|>{\baselineskip=8pt}p{\coln}|}
  
  \hline
  
  \apghead{} & \apghead{} & \multicolumn{5}{c|}{\apghead{Nilai}} \\ \hhline{|>{\borderblue}->{\borderblack}|>{\borderblue}->{\borderblack}|*5{-}|}
  \rowcolor[HTML]{A3E5F5} \multicolumn{1}{|c|}{\multirow{-2}{*}{\apghead{No.}}} & \multicolumn{1}{c|}{\multirow{-2}{*}{\apghead{Pertanyaan}}} & \apgheadcell{1} & \apgheadcell{2} & \apgheadcell{3} & \apgheadcell{4} & \apgheadcell{5} \\ \hline
  \endfirsthead
  
  \hline
  \apghead{} & \apghead{} & \multicolumn{5}{c|}{\apghead{Nilai}} \\ \hhline{|>{\borderblue}->{\borderblack}|>{\borderblue}->{\borderblack}|*5{-}|}  
  \rowcolor[HTML]{A3E5F5} \multicolumn{1}{|c|}{\multirow{-2}{*}{\apghead{No.}}} & \multicolumn{1}{c|}{\multirow{-2}{*}{\apghead{Pertanyaan}}} & \apgheadcell{1} & \apgheadcell{2} & \apgheadcell{3} & \apgheadcell{4} & \apgheadcell{5} \\ \hline
  \endhead
  \hline \endfoot
  
  1. &  Saya rasa saya akan sering menggunakan aplikasi ini &  &  &  &  &  \\ \hline
  2. &  Saya rasa aplikasi ini terlalu rumit, padahal bisa lebih disederhanakan &  &  &  &  &  \\ \hline
  3. &  Saya rasa aplikasi mudah untuk digunakan  &  &  &  &  &  \\ \hline
  4. &  Saya rasa saya membutuhkan bantuan dari orang teknis untuk dapat menggunakan aplikasi ini  &  &  &  &  &  \\ \hline
  5. &  Saya menemukan bahwa terdapat berbagai macam fungsi yang terintegrasi dengan baik dalam aplikasi ini  &  &  &  &  &  \\ \hline
  6. &  Saya rasa terdapat banyak hal yang tidak konsisten pada aplikasi ini  &  &  &  &  &  \\ \hline
  7. &  Saya rasa mayoritas pengguna akan belajar menggunakan aplikasi ini dengan cepat  &  &  &  &  &  \\ \hline
  8. &  Saya menemukan bahwa aplikasi ini sangat tidak praktis &  &  &  &  &  \\ \hline
  9. &  Saya sangat percaya diri dalam menggunakan aplikasi ini  &  &  &  &  &  \\ \hline
  10. &  Saya harus belajar banyak hal terlebih dahulu sebelum saya dapat menggunakan aplikasi ini  &  &  &  &  &  \\ \hline


\end{longtable}
\end{footnotesize}
\justifying

\subsection{\textit{Intrinsic Motivation Inventory} (IMI)}
\label{subsec:imi}
Beri tanda centang pada nilai 1 sampai dengan 7. Nilai 1 menunjukkan Anda sangat tidak setuju dengan pernyataan, sedangkan nilai 7 menunjukkan Anda sangat setuju. Tujuan dari pertanyaan ini adalah untuk menguji aspek \textit{user experience} dari aplikasi, yaitu \textit{helpful} dan \textit{motivating}.
 

\RaggedLeft
\begin{footnotesize}
\begin{longtable}[c]{|m{0.04\textwidth}|>{\baselineskip=8pt}m{0.45\textwidth}|>{\baselineskip=8pt}p{\coln}|>{\baselineskip=8pt}p{\coln}|>{\baselineskip=8pt}p{\coln}|>{\baselineskip=8pt}p{\coln}|>{\baselineskip=8pt}p{\coln}|>{\baselineskip=8pt}p{\coln}|>{\baselineskip=8pt}p{\coln}|}
  
  \hline
  
  \apghead{} & \apghead{} & \multicolumn{7}{c|}{\apghead{Nilai}} \\ \hhline{|>{\borderblue}->{\borderblack}|>{\borderblue}->{\borderblack}|*7{-}|}
  \rowcolor[HTML]{A3E5F5} \multicolumn{1}{|c|}{\multirow{-2}{*}{\apghead{No.}}} & \multicolumn{1}{c|}{\multirow{-2}{*}{\apghead{Pertanyaan}}} & \apgheadcell{1} & \apgheadcell{2} & \apgheadcell{3} & \apgheadcell{4} & \apgheadcell{5} & \apgheadcell{6} & \apgheadcell{7} \\ \hline
  \endfirsthead
  
  \hline
  \apghead{} & \apghead{} & \multicolumn{7}{c|}{\apghead{Nilai}} \\ \hhline{|>{\borderblue}->{\borderblack}|>{\borderblue}->{\borderblack}|*7{-}|}  
  \rowcolor[HTML]{A3E5F5} \multicolumn{1}{|c|}{\multirow{-2}{*}{\apghead{No.}}} & \multicolumn{1}{c|}{\multirow{-2}{*}{\apghead{Pertanyaan}}} & \apgheadcell{1} & \apgheadcell{2} & \apgheadcell{3} & \apgheadcell{4} & \apgheadcell{5} & \apgheadcell{6} & \apgheadcell{7} \\ \hline
  \endhead
  \hline \endfoot
  
  \rowcolor[HTML]{DCF3FC} \multicolumn{9}{|l|}{\textbf{\textit{Value / Usefulness}}} \\ \hline
  1. & Saya merasa mengatur pencegahan distraksi berharga bagi saya &  &  &  &  &  &  &  \\ \hline
  2. & Saya pikir dengan mengatur penjadwalan pemblokiran aplikasi dapat berguna untuk membantu saya mencegah distraksi dari smartphone  &  &  &  &  &  &  &  \\ \hline
  3. & Saya rasa aplikasi ini penting untuk digunakan karena dapat membantu saya untuk mencegah distraksi dari smartphone  &  &  &  &  &  &  &  \\ \hline
  4. & Saya bersedia melakukan pencegahan distraksi lagi karena memberikan nilai bagi saya  &  &  &  &  &  &  &  \\ \hline
  5. & Saya pikir dengan mengatur fitur-fitur pada aplikasi ini dapat membantu saya mencegah distraksi dari smartphone  &  &  &  &  &  &  &  \\ \hline
  6. & Saya percaya dengan mengatur pencegahan distraksi dapat bermanfaat bagi saya  &  &  &  &  &  &  &  \\ \hline
  7. & Saya pikir mengatur pencegahan distraksi adalah kegiatan yang penting  &  &  &  &  &  &  &  \\ \hline
  
  \rowcolor[HTML]{DCF3FC} \multicolumn{9}{|l|}{\textbf{\textit{Interest / Enjoyment}}} \\ \hline
  1. & Saya sangat menikmati mengatur pencegahan distraksi pada aplikasi ini  &  &  &  &  &  &  &  \\ \hline
  2. & Mengatur pencegahan distraksi menyenangkan untuk dilakukan  &  &  &  &  &  &  &  \\ \hline
  3. & Saya pikir mengatur pencegahan distraksi adalah kegiatan yang membosankan  &  &  &  &  &  &  &  \\ \hline
  4. & Mengatur pencegahan distraksi tidak menarik perhatian saya  &  &  &  &  &  &  &  \\ \hline
  5. & Saya dapat mendeskripsikan mengatur pencegahan distraksi sebagai sangat menarik  &  &  &  &  &  &  &  \\ \hline
  6. & Saya rasa mengatur pencegah distraksi cukup menyenangkan &  &  &  &  &  &  &  \\ \hline
  7. & Ketika saya melakukan pencegahan distraksi, saya memikirkan betapa saya menikmatinya  &  &  &  &  &  &  &  \\ \hline
  
  \rowcolor[HTML]{DCF3FC} \multicolumn{9}{|l|}{\textbf{\textit{Pressure / Tension}}} \\ \hline
  1. & Saya tidak merasa gugup sama sekali selama mengatur pencegahan distraksi &  &  &  &  &  &  &  \\ \hline
  2. & Saya merasa sangat tegang selama mengatur pencegahan distraksi  &  &  &  &  &  &  &  \\ \hline
  3. & Saya merasa santai selama mengatur pencegahan distraksi  &  &  &  &  &  &  &  \\ \hline
  4. & Saya merasa cemas selama mengatur pencegahan distraksi  &  &  &  &  &  &  &  \\ \hline
  5. & Saya merasa tertekan selama mengatur pencegahan distraksi  &  &  &  &  &  &  &  \\ \hline


\end{longtable}
\end{footnotesize}
\justifying
\chapter{Hasil \textit{Usability Testing} Prototipe \textit{High-Fidelity} Iterasi Pertama}
\label{chpt:hasil_test_hifi1}

\section{Rangkuman Hasil SUS}
% \normalsize

Jumlah partisipan adalah 5 orang.
Pertanyaan kuesioner SUS mengacu pada Lampiran \ref{subsec:sus}.

\begin{figure}[h]
  \centering
  \includegraphics[width=0.7\textwidth]{hifi/chart-sus.png}
\end{figure}


\section{Konversi Penghitungan Skor SUS dan SEQ}
\normalsize
Berikut adalah hasil konversi perhitungan skor SUS dan skor SEQ. Daftar \textit{task} mengacu pada Lampiran \ref{chpt:skenario_hifi1}.

\begin{figure}[h]
  \centering
  \includegraphics[width=\textwidth]{hifi/chart-sus-seq.png}
\end{figure}

% \section{Waktu Penyelesaian Task}
% \normalsize

% Berikut adalah rangkuman waktu yang diperlukan setiap partisipan untuk menyelesaikan \textit{task} yang diberikan. Daftar \textit{task} mengacu pada Lampiran \ref{chpt:skenario_hifi1}.

% \begin{figure}[h]
%   \centering
%   \includegraphics[width=\textwidth]{hifi/chart-penyelesaian.png}
% \end{figure}

\section{Konversi Perhitungan Skor IMI}
\normalsize

Berikut adalah rata-rata nilai 3 sub skala \textit{Intrinsic Motivation Inventory}. Daftar pertanyaan pada setiap sub skala IMI mengacu pada Lampiran \ref{subsec:imi}.

\begin{figure}[h]
  \centering
  \includegraphics[width=0.95\textwidth]{hifi/chart-imi.png}
\end{figure}
\chapter{Hasil \textit{Usability Testing} Prototipe \textit{High-Fidelity} Iterasi Kedua}
\label{chpt:hasil_test_hifi2}

\section{Rangkuman Hasil SUS}
\normalsize

Jumlah partisipan adalah 5 orang.
Pertanyaan kuesioner SUS mengacu pada Lampiran \ref{subsec:sus}.

\begin{figure}[h]
  \centering
  \includegraphics[width=0.7\textwidth]{hifi2/chart-sus.png}
\end{figure}


\section{Konversi Penghitungan Skor SUS dan SEQ}
\normalsize
Berikut adalah hasil konversi perhitungan skor SUS dan skor SEQ. Daftar \textit{task} mengacu pada Lampiran \ref{chpt:skenario_hifi1}.

\begin{figure}[h]
  \centering
  \includegraphics[width=\textwidth]{hifi2/chart-sus-seq.png}
\end{figure}

% \section{Waktu Penyelesaian Task}
% \normalsize

% Berikut adalah rangkuman waktu yang diperlukan setiap partisipan untuk menyelesaikan \textit{task} yang diberikan. Daftar \textit{task} mengacu pada Lampiran \ref{chpt:skenario_hifi1}.

% \begin{figure}[h]
%   \centering
%   \includegraphics[width=\textwidth]{hifi2/chart-penyelesaian.png}
% \end{figure}

\section{Konversi Perhitungan Skor IMI}
\normalsize

Berikut adalah rata-rata nilai 3 sub skala \textit{Intrinsic Motivation Inventory}. Daftar pertanyaan pada setiap sub skala IMI mengacu pada Lampiran \ref{subsec:imi}.

\begin{figure}[h]
  \centering
  \includegraphics[width=0.95\textwidth]{hifi2/chart-imi.png}
\end{figure}

\end{document}
