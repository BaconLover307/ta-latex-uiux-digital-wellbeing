\chapter{Rancangan \textit{Usability Testing} Prototipe \textit{High-Fidelity}}
\label{chpt:testing_hifi}

% Vars
\newlength{\coln}
\setlength{\coln}{0.02\textwidth}

% Functions
\newcommand{\apghead}[1]{\cellcolor[HTML]{A3E5F5}\textbf{#1}}
\newcommand{\apgheadcell}[1]{\multicolumn{1}{c|}{\apghead{#1}}}

\newcommand{\borderblue}{\arrayrulecolor[HTML]{A3E5F5}}
\newcommand{\borderblack}{\arrayrulecolor{black}}


\large \section{Aktivitas Pengujian}
\normalsize
Skenario dan \textit{task} pengujian mengacu pada Lampiran \ref{chpt:skenario_hifi1}.

\large \section{\textit{Post-task Questions}}
\normalsize

Setiap menyelesaikan sebuah task, partisian diminta untuk memberikan kesan terhadap \textit{task} yang dilakukan. Selain itu, dikumpulkan data-data berikut

\subsection{\textit{Single Ease Question} (SEQ)}

Setiap menyelesaikan sebuah \textit{task}, diberikan sebuah pertanyaan dengan \textit{likert scale} dari 1 sampai dengan 7, di mana 1 menunjukkan \textit{task} sangat sulit dan 7 menunjukkan \textit{task} sangat mudah. Tujuan pertanyaan ini adalah untuk mengukur tingkat kemudahan fitur untuk dipelajari dan digunakan pengguna. (\textit{learnability}). Selain itu, dicatat apakah berhasil dalam menyelesaikan \textit{task}-nya atau tidak. Pertanyaan yang ditanyakan adalah:

\begin{itemize}
  \item Bagaimana tingkat kemudahan yang Anda rasakan dalam melakukan task ini?
\end{itemize}


% \subsection{Pengukuran waktu pengerjaan task}
% Untuk setiap \textit{task}, dilakukan pengukuran waktu pengerjaan dari partisipan. Hal ini dilakukan untuk mengukur \textit{efficiency} dari fitur-fitur prototipe aplikasi. 


\large \section{\textit{Post-test questions}}
\normalsize


\subsection{\textit{System Usability Scale} (SUS)}
\label{subsec:sus}
Beri tanda centang pada nilai 1 sampai dengan 5. Nilai 1 menunjukkan Anda sangat tidak setuju dengan pernyataan, sedangkan nilai 5 menunjukkan Anda sangat setuju. Tujuan dari pertanyaan ini adalah untuk menguji \textit{usability} dari aplikasi.

\RaggedLeft
\begin{footnotesize}
\begin{longtable}[c]{|m{0.04\textwidth}|>{\baselineskip=8pt}m{0.57\textwidth}|>{\baselineskip=8pt}p{\coln}|>{\baselineskip=8pt}p{\coln}|>{\baselineskip=8pt}p{\coln}|>{\baselineskip=8pt}p{\coln}|>{\baselineskip=8pt}p{\coln}|}
  
  \hline
  
  \apghead{} & \apghead{} & \multicolumn{5}{c|}{\apghead{Nilai}} \\ \hhline{|>{\borderblue}->{\borderblack}|>{\borderblue}->{\borderblack}|*5{-}|}
  \rowcolor[HTML]{A3E5F5} \multicolumn{1}{|c|}{\multirow{-2}{*}{\apghead{No.}}} & \multicolumn{1}{c|}{\multirow{-2}{*}{\apghead{Pertanyaan}}} & \apgheadcell{1} & \apgheadcell{2} & \apgheadcell{3} & \apgheadcell{4} & \apgheadcell{5} \\ \hline
  \endfirsthead
  
  \hline
  \apghead{} & \apghead{} & \multicolumn{5}{c|}{\apghead{Nilai}} \\ \hhline{|>{\borderblue}->{\borderblack}|>{\borderblue}->{\borderblack}|*5{-}|}  
  \rowcolor[HTML]{A3E5F5} \multicolumn{1}{|c|}{\multirow{-2}{*}{\apghead{No.}}} & \multicolumn{1}{c|}{\multirow{-2}{*}{\apghead{Pertanyaan}}} & \apgheadcell{1} & \apgheadcell{2} & \apgheadcell{3} & \apgheadcell{4} & \apgheadcell{5} \\ \hline
  \endhead
  \hline \endfoot
  
  1. &  Saya rasa saya akan sering menggunakan aplikasi ini &  &  &  &  &  \\ \hline
  2. &  Saya rasa aplikasi ini terlalu rumit, padahal bisa lebih disederhanakan &  &  &  &  &  \\ \hline
  3. &  Saya rasa aplikasi mudah untuk digunakan  &  &  &  &  &  \\ \hline
  4. &  Saya rasa saya membutuhkan bantuan dari orang teknis untuk dapat menggunakan aplikasi ini  &  &  &  &  &  \\ \hline
  5. &  Saya menemukan bahwa terdapat berbagai macam fungsi yang terintegrasi dengan baik dalam aplikasi ini  &  &  &  &  &  \\ \hline
  6. &  Saya rasa terdapat banyak hal yang tidak konsisten pada aplikasi ini  &  &  &  &  &  \\ \hline
  7. &  Saya rasa mayoritas pengguna akan belajar menggunakan aplikasi ini dengan cepat  &  &  &  &  &  \\ \hline
  8. &  Saya menemukan bahwa aplikasi ini sangat tidak praktis &  &  &  &  &  \\ \hline
  9. &  Saya sangat percaya diri dalam menggunakan aplikasi ini  &  &  &  &  &  \\ \hline
  10. &  Saya harus belajar banyak hal terlebih dahulu sebelum saya dapat menggunakan aplikasi ini  &  &  &  &  &  \\ \hline


\end{longtable}
\end{footnotesize}
\justifying

\subsection{\textit{Intrinsic Motivation Inventory} (IMI)}
\label{subsec:imi}
Beri tanda centang pada nilai 1 sampai dengan 7. Nilai 1 menunjukkan Anda sangat tidak setuju dengan pernyataan, sedangkan nilai 7 menunjukkan Anda sangat setuju. Tujuan dari pertanyaan ini adalah untuk menguji aspek \textit{user experience} dari aplikasi, yaitu \textit{helpful} dan \textit{motivating}.
 

\RaggedLeft
\begin{footnotesize}
\begin{longtable}[c]{|m{0.04\textwidth}|>{\baselineskip=8pt}m{0.45\textwidth}|>{\baselineskip=8pt}p{\coln}|>{\baselineskip=8pt}p{\coln}|>{\baselineskip=8pt}p{\coln}|>{\baselineskip=8pt}p{\coln}|>{\baselineskip=8pt}p{\coln}|>{\baselineskip=8pt}p{\coln}|>{\baselineskip=8pt}p{\coln}|}
  
  \hline
  
  \apghead{} & \apghead{} & \multicolumn{7}{c|}{\apghead{Nilai}} \\ \hhline{|>{\borderblue}->{\borderblack}|>{\borderblue}->{\borderblack}|*7{-}|}
  \rowcolor[HTML]{A3E5F5} \multicolumn{1}{|c|}{\multirow{-2}{*}{\apghead{No.}}} & \multicolumn{1}{c|}{\multirow{-2}{*}{\apghead{Pertanyaan}}} & \apgheadcell{1} & \apgheadcell{2} & \apgheadcell{3} & \apgheadcell{4} & \apgheadcell{5} & \apgheadcell{6} & \apgheadcell{7} \\ \hline
  \endfirsthead
  
  \hline
  \apghead{} & \apghead{} & \multicolumn{7}{c|}{\apghead{Nilai}} \\ \hhline{|>{\borderblue}->{\borderblack}|>{\borderblue}->{\borderblack}|*7{-}|}  
  \rowcolor[HTML]{A3E5F5} \multicolumn{1}{|c|}{\multirow{-2}{*}{\apghead{No.}}} & \multicolumn{1}{c|}{\multirow{-2}{*}{\apghead{Pertanyaan}}} & \apgheadcell{1} & \apgheadcell{2} & \apgheadcell{3} & \apgheadcell{4} & \apgheadcell{5} & \apgheadcell{6} & \apgheadcell{7} \\ \hline
  \endhead
  \hline \endfoot
  
  \rowcolor[HTML]{DCF3FC} \multicolumn{9}{|l|}{\textbf{\textit{Value / Usefulness}}} \\ \hline
  1. & Saya merasa mengatur pencegahan distraksi berharga bagi saya &  &  &  &  &  &  &  \\ \hline
  2. & Saya pikir dengan mengatur penjadwalan pemblokiran aplikasi dapat berguna untuk membantu saya mencegah distraksi dari smartphone  &  &  &  &  &  &  &  \\ \hline
  3. & Saya rasa aplikasi ini penting untuk digunakan karena dapat membantu saya untuk mencegah distraksi dari smartphone  &  &  &  &  &  &  &  \\ \hline
  4. & Saya bersedia melakukan pencegahan distraksi lagi karena memberikan nilai bagi saya  &  &  &  &  &  &  &  \\ \hline
  5. & Saya pikir dengan mengatur fitur-fitur pada aplikasi ini dapat membantu saya mencegah distraksi dari smartphone  &  &  &  &  &  &  &  \\ \hline
  6. & Saya percaya dengan mengatur pencegahan distraksi dapat bermanfaat bagi saya  &  &  &  &  &  &  &  \\ \hline
  7. & Saya pikir mengatur pencegahan distraksi adalah kegiatan yang penting  &  &  &  &  &  &  &  \\ \hline
  
  \rowcolor[HTML]{DCF3FC} \multicolumn{9}{|l|}{\textbf{\textit{Interest / Enjoyment}}} \\ \hline
  1. & Saya sangat menikmati mengatur pencegahan distraksi pada aplikasi ini  &  &  &  &  &  &  &  \\ \hline
  2. & Mengatur pencegahan distraksi menyenangkan untuk dilakukan  &  &  &  &  &  &  &  \\ \hline
  3. & Saya pikir mengatur pencegahan distraksi adalah kegiatan yang membosankan  &  &  &  &  &  &  &  \\ \hline
  4. & Mengatur pencegahan distraksi tidak menarik perhatian saya  &  &  &  &  &  &  &  \\ \hline
  5. & Saya dapat mendeskripsikan mengatur pencegahan distraksi sebagai sangat menarik  &  &  &  &  &  &  &  \\ \hline
  6. & Saya rasa mengatur pencegah distraksi cukup menyenangkan &  &  &  &  &  &  &  \\ \hline
  7. & Ketika saya melakukan pencegahan distraksi, saya memikirkan betapa saya menikmatinya  &  &  &  &  &  &  &  \\ \hline
  
  \rowcolor[HTML]{DCF3FC} \multicolumn{9}{|l|}{\textbf{\textit{Pressure / Tension}}} \\ \hline
  1. & Saya tidak merasa gugup sama sekali selama mengatur pencegahan distraksi &  &  &  &  &  &  &  \\ \hline
  2. & Saya merasa sangat tegang selama mengatur pencegahan distraksi  &  &  &  &  &  &  &  \\ \hline
  3. & Saya merasa santai selama mengatur pencegahan distraksi  &  &  &  &  &  &  &  \\ \hline
  4. & Saya merasa cemas selama mengatur pencegahan distraksi  &  &  &  &  &  &  &  \\ \hline
  5. & Saya merasa tertekan selama mengatur pencegahan distraksi  &  &  &  &  &  &  &  \\ \hline


\end{longtable}
\end{footnotesize}
\justifying