\chapter*{\MakeUppercase{Kata Pengantar}}
\addcontentsline{toc}{chapter}{KATA PENGANTAR}

Puji syukur penulis ucapkan kepada Tuhan Yang Maha Esa atas berkat dan rahmat-Nya, sehingga penulis dapat menyelesaikan laporan penelitian tugas akhir ini yang berjudul "{\thetitle}". Keberhasilan penulis dalam menyelesaikan laporan ini tidak terlepas dari dukungan dan bantuan dari berbagai pihak. Untuk itu, penulis mengucapkan terima kasih kepada:

\begin{enumerate}
  \item Bapak Adi Mulyanto, S.T, M.T., selaku dosen pembimbing tugas akhir atas bimbingan dan masukan yang diberikan selama masa pengerjaan tugas akhir.
   
  % \item ... selaku dosen penguji tugas akhir yang telah memberikan saran, kritik, dan rekomendasi 
   
  \item Bapak Dicky Prima Satya, S.T., M.T., Bapak Adi Mulyanto, S.T., M.T., Ibu Latifa Dwiyanti, S.T., M.T., dan Bapak Nugraha Priya Utama, S.T., M.A., Ph.D., selaku koordinator tim tugas akhir yang memberikan arahan dalam pengerjaan tugas akhir ini,

  \item Ibu Dessi Puji Lestari, S.T., M.Eng., Ph.D., selaku Ketua Program Studi Teknik Informatika Institut Teknologi Bandung,
   
  \item Bapak Dr. Techn. Muhammad Zuhri Catur Candra, S.T, M.T. dan Ibu Ginar Santika Niwanputri, S.T, M.Sc., selaku dosen wali yang telah membimbing penulis selama berkuliah tiga tahun di Teknik Informatika,
  
  \item Bapak Yudistira Dwi Wardhana Asnar, S.T., Ph.D. yang telah memberikan ide dan inspirasi mengenai topik dari tugas akhir yang dikerjakan, 
  
  \item Para karyawan dan staff dari TU STEI yang telah membantu kebutuhan administrasi dalam menyelesaikan laporan Tugas Akhir,
  
  \item Orang tua, kakak, dan keluarga penulis yang senantiasa memberikan dukungan dan semangat selama menjalani perkuliahan di Teknik Informatika,

  \item Hollyana Puteri Haryono, selaku rekan terdekat penulis yang bersedia menemani serta memberikan semangat dan inspirasi dalam mengerjakan tugas akhir ini,
  
  % \item Teman-teman dari grup "Bunker": Naufal "Bapak" Prima, Michel "Peng" Fang, Junho Choi "Oppa" Hedyatmo, Jonathan "Jojo" Yudi, Matthew "Mek" Kevin, Kamal "Mastree" Shafi, Garry "Geri" Kuwanto, Morgen "Koh" Sudyanto, Muhammad "Euy" Hasan, Fabian "God" Zhafransyah, Reyvan "Berayfun" Rizky, Mario "Margun" Gunawan, Vincent "Lie" Lienardo, Faris "KissShot" Kautsar, Farras Hibban, Nafkhan "Camcam" Alzamzami, Fauzan "Kakek" Rafi, dan Naufal Dean yang telah memberi dukungan, bantuan, dan kata-kata pedas kepada penulis serta memberi hiburan  dengan menjadi teman-teman terbaik dalam masa perkuliahan.
  
  \item Teman-teman dari grup "Bunker": Naufal Prima, Michel Fang, Junho Choi, Jonathan Yudi, Matthew Kevin, Kamal Shafi, Garry Kuwanto, Morgen Sudyanto, Muhammad Hasan, Fabian Zhafransyah, Reyvan Rizky, Mario Gunawan, Vincentius Lienardo, Faris Kautsar, Farras Hibban, Camcam, Fauzan Rafi, dan Naufal Dean, yang telah memberi dukungan, bantuan, dan kata-kata pedas kepada penulis untuk mendorong penulis dalam mengerjakan tugas akhir, serta memberi hiburan dengan menjadi teman-teman terbaik dalam masa perkuliahan,
  
  \item Teman sebimbingan yang bersedia untuk saling menyemangati dan memberikan pendapat, kritik, saran terkait tugas akhir penulis,
  
  \item Teman-teman dari grup "Latex TA pipel", terutama Faris Rizki Ekananda, yang telah membantu penulis dalam mengatur format dokumen laporan tugas akhir,
  
  \item Gitta, Gian, dan teman-teman dari FNFBandung yang senantiasa menjadi penghibur dan penjaga kesehatan fisik maupun mental penulis selama pengerjaan tugas akhir,
  
  \item Para narasumber yang bersedia untuk membantu penulis sebagai partisipan pengumpulan data dan pengujian,

  \item Teman-teman mahasiswa Program Studi Teknik Informatika ITB 2018 yang senantiasa memberikan dukungan dan semangat dalam pengerjaan Tugas akhir,
   
  \item Pihak-pihak lain yang tidak dapat disebutkan satu per satu yang turut serta membantu penulis untuk menyelesaikan Tugas Akhir.
   
\end{enumerate}

Akhir kata, semoga penelitian tugas akhir ini dapat bermanfaat bagi semua pihak yang membutuhkannya

\vspace{15mm}

\begin{flushright}
  Bandung, 02 September 2022 \\
  \vspace{2.5cm}
  Penulis
\end{flushright}
\vfill


% Gunakan bagian ini untuk memberikan ucapan terima kasih kepada semua pihak yang secara langsung atau tidak langsung membantu penyelesaian tugas akhir, termasuk pemberi beasiswa jika ada. Utamakan untuk memberikan ucapan terima kasih kepada tim pembimbing tugas akhir dan staf pengajar atau pihak program studi, bahkan sebelum mengucapkan terima kasih kepada keluarga. Ucapan terima kasih sebaiknya bukan hanya menyebutkan nama orang saja, tetapi juga memberikan penjelasan bagaimana bentuk bantuan/dukungan yang diberikan. Gunakan bahasa yang baik dan sopan serta memberikan kesan yang enak untuk dibaca. Sebagai contoh: “Tidak lupa saya ucapkan terima kasih kepada teman dekat saya, Tito, yang sejak satu tahun terakhir ini selalu memberikan semangat dan mengingatkan saya apabila lengah dalam mengerjakan Tugas Akhir ini. Tito juga banyak membantu mengoreksi format dan layout tulisan. Apresiasi saya sampaikan kepada pemberi beasiswa, Yayasan Beasiswa, yang telah memberikan bantuan dana kuliah dan biaya hidup selama dua tahun. Bantuan dana tersebut sangat membantu saya untuk dapat lebih fokus dalam menyelesaikan pendidikan saya. ....”. Ucapan permintaan maaf karena kekurangsempurnaan hasil Tugas Akhir tidak perlu ditulis.
