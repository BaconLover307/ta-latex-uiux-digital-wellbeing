\clearpage
\chapter*{ABSTRACT}
\addcontentsline{toc}{chapter}{Abstract}

\begin{center}
  \textbf{\MakeUppercase{Interaction Design of Google Digital Wellbeing Application using User-Centered Design Approach}} \\[1em]
  
  By: \\
  \MakeUppercase{\theauthor} \\

\end{center}

\begin{singlespace}
  % $ latar belakang
  The increasing use of smartphones is negatively affecting the digital well-being of its users. Excessive use of smartphone can lead users to have high dependency to the point of addiction. This case raises the urgency for research in the field of Human Computer Interaction on the intentionality of not using technology. This research led to the concept of Digital Wellbeing, which Google has adopted to develop an app with such concept. However, it was found that the app had several problems according to complaints in the app reviews regarding the lack of utility provided.

  % $ proses penelitian
  Therefore, it is necessary to develop the interaction design for an application that can overcome the problems of the Google Digital Wellbeing application. The design process uses the User-Centered Design methodology. Data collection begins with analyzing reviews of the Digital Wellbeing application, then completed with interviews with application users. The result of the final project is a high-fidelity prototype of the application for Android mobile device display. The interaction design is designed by prioritizing usability goals of utility and learnability, and directing to user experience goals of helpful and motivating.

  % $ hasil penelitian
  Usability testing is done with targeted users in accordance with the specified personas, using measurement metrics such as SUS, SEQ, and IMI with Value/Usefulness, Interest/Enjoyment, and Pressure/Tension subscales to measure the achievement of goals. The test results showed that the prototype successfully solved the problems found and achieved the expected goals. It was concluded that the Search bar, App Group, and Activation Schedule features of the prototype play a major role in improving the utility of the application, and the Daily Goal feature is a leading feature in increasing user motivation to improve their digital habits.

  \noindent \textbf{Keywords:} Digital Wellbeing, interaction design, user-centered design, prototype, widget

\end{singlespace}
