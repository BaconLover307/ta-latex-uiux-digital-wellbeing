\clearpage
\chapter*{ABSTRAK}
\addcontentsline{toc}{chapter}{ABSTRAK}

\begin{center}
  \textbf{\MakeUppercase{\thetitle}} \\[1em]
  
  Oleh: \\
  \MakeUppercase{\theauthor} \\
\end{center}

\begin{singlespace}
  % $ latar belakang
  Penggunaan \textit{smartphone} yang semakin tinggi mempengaruhi kesejahteraan digital dari penggunanya secara negatif.
  Salah satu penyebabnya adalah distraksi terkait \textit{smartphone}, baik dari sisi \textit{smartphone} atau keinginan pengguna untuk memakainya.
  Distraksi berlebihan mampu menimbulkan ketergantungan pada \textit{smartphone} hingga dapat mencapai tingkat adiksi.
  Kasus ini memunculkan urgensi atas penelitian di bidang \textit{Human Computer Interaction} tentang kesengajaan untuk tidak menggunakan teknologi.
  Penelitian ini memunculkan konsep \textit{Digital Wellbeing} yang diadopsi Google untuk mengembangkan sebuah aplikasi pencegah distraksi.
  Namun ditemukan bahwa aplikasi tersebut memiliki beberapa masalah yang tercerminkan pada banyaknya keluhan pada ulasan aplikasi.
  
  % $ proses penelitian
  Oleh karena itu, diperlukan sebuah desain interaksi aplikasi pencegah distraksi yang dapat mengatasi masalah-masalah dari aplikasi Digital Wellbeing.
  Proses perancangan menggunakan metodologi \textit{User-Centered Design}.
  Pengumpulan data diawali dengan menganalisis ulasan dari aplikasi Digital Wellbeing, kemudian dilengkapi dengan wawancara kepada pengguna aplikasi.
  Hasil tugas akhir berupa prototipe \textit{high-fidelity} aplikasi untuk tampilan perangkat \textit{mobile} Android.
  Desain interaksi memprioritaskan \textit{usability goals efficiency} dan \textit{learnability}, serta mengarahkan kepada \textit{user experience goals helpful} dan \textit{motivating}.
  
  % $ hasil penelitian
  Pengujian dilakukan dengan \textit{usability testing} kepada target pengguna yang sesuai dengan persona yang ditentukan, menggunakan metrik pengukuran SUS, SEQ, dan IMI untuk subskala \textit{Value/Usefulness}, \textit{Interest/Enjoyment}, dan \textit{Pressure/Tension} untuk mengukur ketercapaian \textit{goals}, serta opini partisipan mengenai perubahan dari aplikasi Digital Wellbeing.
  Hasil pengujian menunjukkan bahwa prototipe berhasil menyelesaikan masalah-masalah yang ditemukan, serta mencapai \textit{goals} yang diharapkan.
  Dari pengujian disimpulkan bahwa fitur Search bar, App Group, dan Daftar Jadwal Aktivasi dari prototipe berperan besar dalam meningkatkan efisiensi aplikasi, serta fitur Daily Goal menjadi fitur unggulan dalam meningkatkan motivasi pengguna dalam mencegah distraksi.

\noindent \textbf{Kata kunci:} \textit{Digital Wellbeing}, pencegah distraksi, desain interaksi, \textit{user-centered design}, prototipe, \textit{widget}
\end{singlespace}